\documentclass{openreader}
\title{Πρὸς Κορινθίους αʹ}
\date{}
\setmainlanguage{english}
\setmainfont{Charis SIL}
\setotherlanguage{greek}
\newfontfamily\greekfont[Script=Greek]{SBL BibLit}
\FootnoteStyle{lettered-by-verse}
\begin{document}
\ORBselectlanguage{greek}
\maketitle
\raggedbottom 
\fontsize{16pt}{24pt}\selectfont

\ChapterHeader{1}{Chapter 1}

\MainTextVerseMark{1}{1}Παῦλος κλητὸς\FnParseFormGloss{a}{nominative masculine singular}{κλητός}{called} ἀπόστολος ⸂Χριστοῦ Ἰησοῦ⸃ διὰ θελήματος θεοῦ καὶ Σωσθένης\FnParseFormGloss{b}{nominative masculine singular}{Σωσθένης}{Sosthenes} ὁ ἀδελφὸς 
\MainTextVerseMark{1}{2}τῇ ἐκκλησίᾳ τοῦ θεοῦ, ⸂ἡγιασμένοις\FnParseFormGloss{a}{perfect passive participle dative masculine plural}{ἁγιάζω}{to sanctify} ἐν Χριστῷ Ἰησοῦ, τῇ οὔσῃ\FnParse{b}{present active participle dative feminine singular} ἐν Κορίνθῳ,⸃\FnParseFormGloss{c}{dative feminine singular}{Κόρινθος}{Corinth} κλητοῖς\FnParseFormGloss{d}{dative masculine plural}{κλητός}{called} ἁγίοις, σὺν πᾶσιν τοῖς ἐπικαλουμένοις\FnParse{e}{present middle participle dative masculine plural} τὸ ὄνομα τοῦ κυρίου ἡμῶν Ἰησοῦ Χριστοῦ ἐν παντὶ τόπῳ ⸀αὐτῶν καὶ ἡμῶν· 
\MainTextVerseMark{1}{3}χάρις ὑμῖν καὶ εἰρήνη ἀπὸ θεοῦ πατρὸς ἡμῶν καὶ κυρίου Ἰησοῦ Χριστοῦ. 
\MainTextVerseMark{1}{4}Εὐχαριστῶ\FnParse{a}{present active indicative 1st singular} τῷ θεῷ ⸀μου πάντοτε περὶ ὑμῶν ἐπὶ τῇ χάριτι τοῦ θεοῦ τῇ δοθείσῃ\FnParse{b}{aorist passive participle dative feminine singular} ὑμῖν ἐν Χριστῷ Ἰησοῦ, 
\MainTextVerseMark{1}{5}ὅτι ἐν παντὶ ἐπλουτίσθητε\FnParseFormGloss{a}{aorist passive indicative 2nd plural}{πλουτίζω}{to make rich} ἐν αὐτῷ, ἐν παντὶ λόγῳ καὶ πάσῃ γνώσει,\FnParseFormGloss{b}{dative feminine singular}{γνῶσις}{knowledge} 
\MainTextVerseMark{1}{6}καθὼς τὸ μαρτύριον\FnParseFormGloss{a}{nominative neuter singular}{μαρτύριον}{testimony} τοῦ Χριστοῦ ἐβεβαιώθη\FnParseFormGloss{b}{aorist passive indicative 3rd singular}{βεβαιόω}{to confirm} ἐν ὑμῖν, 
\MainTextVerseMark{1}{7}ὥστε ὑμᾶς μὴ ὑστερεῖσθαι\FnParseFormGloss{a}{present passive infinitive}{ὑστερέω}{to lack} ἐν μηδενὶ χαρίσματι,\FnParseFormGloss{b}{dative neuter singular}{χάρισμα}{gracious gift} ἀπεκδεχομένους\FnParseFormGloss{c}{present middle participle accusative masculine plural}{ἀπεκδέχομαι}{wait eagerly for} τὴν ἀποκάλυψιν\FnParseFormGloss{d}{accusative feminine singular}{ἀποκάλυψις}{revelation} τοῦ κυρίου ἡμῶν Ἰησοῦ Χριστοῦ· 
\MainTextVerseMark{1}{8}ὃς καὶ βεβαιώσει\FnParseFormGloss{a}{future active indicative 3rd singular}{βεβαιόω}{to confirm} ὑμᾶς ἕως τέλους ἀνεγκλήτους\FnParseFormGloss{b}{accusative masculine plural}{ἀνέγκλητος}{blameless} ἐν τῇ ἡμέρᾳ τοῦ κυρίου ἡμῶν Ἰησοῦ Χριστοῦ. 
\MainTextVerseMark{1}{9}πιστὸς ὁ θεὸς δι’ οὗ ἐκλήθητε\FnParse{a}{aorist passive indicative 2nd plural} εἰς κοινωνίαν\FnParseFormGloss{b}{accusative feminine singular}{κοινωνία}{fellowship} τοῦ υἱοῦ αὐτοῦ Ἰησοῦ Χριστοῦ τοῦ κυρίου ἡμῶν. 
\MainTextVerseMark{1}{10}Παρακαλῶ\FnParse{a}{present active indicative 1st singular} δὲ ὑμᾶς, ἀδελφοί, διὰ τοῦ ὀνόματος τοῦ κυρίου ἡμῶν Ἰησοῦ Χριστοῦ ἵνα τὸ αὐτὸ λέγητε\FnParse{b}{present active subjunctive 2nd plural} πάντες, καὶ μὴ ᾖ\FnParse{c}{present active subjunctive 3rd singular} ἐν ὑμῖν σχίσματα,\FnParseFormGloss{d}{nominative neuter plural}{σχίσμα}{tear} ἦτε\FnParse{e}{present active subjunctive 2nd plural} δὲ κατηρτισμένοι\FnParseFormGloss{f}{perfect passive participle nominative masculine plural}{καταρτίζω}{to restore} ἐν τῷ αὐτῷ νοῒ\FnParseFormGloss{g}{dative masculine singular}{νοῦς}{mind} καὶ ἐν τῇ αὐτῇ γνώμῃ.\FnParseFormGloss{h}{dative feminine singular}{γνώμη}{purpose} 
\MainTextVerseMark{1}{11}ἐδηλώθη\FnParseFormGloss{a}{aorist passive indicative 3rd singular}{δηλόω}{to make clear} γάρ μοι περὶ ὑμῶν, ἀδελφοί μου, ὑπὸ τῶν Χλόης\FnParseFormGloss{b}{genitive feminine singular}{Χλόη}{Chloe} ὅτι ἔριδες\FnParseFormGloss{c}{nominative feminine plural}{ἔρις}{quarrel} ἐν ὑμῖν εἰσιν.\FnParse{d}{present active indicative 3rd plural} 
\MainTextVerseMark{1}{12}λέγω\FnParse{a}{present active indicative 1st singular} δὲ τοῦτο ὅτι ἕκαστος ὑμῶν λέγει·\FnParse{b}{present active indicative 3rd singular} Ἐγὼ μέν εἰμι\FnParse{c}{present active indicative 1st singular} Παύλου, Ἐγὼ δὲ Ἀπολλῶ,\FnParseFormGloss{d}{genitive masculine singular}{Ἀπολλῶς}{Apollos} Ἐγὼ δὲ Κηφᾶ,\FnParseFormGloss{e}{genitive masculine singular}{Κηφᾶς}{Cephas} Ἐγὼ δὲ Χριστοῦ. 
\MainTextVerseMark{1}{13}μεμέρισται\FnParseFormGloss{a}{perfect passive indicative 3rd singular}{μερίζω}{to give} ὁ Χριστός; μὴ Παῦλος ἐσταυρώθη\FnParse{b}{aorist passive indicative 3rd singular} ὑπὲρ ὑμῶν, ἢ εἰς τὸ ὄνομα Παύλου ἐβαπτίσθητε;\FnParse{c}{aorist passive indicative 2nd plural} 
\MainTextVerseMark{1}{14}⸀εὐχαριστῶ\FnParse{a}{present active indicative 1st singular} ὅτι οὐδένα ὑμῶν ἐβάπτισα\FnParse{b}{aorist active indicative 1st singular} εἰ μὴ Κρίσπον\FnParseFormGloss{c}{accusative masculine singular}{Κρίσπος}{Crispus} καὶ Γάϊον,\FnParseFormGloss{d}{accusative masculine singular}{Γάϊος}{Gaius} 
\MainTextVerseMark{1}{15}ἵνα μή τις εἴπῃ\FnParse{a}{aorist active subjunctive 3rd singular} ὅτι εἰς τὸ ἐμὸν ὄνομα ⸀ἐβαπτίσθητε·\FnParse{b}{aorist passive indicative 2nd plural} 
\MainTextVerseMark{1}{16}ἐβάπτισα\FnParse{a}{aorist active indicative 1st singular} δὲ καὶ τὸν Στεφανᾶ\FnParseFormGloss{b}{genitive masculine singular}{Στεφανᾶς}{Stephanas} οἶκον· λοιπὸν οὐκ οἶδα\FnParse{c}{perfect active indicative 1st singular} εἴ τινα ἄλλον ἐβάπτισα.\FnParse{d}{aorist active indicative 1st singular} 
\MainTextVerseMark{1}{17}οὐ γὰρ ἀπέστειλέν\FnParse{a}{aorist active indicative 3rd singular} με Χριστὸς βαπτίζειν\FnParse{b}{present active infinitive} ἀλλὰ εὐαγγελίζεσθαι,\FnParse{c}{present middle infinitive} οὐκ ἐν σοφίᾳ λόγου, ἵνα μὴ κενωθῇ\FnParseFormGloss{d}{aorist passive subjunctive 3rd singular}{κενόω}{to empty} ὁ σταυρὸς\FnParseFormGloss{e}{nominative masculine singular}{σταυρός}{cross} τοῦ Χριστοῦ. 
\MainTextVerseMark{1}{18}Ὁ λόγος γὰρ ὁ τοῦ σταυροῦ\FnParseFormGloss{a}{genitive masculine singular}{σταυρός}{cross} τοῖς μὲν ἀπολλυμένοις\FnParse{b}{present middle participle dative masculine plural} μωρία\FnParseFormGloss{c}{nominative feminine singular}{μωρία}{foolishness} ἐστίν,\FnParse{d}{present active indicative 3rd singular} τοῖς δὲ σῳζομένοις\FnParse{e}{present passive participle dative masculine plural} ἡμῖν δύναμις θεοῦ ἐστιν.\FnParse{f}{present active indicative 3rd singular} 
\MainTextVerseMark{1}{19}γέγραπται\FnParse{a}{perfect passive indicative 3rd singular} γάρ· Ἀπολῶ\FnParse{b}{future active indicative 1st singular} τὴν σοφίαν τῶν σοφῶν,\FnParseFormGloss{c}{genitive masculine plural}{σοφός}{wise} καὶ τὴν σύνεσιν\FnParseFormGloss{d}{accusative feminine singular}{σύνεσις}{understanding} τῶν συνετῶν\FnParseFormGloss{e}{genitive masculine plural}{συνετός}{intelligent} ἀθετήσω.\FnParseFormGloss{f}{future active indicative 1st singular}{ἀθετέω}{to reject} 
\MainTextVerseMark{1}{20}ποῦ σοφός;\FnParseFormGloss{a}{nominative masculine singular}{σοφός}{wise} ποῦ γραμματεύς; ποῦ συζητητὴς\FnParseFormGloss{b}{nominative masculine singular}{συζητητής}{philosopher} τοῦ αἰῶνος τούτου; οὐχὶ ἐμώρανεν\FnParseFormGloss{c}{aorist active indicative 3rd singular}{μωραίνω}{to make foolish} ὁ θεὸς τὴν σοφίαν τοῦ ⸀κόσμου; 
\MainTextVerseMark{1}{21}ἐπειδὴ\FnFormGloss{a}{ἐπειδή}{when} γὰρ ἐν τῇ σοφίᾳ τοῦ θεοῦ οὐκ ἔγνω\FnParse{b}{aorist active indicative 3rd singular} ὁ κόσμος διὰ τῆς σοφίας τὸν θεόν, εὐδόκησεν\FnParseFormGloss{c}{aorist active indicative 3rd singular}{εὐδοκέω}{to be well pleased} ὁ θεὸς διὰ τῆς μωρίας\FnParseFormGloss{d}{genitive feminine singular}{μωρία}{foolishness} τοῦ κηρύγματος\FnParseFormGloss{e}{genitive neuter singular}{κήρυγμα}{preaching} σῶσαι\FnParse{f}{aorist active infinitive} τοὺς πιστεύοντας.\FnParse{g}{present active participle accusative masculine plural} 
\MainTextVerseMark{1}{22}ἐπειδὴ\FnFormGloss{a}{ἐπειδή}{when} καὶ Ἰουδαῖοι ⸀σημεῖα αἰτοῦσιν\FnParse{b}{present active indicative 3rd plural} καὶ Ἕλληνες\FnParseFormGloss{c}{nominative masculine plural}{Ἕλλην}{Greek} σοφίαν ζητοῦσιν·\FnParse{d}{present active indicative 3rd plural} 
\MainTextVerseMark{1}{23}ἡμεῖς δὲ κηρύσσομεν\FnParse{a}{present active indicative 1st plural} Χριστὸν ἐσταυρωμένον,\FnParse{b}{perfect passive participle accusative masculine singular} Ἰουδαίοις μὲν σκάνδαλον\FnParseFormGloss{c}{accusative neuter singular}{σκάνδαλον}{stumbling block} ⸀ἔθνεσιν δὲ μωρίαν,\FnParseFormGloss{d}{accusative feminine singular}{μωρία}{foolishness} 
\MainTextVerseMark{1}{24}αὐτοῖς δὲ τοῖς κλητοῖς,\FnParseFormGloss{a}{dative masculine plural}{κλητός}{called} Ἰουδαίοις τε καὶ Ἕλλησιν,\FnParseFormGloss{b}{dative masculine plural}{Ἕλλην}{Greek} Χριστὸν θεοῦ δύναμιν καὶ θεοῦ σοφίαν. 
\MainTextVerseMark{1}{25}ὅτι τὸ μωρὸν\FnParseFormGloss{a}{nominative neuter singular}{μωρός}{foolish} τοῦ θεοῦ σοφώτερον\FnParseFormGloss{b}{nominative neuter singular}{σοφός}{wise} τῶν ἀνθρώπων ἐστίν,\FnParse{c}{present active indicative 3rd singular} καὶ τὸ ἀσθενὲς\FnParseFormGloss{d}{nominative neuter singular}{ἀσθενής}{weak} τοῦ θεοῦ ἰσχυρότερον\FnParseFormGloss{e}{nominative neuter singular}{ἰσχυρός}{powerful} τῶν ⸀ἀνθρώπων. 
\MainTextVerseMark{1}{26}Βλέπετε\FnParse{a}{present active indicative 2nd plural} γὰρ τὴν κλῆσιν\FnParseFormGloss{b}{accusative feminine singular}{κλῆσις}{call} ὑμῶν, ἀδελφοί, ὅτι οὐ πολλοὶ σοφοὶ\FnParseFormGloss{c}{nominative masculine plural}{σοφός}{wise} κατὰ σάρκα, οὐ πολλοὶ δυνατοί, οὐ πολλοὶ εὐγενεῖς·\FnParseFormGloss{d}{nominative masculine plural}{εὐγενής}{of noble birth} 
\MainTextVerseMark{1}{27}ἀλλὰ τὰ μωρὰ\FnParseFormGloss{a}{accusative neuter plural}{μωρός}{foolish} τοῦ κόσμου ἐξελέξατο\FnParseFormGloss{b}{aorist middle indicative 3rd singular}{ἐκλέγομαι}{to chose} ὁ θεός, ἵνα ⸂καταισχύνῃ\FnParseFormGloss{c}{present active subjunctive 3rd singular}{καταισχύνω}{to dishonor} τοὺς σοφούς⸃,\FnParseFormGloss{d}{accusative masculine plural}{σοφός}{wise} καὶ τὰ ἀσθενῆ\FnParseFormGloss{e}{accusative neuter plural}{ἀσθενής}{weak} τοῦ κόσμου ἐξελέξατο\FnParseFormGloss{f}{aorist middle indicative 3rd singular}{ἐκλέγομαι}{to chose} ὁ θεός, ἵνα καταισχύνῃ\FnParseFormGloss{g}{present active subjunctive 3rd singular}{καταισχύνω}{to dishonor} τὰ ἰσχυρά,\FnParseFormGloss{h}{accusative neuter plural}{ἰσχυρός}{powerful} 
\MainTextVerseMark{1}{28}καὶ τὰ ἀγενῆ\FnParseFormGloss{a}{accusative neuter plural}{ἀγενής}{lowly} τοῦ κόσμου καὶ τὰ ἐξουθενημένα\FnParseFormGloss{b}{perfect passive participle accusative neuter plural}{ἐξουθενέω}{to treat with contempt} ἐξελέξατο\FnParseFormGloss{c}{aorist middle indicative 3rd singular}{ἐκλέγομαι}{to chose} ὁ θεός, ⸀τὰ μὴ ὄντα,\FnParse{d}{present active participle accusative neuter plural} ἵνα τὰ ὄντα\FnParse{e}{present active participle accusative neuter plural} καταργήσῃ,\FnParseFormGloss{f}{aorist active subjunctive 3rd singular}{καταργέω}{to nullify} 
\MainTextVerseMark{1}{29}ὅπως μὴ καυχήσηται\FnParse{a}{aorist middle subjunctive 3rd singular} πᾶσα σὰρξ ἐνώπιον τοῦ θεοῦ. 
\MainTextVerseMark{1}{30}ἐξ αὐτοῦ δὲ ὑμεῖς ἐστε\FnParse{a}{present active indicative 2nd plural} ἐν Χριστῷ Ἰησοῦ, ὃς ἐγενήθη\FnParse{b}{aorist passive indicative 3rd singular} ⸂σοφία ἡμῖν⸃ ἀπὸ θεοῦ, δικαιοσύνη τε καὶ ἁγιασμὸς\FnParseFormGloss{c}{nominative masculine singular}{ἁγιασμός}{holiness} καὶ ἀπολύτρωσις,\FnParseFormGloss{d}{nominative feminine singular}{ἀπολύτρωσις}{redemption} 
\MainTextVerseMark{1}{31}ἵνα καθὼς γέγραπται·\FnParse{a}{perfect passive indicative 3rd singular} Ὁ καυχώμενος\FnParse{b}{present middle participle nominative masculine singular} ἐν κυρίῳ καυχάσθω.\FnParse{c}{present middle imperative 3rd singular} 
\ChapterHeader{2}{Chapter 2}

\MainTextVerseMark{2}{1}Κἀγὼ ἐλθὼν\FnParse{a}{aorist active participle nominative masculine singular} πρὸς ὑμᾶς, ἀδελφοί, ἦλθον\FnParse{b}{aorist active indicative 1st singular} οὐ καθ’ ὑπεροχὴν\FnParseFormGloss{c}{accusative feminine singular}{ὑπεροχή}{authority} λόγου ἢ σοφίας καταγγέλλων\FnParseFormGloss{d}{present active participle nominative masculine singular}{καταγγέλλω}{to preach} ὑμῖν τὸ ⸀μαρτύριον\FnParseFormGloss{e}{accusative neuter singular}{μαρτύριον}{testimony} τοῦ θεοῦ. 
\MainTextVerseMark{2}{2}οὐ γὰρ ἔκρινά\FnParse{a}{aorist active indicative 1st singular} ⸂τι εἰδέναι⸃\FnParse{b}{perfect active infinitive} ἐν ὑμῖν εἰ μὴ Ἰησοῦν Χριστὸν καὶ τοῦτον ἐσταυρωμένον·\FnParse{c}{perfect passive participle accusative masculine singular} 
\MainTextVerseMark{2}{3}κἀγὼ ἐν ἀσθενείᾳ\FnParseFormGloss{a}{dative feminine singular}{ἀσθένεια}{weakness} καὶ ἐν φόβῳ καὶ ἐν τρόμῳ\FnParseFormGloss{b}{dative masculine singular}{τρόμος}{trembling} πολλῷ ἐγενόμην\FnParse{c}{aorist middle indicative 1st singular} πρὸς ὑμᾶς, 
\MainTextVerseMark{2}{4}καὶ ὁ λόγος μου καὶ τὸ κήρυγμά\FnParseFormGloss{a}{nominative neuter singular}{κήρυγμα}{preaching} μου οὐκ ἐν ⸂πειθοῖ\FnParseFormGloss{b}{dative masculine plural}{πειθός}{persuasive} σοφίας⸃ ἀλλ’ ἐν ἀποδείξει\FnParseFormGloss{c}{dative feminine singular}{ἀπόδειξις}{demonstration} πνεύματος καὶ δυνάμεως, 
\MainTextVerseMark{2}{5}ἵνα ἡ πίστις ὑμῶν μὴ ᾖ\FnParse{a}{present active subjunctive 3rd singular} ἐν σοφίᾳ ἀνθρώπων ἀλλ’ ἐν δυνάμει θεοῦ. 
\MainTextVerseMark{2}{6}Σοφίαν δὲ λαλοῦμεν\FnParse{a}{present active indicative 1st plural} ἐν τοῖς τελείοις,\FnParseFormGloss{b}{dative masculine plural}{τέλειος}{perfect} σοφίαν δὲ οὐ τοῦ αἰῶνος τούτου οὐδὲ τῶν ἀρχόντων τοῦ αἰῶνος τούτου τῶν καταργουμένων·\FnParseFormGloss{c}{present passive participle genitive masculine plural}{καταργέω}{to nullify} 
\MainTextVerseMark{2}{7}ἀλλὰ λαλοῦμεν\FnParse{a}{present active indicative 1st plural} ⸂θεοῦ σοφίαν⸃ ἐν μυστηρίῳ,\FnParseFormGloss{b}{dative neuter singular}{μυστήριον}{mystery} τὴν ἀποκεκρυμμένην,\FnParseFormGloss{c}{perfect passive participle accusative feminine singular}{ἀποκρύπτω}{to hide} ἣν προώρισεν\FnParseFormGloss{d}{aorist active indicative 3rd singular}{προορίζω}{to predestine} ὁ θεὸς πρὸ τῶν αἰώνων εἰς δόξαν ἡμῶν· 
\MainTextVerseMark{2}{8}ἣν οὐδεὶς τῶν ἀρχόντων τοῦ αἰῶνος τούτου ἔγνωκεν,\FnParse{a}{perfect active indicative 3rd singular} εἰ γὰρ ἔγνωσαν,\FnParse{b}{aorist active indicative 3rd plural} οὐκ ἂν τὸν κύριον τῆς δόξης ἐσταύρωσαν·\FnParse{c}{aorist active indicative 3rd plural} 
\MainTextVerseMark{2}{9}ἀλλὰ καθὼς γέγραπται·\FnParse{a}{perfect passive indicative 3rd singular} Ἃ ὀφθαλμὸς οὐκ εἶδεν\FnParse{b}{aorist active indicative 3rd singular} καὶ οὖς οὐκ ἤκουσεν\FnParse{c}{aorist active indicative 3rd singular} καὶ ἐπὶ καρδίαν ἀνθρώπου οὐκ ἀνέβη,\FnParse{d}{aorist active indicative 3rd singular} ⸀ὅσα ἡτοίμασεν\FnParse{e}{aorist active indicative 3rd singular} ὁ θεὸς τοῖς ἀγαπῶσιν\FnParse{f}{present active participle dative masculine plural} αὐτόν. 
\MainTextVerseMark{2}{10}ἡμῖν ⸀γὰρ ⸂ἀπεκάλυψεν\FnParseFormGloss{a}{aorist active indicative 3rd singular}{ἀποκαλύπτω}{to reveal} ὁ θεὸς⸃ διὰ τοῦ ⸀πνεύματος, τὸ γὰρ πνεῦμα πάντα ἐραυνᾷ,\FnParseFormGloss{b}{present active indicative 3rd singular}{ἐραυνάω}{to search} καὶ τὰ βάθη\FnParseFormGloss{c}{accusative neuter plural}{βάθος}{depth} τοῦ θεοῦ. 
\MainTextVerseMark{2}{11}τίς γὰρ οἶδεν\FnParse{a}{perfect active indicative 3rd singular} ἀνθρώπων τὰ τοῦ ἀνθρώπου εἰ μὴ τὸ πνεῦμα τοῦ ἀνθρώπου τὸ ἐν αὐτῷ; οὕτως καὶ τὰ τοῦ θεοῦ οὐδεὶς ⸀ἔγνωκεν\FnParse{b}{perfect active indicative 3rd singular} εἰ μὴ τὸ πνεῦμα τοῦ θεοῦ. 
\MainTextVerseMark{2}{12}ἡμεῖς δὲ οὐ τὸ πνεῦμα τοῦ κόσμου ἐλάβομεν\FnParse{a}{aorist active indicative 1st plural} ἀλλὰ τὸ πνεῦμα τὸ ἐκ τοῦ θεοῦ, ἵνα εἰδῶμεν\FnParse{b}{perfect active subjunctive 1st plural} τὰ ὑπὸ τοῦ θεοῦ χαρισθέντα\FnParseFormGloss{c}{aorist passive participle accusative neuter plural}{χαρίζομαι}{to give grace} ἡμῖν· 
\MainTextVerseMark{2}{13}ἃ καὶ λαλοῦμεν\FnParse{a}{present active indicative 1st plural} οὐκ ἐν διδακτοῖς\FnParseFormGloss{b}{dative masculine plural}{διδακτός}{taught} ἀνθρωπίνης\FnParseFormGloss{c}{genitive feminine singular}{ἀνθρώπινος}{human} σοφίας λόγοις, ἀλλ’ ἐν διδακτοῖς\FnParseFormGloss{d}{dative masculine plural}{διδακτός}{taught} ⸀πνεύματος, πνευματικοῖς\FnParseFormGloss{e}{dative neuter plural}{πνευματικός}{spiritual} πνευματικὰ\FnParseFormGloss{f}{accusative neuter plural}{πνευματικός}{spiritual} συγκρίνοντες.\FnParseFormGloss{g}{present active participle nominative masculine plural}{συγκρίνω}{to express} 
\MainTextVerseMark{2}{14}Ψυχικὸς\FnParseFormGloss{a}{nominative masculine singular}{ψυχικός}{physical} δὲ ἄνθρωπος οὐ δέχεται\FnParse{b}{present middle indicative 3rd singular} τὰ τοῦ πνεύματος τοῦ θεοῦ, μωρία\FnParseFormGloss{c}{nominative feminine singular}{μωρία}{foolishness} γὰρ αὐτῷ ἐστίν,\FnParse{d}{present active indicative 3rd singular} καὶ οὐ δύναται\FnParse{e}{present middle indicative 3rd singular} γνῶναι,\FnParse{f}{aorist active infinitive} ὅτι πνευματικῶς\FnFormGloss{g}{πνευματικῶς}{spiritually} ἀνακρίνεται·\FnParseFormGloss{h}{present passive indicative 3rd singular}{ἀνακρίνω}{to examine} 
\MainTextVerseMark{2}{15}ὁ δὲ πνευματικὸς\FnParseFormGloss{a}{nominative masculine singular}{πνευματικός}{spiritual} ἀνακρίνει\FnParseFormGloss{b}{present active indicative 3rd singular}{ἀνακρίνω}{to examine} ⸀τὰ πάντα, αὐτὸς δὲ ὑπ’ οὐδενὸς ἀνακρίνεται.\FnParseFormGloss{c}{present passive indicative 3rd singular}{ἀνακρίνω}{to examine} 
\MainTextVerseMark{2}{16}τίς γὰρ ἔγνω\FnParse{a}{aorist active indicative 3rd singular} νοῦν\FnParseFormGloss{b}{accusative masculine singular}{νοῦς}{mind} κυρίου, ὃς συμβιβάσει\FnParseFormGloss{c}{future active indicative 3rd singular}{συμβιβάζω}{to be held together} αὐτόν; ἡμεῖς δὲ νοῦν\FnParseFormGloss{d}{accusative masculine singular}{νοῦς}{mind} Χριστοῦ ἔχομεν.\FnParse{e}{present active indicative 1st plural} 
\ChapterHeader{3}{Chapter 3}

\MainTextVerseMark{3}{1}Κἀγώ, ἀδελφοί, οὐκ ἠδυνήθην\FnParse{a}{aorist passive indicative 1st singular} ⸂λαλῆσαι\FnParse{b}{aorist active infinitive} ὑμῖν⸃ ὡς πνευματικοῖς\FnParseFormGloss{c}{dative masculine plural}{πνευματικός}{spiritual} ἀλλ’ ὡς ⸀σαρκίνοις,\FnParseFormGloss{d}{dative masculine plural}{σάρκινος}{fleshly} ὡς νηπίοις\FnParseFormGloss{e}{dative masculine plural}{νήπιος}{child} ἐν Χριστῷ. 
\MainTextVerseMark{3}{2}γάλα\FnParseFormGloss{a}{accusative neuter singular}{γάλα}{milk} ὑμᾶς ἐπότισα,\FnParseFormGloss{b}{aorist active indicative 1st singular}{ποτίζω}{to give a drink} ⸀οὐ βρῶμα,\FnParseFormGloss{c}{accusative neuter singular}{βρῶμα}{food} οὔπω\FnFormGloss{d}{οὔπω}{not yet} γὰρ ἐδύνασθε.\FnParse{e}{imperfect middle indicative 2nd plural} ἀλλ’ ⸀οὐδὲ ἔτι νῦν δύνασθε,\FnParse{f}{present middle indicative 2nd plural} 
\MainTextVerseMark{3}{3}ἔτι γὰρ σαρκικοί\FnParseFormGloss{a}{nominative masculine plural}{σαρκικός}{material} ἐστε.\FnParse{b}{present active indicative 2nd plural} ὅπου γὰρ ἐν ὑμῖν ζῆλος\FnParseFormGloss{c}{nominative masculine singular}{ζῆλος}{zeal} καὶ ⸀ἔρις,\FnParseFormGloss{d}{nominative feminine singular}{ἔρις}{quarrel} οὐχὶ σαρκικοί\FnParseFormGloss{e}{nominative masculine plural}{σαρκικός}{material} ἐστε\FnParse{f}{present active indicative 2nd plural} καὶ κατὰ ἄνθρωπον περιπατεῖτε;\FnParse{g}{present active indicative 2nd plural} 
\MainTextVerseMark{3}{4}ὅταν γὰρ λέγῃ\FnParse{a}{present active subjunctive 3rd singular} τις· Ἐγὼ μέν εἰμι\FnParse{b}{present active indicative 1st singular} Παύλου, ἕτερος δέ· Ἐγὼ Ἀπολλῶ,\FnParseFormGloss{c}{genitive masculine singular}{Ἀπολλῶς}{Apollos} ⸂οὐκ ἄνθρωποί⸃ ἐστε;\FnParse{d}{present active indicative 2nd plural} 
\MainTextVerseMark{3}{5}⸀Τί οὖν ἐστιν\FnParse{a}{present active indicative 3rd singular} ⸂Ἀπολλῶς;\FnParseFormGloss{b}{nominative masculine singular}{Ἀπολλῶς}{Apollos} τί δέ ἐστιν\FnParse{c}{present active indicative 3rd singular} Παῦλος⸃; ⸀διάκονοι\FnParseFormGloss{d}{nominative masculine plural}{διάκονος}{servant} δι’ ὧν ἐπιστεύσατε,\FnParse{e}{aorist active indicative 2nd plural} καὶ ἑκάστῳ ὡς ὁ κύριος ἔδωκεν.\FnParse{f}{aorist active indicative 3rd singular} 
\MainTextVerseMark{3}{6}ἐγὼ ἐφύτευσα,\FnParseFormGloss{a}{aorist active indicative 1st singular}{φυτεύω}{to plant} Ἀπολλῶς\FnParseFormGloss{b}{nominative masculine singular}{Ἀπολλῶς}{Apollos} ἐπότισεν,\FnParseFormGloss{c}{aorist active indicative 3rd singular}{ποτίζω}{to give a drink} ἀλλὰ ὁ θεὸς ηὔξανεν·\FnParseFormGloss{d}{imperfect active indicative 3rd singular}{αὐξάνω}{to cause to grow} 
\MainTextVerseMark{3}{7}ὥστε οὔτε ὁ φυτεύων\FnParseFormGloss{a}{present active participle nominative masculine singular}{φυτεύω}{to plant} ἐστίν\FnParse{b}{present active indicative 3rd singular} τι οὔτε ὁ ποτίζων,\FnParseFormGloss{c}{present active participle nominative masculine singular}{ποτίζω}{to give a drink} ἀλλ’ ὁ αὐξάνων\FnParseFormGloss{d}{present active participle nominative masculine singular}{αὐξάνω}{to cause to grow} θεός. 
\MainTextVerseMark{3}{8}ὁ φυτεύων\FnParseFormGloss{a}{present active participle nominative masculine singular}{φυτεύω}{to plant} δὲ καὶ ὁ ποτίζων\FnParseFormGloss{b}{present active participle nominative masculine singular}{ποτίζω}{to give a drink} ἕν εἰσιν,\FnParse{c}{present active indicative 3rd plural} ἕκαστος δὲ τὸν ἴδιον μισθὸν\FnParseFormGloss{d}{accusative masculine singular}{μισθός}{wage} λήμψεται\FnParse{e}{future middle indicative 3rd singular} κατὰ τὸν ἴδιον κόπον,\FnParseFormGloss{f}{accusative masculine singular}{κόπος}{labor} 
\MainTextVerseMark{3}{9}θεοῦ γάρ ἐσμεν\FnParse{a}{present active indicative 1st plural} συνεργοί·\FnParseFormGloss{b}{nominative masculine plural}{συνεργός}{fellow worker} θεοῦ γεώργιον,\FnParseFormGloss{c}{nominative neuter singular}{γεώργιον}{field} θεοῦ οἰκοδομή\FnParseFormGloss{d}{nominative feminine singular}{οἰκοδομή}{building} ἐστε.\FnParse{e}{present active indicative 2nd plural} 
\MainTextVerseMark{3}{10}Κατὰ τὴν χάριν τοῦ θεοῦ τὴν δοθεῖσάν\FnParse{a}{aorist passive participle accusative feminine singular} μοι ὡς σοφὸς\FnParseFormGloss{b}{nominative masculine singular}{σοφός}{wise} ἀρχιτέκτων\FnParseFormGloss{c}{nominative masculine singular}{ἀρχιτέκτων}{expert builder} θεμέλιον\FnParseFormGloss{d}{accusative masculine singular}{θεμέλιος}{foundation} ⸀ἔθηκα,\FnParse{e}{aorist active indicative 1st singular} ἄλλος δὲ ἐποικοδομεῖ.\FnParseFormGloss{f}{present active indicative 3rd singular}{ἐποικοδομέω}{to build up} ἕκαστος δὲ βλεπέτω\FnParse{g}{present active imperative 3rd singular} πῶς ἐποικοδομεῖ·\FnParseFormGloss{h}{present active indicative 3rd singular}{ἐποικοδομέω}{to build up} 
\MainTextVerseMark{3}{11}θεμέλιον\FnParseFormGloss{a}{accusative masculine singular}{θεμέλιος}{foundation} γὰρ ἄλλον οὐδεὶς δύναται\FnParse{b}{present middle indicative 3rd singular} θεῖναι\FnParse{c}{aorist active infinitive} παρὰ τὸν κείμενον,\FnParseFormGloss{d}{present middle participle accusative masculine singular}{κεῖμαι}{to lay} ὅς ἐστιν\FnParse{e}{present active indicative 3rd singular} Ἰησοῦς Χριστός· 
\MainTextVerseMark{3}{12}εἰ δέ τις ἐποικοδομεῖ\FnParseFormGloss{a}{present active indicative 3rd singular}{ἐποικοδομέω}{to build up} ἐπὶ τὸν ⸀θεμέλιον\FnParseFormGloss{b}{accusative masculine singular}{θεμέλιος}{foundation} ⸂χρυσόν,\FnParseFormGloss{c}{accusative masculine singular}{χρυσός}{gold} ἄργυρον⸃,\FnParseFormGloss{d}{accusative masculine singular}{ἄργυρος}{silver} λίθους τιμίους,\FnParseFormGloss{e}{accusative masculine plural}{τίμιος}{precious} ξύλα,\FnParseFormGloss{f}{accusative neuter plural}{ξύλον}{wood} χόρτον,\FnParseFormGloss{g}{accusative masculine singular}{χόρτος}{grass} καλάμην,\FnParseFormGloss{h}{accusative feminine singular}{καλάμη}{straw} 
\MainTextVerseMark{3}{13}ἑκάστου τὸ ἔργον φανερὸν\FnParseFormGloss{a}{nominative neuter singular}{φανερός}{visible} γενήσεται,\FnParse{b}{future middle indicative 3rd singular} ἡ γὰρ ἡμέρα δηλώσει·\FnParseFormGloss{c}{future active indicative 3rd singular}{δηλόω}{to make clear} ὅτι ἐν πυρὶ ἀποκαλύπτεται,\FnParseFormGloss{d}{present passive indicative 3rd singular}{ἀποκαλύπτω}{to reveal} καὶ ἑκάστου τὸ ἔργον ὁποῖόν\FnParseFormGloss{e}{nominative neuter singular}{ὁποῖος}{what kind of} ἐστιν\FnParse{f}{present active indicative 3rd singular} τὸ πῦρ ⸀αὐτὸ δοκιμάσει.\FnParseFormGloss{g}{future active indicative 3rd singular}{δοκιμάζω}{to test} 
\MainTextVerseMark{3}{14}εἴ τινος τὸ ἔργον μενεῖ\FnParse{a}{future active indicative 3rd singular} ὃ ἐποικοδόμησεν,\FnParseFormGloss{b}{aorist active indicative 3rd singular}{ἐποικοδομέω}{to build up} μισθὸν\FnParseFormGloss{c}{accusative masculine singular}{μισθός}{wage} λήμψεται·\FnParse{d}{future middle indicative 3rd singular} 
\MainTextVerseMark{3}{15}εἴ τινος τὸ ἔργον κατακαήσεται,\FnParseFormGloss{a}{future passive indicative 3rd singular}{κατακαίω}{to burn up} ζημιωθήσεται,\FnParseFormGloss{b}{future passive indicative 3rd singular}{ζημιόω}{to forfeit} αὐτὸς δὲ σωθήσεται,\FnParse{c}{future passive indicative 3rd singular} οὕτως δὲ ὡς διὰ πυρός. 
\MainTextVerseMark{3}{16}Οὐκ οἴδατε\FnParse{a}{perfect active indicative 2nd plural} ὅτι ναὸς θεοῦ ἐστε\FnParse{b}{present active indicative 2nd plural} καὶ τὸ πνεῦμα τοῦ θεοῦ ⸂οἰκεῖ\FnParseFormGloss{c}{present active indicative 3rd singular}{οἰκέω}{to live} ἐν ὑμῖν⸃; 
\MainTextVerseMark{3}{17}εἴ τις τὸν ναὸν τοῦ θεοῦ φθείρει,\FnParseFormGloss{a}{present active indicative 3rd singular}{φθείρω}{to destroy} φθερεῖ\FnParseFormGloss{b}{future active indicative 3rd singular}{φθείρω}{to destroy} τοῦτον ὁ θεός· ὁ γὰρ ναὸς τοῦ θεοῦ ἅγιός ἐστιν,\FnParse{c}{present active indicative 3rd singular} οἵτινές ἐστε\FnParse{d}{present active indicative 2nd plural} ὑμεῖς. 
\MainTextVerseMark{3}{18}Μηδεὶς ἑαυτὸν ἐξαπατάτω·\FnParseFormGloss{a}{present active imperative 3rd singular}{ἐξαπατάω}{to deceive} εἴ τις δοκεῖ\FnParse{b}{present active indicative 3rd singular} σοφὸς\FnParseFormGloss{c}{nominative masculine singular}{σοφός}{wise} εἶναι\FnParse{d}{present active infinitive} ἐν ὑμῖν ἐν τῷ αἰῶνι τούτῳ, μωρὸς\FnParseFormGloss{e}{nominative masculine singular}{μωρός}{foolish} γενέσθω,\FnParse{f}{aorist middle imperative 3rd singular} ἵνα γένηται\FnParse{g}{aorist middle subjunctive 3rd singular} σοφός,\FnParseFormGloss{h}{nominative masculine singular}{σοφός}{wise} 
\MainTextVerseMark{3}{19}ἡ γὰρ σοφία τοῦ κόσμου τούτου μωρία\FnParseFormGloss{a}{nominative feminine singular}{μωρία}{foolishness} παρὰ τῷ θεῷ ἐστιν·\FnParse{b}{present active indicative 3rd singular} γέγραπται\FnParse{c}{perfect passive indicative 3rd singular} γάρ· Ὁ δρασσόμενος\FnParseFormGloss{d}{present middle participle nominative masculine singular}{δράσσομαι}{to catch} τοὺς σοφοὺς\FnParseFormGloss{e}{accusative masculine plural}{σοφός}{wise} ἐν τῇ πανουργίᾳ\FnParseFormGloss{f}{dative feminine singular}{πανουργία}{cunning} αὐτῶν· 
\MainTextVerseMark{3}{20}καὶ πάλιν· Κύριος γινώσκει\FnParse{a}{present active indicative 3rd singular} τοὺς διαλογισμοὺς\FnParseFormGloss{b}{accusative masculine plural}{διαλογισμός}{thought} τῶν σοφῶν\FnParseFormGloss{c}{genitive masculine plural}{σοφός}{wise} ὅτι εἰσὶν\FnParse{d}{present active indicative 3rd plural} μάταιοι.\FnParseFormGloss{e}{nominative masculine plural}{μάταιος}{worthless} 
\MainTextVerseMark{3}{21}ὥστε μηδεὶς καυχάσθω\FnParse{a}{present middle imperative 3rd singular} ἐν ἀνθρώποις· πάντα γὰρ ὑμῶν ἐστιν,\FnParse{b}{present active indicative 3rd singular} 
\MainTextVerseMark{3}{22}εἴτε Παῦλος εἴτε Ἀπολλῶς\FnParseFormGloss{a}{nominative masculine singular}{Ἀπολλῶς}{Apollos} εἴτε Κηφᾶς\FnParseFormGloss{b}{nominative masculine singular}{Κηφᾶς}{Cephas} εἴτε κόσμος εἴτε ζωὴ εἴτε θάνατος εἴτε ἐνεστῶτα\FnParseFormGloss{c}{perfect active participle nominative neuter plural}{ἐνίστημι}{to be present} εἴτε μέλλοντα,\FnParse{d}{present active participle nominative neuter plural} πάντα ⸀ὑμῶν, 
\MainTextVerseMark{3}{23}ὑμεῖς δὲ Χριστοῦ, Χριστὸς δὲ θεοῦ. 
\ChapterHeader{4}{Chapter 4}

\MainTextVerseMark{4}{1}Οὕτως ἡμᾶς λογιζέσθω\FnParse{a}{present middle imperative 3rd singular} ἄνθρωπος ὡς ὑπηρέτας\FnParseFormGloss{b}{accusative masculine plural}{ὑπηρέτης}{servant} Χριστοῦ καὶ οἰκονόμους\FnParseFormGloss{c}{accusative masculine plural}{οἰκονόμος}{manager} μυστηρίων\FnParseFormGloss{d}{genitive neuter plural}{μυστήριον}{mystery} θεοῦ. 
\MainTextVerseMark{4}{2}⸀ὧδε λοιπὸν ζητεῖται\FnParse{a}{present passive indicative 3rd singular} ἐν τοῖς οἰκονόμοις\FnParseFormGloss{b}{dative masculine plural}{οἰκονόμος}{manager} ἵνα πιστός τις εὑρεθῇ.\FnParse{c}{aorist passive subjunctive 3rd singular} 
\MainTextVerseMark{4}{3}ἐμοὶ δὲ εἰς ἐλάχιστόν\FnParseFormGloss{a}{accusative neuter singular}{ἐλάχιστος}{least} ἐστιν,\FnParse{b}{present active indicative 3rd singular} ἵνα ὑφ’ ὑμῶν ἀνακριθῶ\FnParseFormGloss{c}{aorist passive subjunctive 1st singular}{ἀνακρίνω}{to examine} ἢ ὑπὸ ἀνθρωπίνης\FnParseFormGloss{d}{genitive feminine singular}{ἀνθρώπινος}{human} ἡμέρας· ἀλλ’ οὐδὲ ἐμαυτὸν ἀνακρίνω·\FnParseFormGloss{e}{present active indicative 1st singular}{ἀνακρίνω}{to examine} 
\MainTextVerseMark{4}{4}οὐδὲν γὰρ ἐμαυτῷ σύνοιδα,\FnParseFormGloss{a}{perfect active indicative 1st singular}{σύνοιδα}{to share knowledge with} ἀλλ’ οὐκ ἐν τούτῳ δεδικαίωμαι,\FnParse{b}{perfect passive indicative 1st singular} ὁ δὲ ἀνακρίνων\FnParseFormGloss{c}{present active participle nominative masculine singular}{ἀνακρίνω}{to examine} με κύριός ἐστιν.\FnParse{d}{present active indicative 3rd singular} 
\MainTextVerseMark{4}{5}ὥστε μὴ πρὸ καιροῦ τι κρίνετε,\FnParse{a}{present active imperative 2nd plural} ἕως ἂν ἔλθῃ\FnParse{b}{aorist active subjunctive 3rd singular} ὁ κύριος, ὃς καὶ φωτίσει\FnParseFormGloss{c}{future active indicative 3rd singular}{φωτίζω}{to give light} τὰ κρυπτὰ\FnParseFormGloss{d}{accusative neuter plural}{κρυπτός}{hidden} τοῦ σκότους καὶ φανερώσει\FnParse{e}{future active indicative 3rd singular} τὰς βουλὰς\FnParseFormGloss{f}{accusative feminine plural}{βουλή}{plan} τῶν καρδιῶν, καὶ τότε ὁ ἔπαινος\FnParseFormGloss{g}{nominative masculine singular}{ἔπαινος}{praise} γενήσεται\FnParse{h}{future middle indicative 3rd singular} ἑκάστῳ ἀπὸ τοῦ θεοῦ. 
\MainTextVerseMark{4}{6}Ταῦτα δέ, ἀδελφοί, μετεσχημάτισα\FnParseFormGloss{a}{aorist active indicative 1st singular}{μετασχηματίζω}{to transform} εἰς ἐμαυτὸν καὶ ⸀Ἀπολλῶν\FnParseFormGloss{b}{accusative masculine singular}{Ἀπολλῶς}{Apollos} δι’ ὑμᾶς, ἵνα ἐν ἡμῖν μάθητε\FnParseFormGloss{c}{aorist active subjunctive 2nd plural}{μανθάνω}{to learn} τό· Μὴ ὑπὲρ ⸀ἃ ⸀γέγραπται,\FnParse{d}{perfect passive indicative 3rd singular} ἵνα μὴ εἷς ὑπὲρ τοῦ ἑνὸς φυσιοῦσθε\FnParseFormGloss{e}{present passive indicative 2nd plural}{φυσιόω}{to puff up} κατὰ τοῦ ἑτέρου. 
\MainTextVerseMark{4}{7}τίς γάρ σε διακρίνει;\FnParseFormGloss{a}{present active indicative 3rd singular}{διακρίνω}{to make a distinction} τί δὲ ἔχεις\FnParse{b}{present active indicative 2nd singular} ὃ οὐκ ἔλαβες;\FnParse{c}{aorist active indicative 2nd singular} εἰ δὲ καὶ ἔλαβες,\FnParse{d}{aorist active indicative 2nd singular} τί καυχᾶσαι\FnParse{e}{present middle indicative 2nd singular} ὡς μὴ λαβών;\FnParse{f}{aorist active participle nominative masculine singular} 
\MainTextVerseMark{4}{8}Ἤδη κεκορεσμένοι\FnParseFormGloss{a}{perfect passive participle nominative masculine plural}{κορέννυμι}{to be filled to the full} ἐστέ,\FnParse{b}{present active indicative 2nd plural} ἤδη ἐπλουτήσατε,\FnParseFormGloss{c}{aorist active indicative 2nd plural}{πλουτέω}{to be rich} χωρὶς ἡμῶν ἐβασιλεύσατε·\FnParseFormGloss{d}{aorist active indicative 2nd plural}{βασιλεύω}{to reign as a king} καὶ ὄφελόν\FnFormGloss{e}{ὄφελον}{How I wish! How I hope!} γε\FnFormGloss{f}{γέ}{indeed} ἐβασιλεύσατε,\FnParseFormGloss{g}{aorist active indicative 2nd plural}{βασιλεύω}{to reign as a king} ἵνα καὶ ἡμεῖς ὑμῖν συμβασιλεύσωμεν.\FnParseFormGloss{h}{aorist active subjunctive 1st plural}{συμβασιλεύω}{to reign with} 
\MainTextVerseMark{4}{9}δοκῶ\FnParse{a}{present active indicative 1st singular} ⸀γάρ, ὁ θεὸς ἡμᾶς τοὺς ἀποστόλους ἐσχάτους ἀπέδειξεν\FnParseFormGloss{b}{aorist active indicative 3rd singular}{ἀποδείκνυμι}{to display} ὡς ἐπιθανατίους,\FnParseFormGloss{c}{accusative masculine plural}{ἐπιθανάτιος}{condemned to die} ὅτι θέατρον\FnParseFormGloss{d}{nominative neuter singular}{θέατρον}{theatre} ἐγενήθημεν\FnParse{e}{aorist passive indicative 1st plural} τῷ κόσμῳ καὶ ἀγγέλοις καὶ ἀνθρώποις. 
\MainTextVerseMark{4}{10}ἡμεῖς μωροὶ\FnParseFormGloss{a}{nominative masculine plural}{μωρός}{foolish} διὰ Χριστόν, ὑμεῖς δὲ φρόνιμοι\FnParseFormGloss{b}{nominative masculine plural}{φρόνιμος}{wise} ἐν Χριστῷ· ἡμεῖς ἀσθενεῖς,\FnParseFormGloss{c}{nominative masculine plural}{ἀσθενής}{weak} ὑμεῖς δὲ ἰσχυροί·\FnParseFormGloss{d}{nominative masculine plural}{ἰσχυρός}{powerful} ὑμεῖς ἔνδοξοι,\FnParseFormGloss{e}{nominative masculine plural}{ἔνδοξος}{honored} ἡμεῖς δὲ ἄτιμοι.\FnParseFormGloss{f}{nominative masculine plural}{ἄτιμος}{without honor} 
\MainTextVerseMark{4}{11}ἄχρι τῆς ἄρτι ὥρας καὶ πεινῶμεν\FnParseFormGloss{a}{present active indicative 1st plural}{πεινάω}{to be hungry} καὶ διψῶμεν\FnParseFormGloss{b}{present active indicative 1st plural}{διψάω}{to be thirsty} καὶ γυμνιτεύομεν\FnParseFormGloss{c}{present active indicative 1st plural}{γυμνιτεύω}{to be in ragged clothing} καὶ κολαφιζόμεθα\FnParseFormGloss{d}{present passive indicative 1st plural}{κολαφίζω}{to strike with the fists} καὶ ἀστατοῦμεν\FnParseFormGloss{e}{present active indicative 1st plural}{ἀστατέω}{to be homeless} 
\MainTextVerseMark{4}{12}καὶ κοπιῶμεν\FnParseFormGloss{a}{present active indicative 1st plural}{κοπιάω}{to work} ἐργαζόμενοι\FnParse{b}{present middle participle nominative masculine plural} ταῖς ἰδίαις χερσίν· λοιδορούμενοι\FnParseFormGloss{c}{present passive participle nominative masculine plural}{λοιδορέω}{to insult} εὐλογοῦμεν,\FnParse{d}{present active indicative 1st plural} διωκόμενοι\FnParse{e}{present passive participle nominative masculine plural} ἀνεχόμεθα,\FnParseFormGloss{f}{present middle indicative 1st plural}{ἀνέχομαι}{to put up with} 
\MainTextVerseMark{4}{13}⸀δυσφημούμενοι\FnParseFormGloss{a}{present passive participle nominative masculine plural}{δυσφημέω}{to be slandered} παρακαλοῦμεν·\FnParse{b}{present active indicative 1st plural} ὡς περικαθάρματα\FnParseFormGloss{c}{nominative neuter plural}{περικάθαρμα}{scum} τοῦ κόσμου ἐγενήθημεν,\FnParse{d}{aorist passive indicative 1st plural} πάντων περίψημα\FnParseFormGloss{e}{nominative neuter singular}{περίψημα}{refuse} ἕως ἄρτι. 
\MainTextVerseMark{4}{14}Οὐκ ἐντρέπων\FnParseFormGloss{a}{present active participle nominative masculine singular}{ἐντρέπω}{to cause shame} ὑμᾶς γράφω\FnParse{b}{present active indicative 1st singular} ταῦτα, ἀλλ’ ὡς τέκνα μου ἀγαπητὰ ⸀νουθετῶν·\FnParseFormGloss{c}{present active participle nominative masculine singular}{νουθετέω}{to warn} 
\MainTextVerseMark{4}{15}ἐὰν γὰρ μυρίους\FnParseFormGloss{a}{accusative masculine plural}{μύριοι}{ten thousand} παιδαγωγοὺς\FnParseFormGloss{b}{accusative masculine plural}{παιδαγωγός}{guardian} ἔχητε\FnParse{c}{present active subjunctive 2nd plural} ἐν Χριστῷ, ἀλλ’ οὐ πολλοὺς πατέρας, ἐν γὰρ Χριστῷ Ἰησοῦ διὰ τοῦ εὐαγγελίου ἐγὼ ὑμᾶς ἐγέννησα.\FnParse{d}{aorist active indicative 1st singular} 
\MainTextVerseMark{4}{16}παρακαλῶ\FnParse{a}{present active indicative 1st singular} οὖν ὑμᾶς, μιμηταί\FnParseFormGloss{b}{nominative masculine plural}{μιμητής}{imitator} μου γίνεσθε.\FnParse{c}{present middle imperative 2nd plural} 
\MainTextVerseMark{4}{17}διὰ τοῦτο ἔπεμψα\FnParse{a}{aorist active indicative 1st singular} ὑμῖν Τιμόθεον,\FnParseFormGloss{b}{accusative masculine singular}{Τιμόθεος}{Timothy} ὅς ἐστίν\FnParse{c}{present active indicative 3rd singular} ⸂μου τέκνον⸃ ἀγαπητὸν καὶ πιστὸν ἐν κυρίῳ, ὃς ὑμᾶς ἀναμνήσει\FnParseFormGloss{d}{future active indicative 3rd singular}{ἀναμιμνῄσκω}{to remember} τὰς ὁδούς μου τὰς ἐν Χριστῷ ⸀Ἰησοῦ, καθὼς πανταχοῦ\FnFormGloss{e}{πανταχοῦ}{everywhere} ἐν πάσῃ ἐκκλησίᾳ διδάσκω.\FnParse{f}{present active indicative 1st singular} 
\MainTextVerseMark{4}{18}ὡς μὴ ἐρχομένου\FnParse{a}{present middle participle genitive masculine singular} δέ μου πρὸς ὑμᾶς ἐφυσιώθησάν\FnParseFormGloss{b}{aorist passive indicative 3rd plural}{φυσιόω}{to puff up} τινες· 
\MainTextVerseMark{4}{19}ἐλεύσομαι\FnParse{a}{future middle indicative 1st singular} δὲ ταχέως\FnFormGloss{b}{ταχέως}{quickly} πρὸς ὑμᾶς, ἐὰν ὁ κύριος θελήσῃ,\FnParse{c}{aorist active subjunctive 3rd singular} καὶ γνώσομαι\FnParse{d}{future middle indicative 1st singular} οὐ τὸν λόγον τῶν πεφυσιωμένων\FnParseFormGloss{e}{perfect passive participle genitive masculine plural}{φυσιόω}{to puff up} ἀλλὰ τὴν δύναμιν, 
\MainTextVerseMark{4}{20}οὐ γὰρ ἐν λόγῳ ἡ βασιλεία τοῦ θεοῦ ἀλλ’ ἐν δυνάμει. 
\MainTextVerseMark{4}{21}τί θέλετε;\FnParse{a}{present active indicative 2nd plural} ἐν ῥάβδῳ\FnParseFormGloss{b}{dative feminine singular}{ῥάβδος}{rod} ἔλθω\FnParse{c}{aorist active subjunctive 1st singular} πρὸς ὑμᾶς, ἢ ἐν ἀγάπῃ πνεύματί τε πραΰτητος;\FnParseFormGloss{d}{genitive feminine singular}{πραΰτης}{gentleness} 
\ChapterHeader{5}{Chapter 5}

\MainTextVerseMark{5}{1}Ὅλως\FnFormGloss{a}{ὅλως}{wholly} ἀκούεται\FnParse{b}{present passive indicative 3rd singular} ἐν ὑμῖν πορνεία,\FnParseFormGloss{c}{nominative feminine singular}{πορνεία}{sexual immorality} καὶ τοιαύτη πορνεία\FnParseFormGloss{d}{nominative feminine singular}{πορνεία}{sexual immorality} ἥτις οὐδὲ ἐν τοῖς ⸀ἔθνεσιν, ὥστε γυναῖκά τινα τοῦ πατρὸς ἔχειν.\FnParse{e}{present active infinitive} 
\MainTextVerseMark{5}{2}καὶ ὑμεῖς πεφυσιωμένοι\FnParseFormGloss{a}{perfect passive participle nominative masculine plural}{φυσιόω}{to puff up} ἐστέ,\FnParse{b}{present active indicative 2nd plural} καὶ οὐχὶ μᾶλλον ἐπενθήσατε,\FnParseFormGloss{c}{aorist active indicative 2nd plural}{πενθέω}{to mourn} ἵνα ⸀ἀρθῇ\FnParse{d}{aorist passive subjunctive 3rd singular} ἐκ μέσου ὑμῶν ὁ τὸ ἔργον τοῦτο ⸀ποιήσας;\FnParse{e}{aorist active participle nominative masculine singular} 
\MainTextVerseMark{5}{3}Ἐγὼ μὲν γάρ, ⸀ἀπὼν\FnParseFormGloss{a}{present active participle nominative masculine singular}{ἄπειμι}{to be absent} τῷ σώματι παρὼν\FnParseFormGloss{b}{present active participle nominative masculine singular}{πάρειμι}{to be present} δὲ τῷ πνεύματι, ἤδη κέκρικα\FnParse{c}{perfect active indicative 1st singular} ὡς παρὼν\FnParseFormGloss{d}{present active participle nominative masculine singular}{πάρειμι}{to be present} τὸν οὕτως τοῦτο κατεργασάμενον\FnParseFormGloss{e}{aorist middle participle accusative masculine singular}{κατεργάζομαι}{to produce} 
\MainTextVerseMark{5}{4}ἐν τῷ ὀνόματι τοῦ κυρίου ἡμῶν ⸀Ἰησοῦ, συναχθέντων\FnParse{a}{aorist passive participle genitive masculine plural} ὑμῶν καὶ τοῦ ἐμοῦ πνεύματος σὺν τῇ δυνάμει τοῦ κυρίου ἡμῶν ⸁Ἰησοῦ, 
\MainTextVerseMark{5}{5}παραδοῦναι\FnParse{a}{aorist active infinitive} τὸν τοιοῦτον τῷ Σατανᾷ εἰς ὄλεθρον\FnParseFormGloss{b}{accusative masculine singular}{ὄλεθρος}{destruction} τῆς σαρκός, ἵνα τὸ πνεῦμα σωθῇ\FnParse{c}{aorist passive subjunctive 3rd singular} ἐν τῇ ἡμέρᾳ τοῦ ⸀κυρίου. 
\MainTextVerseMark{5}{6}Οὐ καλὸν τὸ καύχημα\FnParseFormGloss{a}{nominative neuter singular}{καύχημα}{something to boast about} ὑμῶν. οὐκ οἴδατε\FnParse{b}{perfect active indicative 2nd plural} ὅτι μικρὰ ζύμη\FnParseFormGloss{c}{nominative feminine singular}{ζύμη}{yeast} ὅλον τὸ φύραμα\FnParseFormGloss{d}{accusative neuter singular}{φύραμα}{lump} ζυμοῖ;\FnParseFormGloss{e}{present active indicative 3rd singular}{ζυμόω}{to leaven} 
\MainTextVerseMark{5}{7}ἐκκαθάρατε\FnParseFormGloss{a}{aorist active imperative 2nd plural}{ἐκκαθαίρω}{to cleanse} τὴν παλαιὰν\FnParseFormGloss{b}{accusative feminine singular}{παλαιός}{old} ζύμην,\FnParseFormGloss{c}{accusative feminine singular}{ζύμη}{yeast} ἵνα ἦτε\FnParse{d}{present active subjunctive 2nd plural} νέον\FnParseFormGloss{e}{nominative neuter singular}{νέος}{new} φύραμα,\FnParseFormGloss{f}{nominative neuter singular}{φύραμα}{lump} καθώς ἐστε\FnParse{g}{present active indicative 2nd plural} ἄζυμοι.\FnParseFormGloss{h}{nominative masculine plural}{ἄζυμος}{unleavened} καὶ γὰρ τὸ πάσχα\FnParseFormGloss{i}{nominative neuter singular}{πάσχα}{Passover} ⸀ἡμῶν ἐτύθη\FnParseFormGloss{j}{aorist passive indicative 3rd singular}{θύω}{to kill} Χριστός· 
\MainTextVerseMark{5}{8}ὥστε ἑορτάζωμεν,\FnParseFormGloss{a}{present active subjunctive 1st plural}{ἑορτάζω}{to celebrate a festival} μὴ ἐν ζύμῃ\FnParseFormGloss{b}{dative feminine singular}{ζύμη}{yeast} παλαιᾷ\FnParseFormGloss{c}{dative feminine singular}{παλαιός}{old} μηδὲ ἐν ζύμῃ\FnParseFormGloss{d}{dative feminine singular}{ζύμη}{yeast} κακίας\FnParseFormGloss{e}{genitive feminine singular}{κακία}{evil} καὶ πονηρίας,\FnParseFormGloss{f}{genitive feminine singular}{πονηρία}{evil} ἀλλ’ ἐν ἀζύμοις\FnParseFormGloss{g}{dative neuter plural}{ἄζυμος}{unleavened} εἰλικρινείας\FnParseFormGloss{h}{genitive feminine singular}{εἰλικρίνεια}{sincerity} καὶ ἀληθείας. 
\MainTextVerseMark{5}{9}Ἔγραψα\FnParse{a}{aorist active indicative 1st singular} ὑμῖν ἐν τῇ ἐπιστολῇ\FnParseFormGloss{b}{dative feminine singular}{ἐπιστολή}{letter} μὴ συναναμίγνυσθαι\FnParseFormGloss{c}{present middle infinitive}{συναναμίγνυμι}{to associate with} πόρνοις,\FnParseFormGloss{d}{dative masculine plural}{πόρνος}{a fornicator} 
\MainTextVerseMark{5}{10}⸀οὐ πάντως\FnFormGloss{a}{πάντως}{surely} τοῖς πόρνοις\FnParseFormGloss{b}{dative masculine plural}{πόρνος}{a fornicator} τοῦ κόσμου τούτου ἢ τοῖς πλεονέκταις\FnParseFormGloss{c}{dative masculine plural}{πλεονέκτης}{greedy person} ⸀καὶ ἅρπαξιν\FnParseFormGloss{d}{dative masculine plural}{ἅρπαξ}{swindling} ἢ εἰδωλολάτραις,\FnParseFormGloss{e}{dative masculine plural}{εἰδωλολάτρης}{idolater} ἐπεὶ\FnFormGloss{f}{ἐπεί}{since} ⸀ὠφείλετε\FnParse{g}{imperfect active indicative 2nd plural} ἄρα ἐκ τοῦ κόσμου ἐξελθεῖν.\FnParse{h}{aorist active infinitive} 
\MainTextVerseMark{5}{11}νῦν δὲ ἔγραψα\FnParse{a}{aorist active indicative 1st singular} ὑμῖν μὴ συναναμίγνυσθαι\FnParseFormGloss{b}{present middle infinitive}{συναναμίγνυμι}{to associate with} ἐάν τις ἀδελφὸς ὀνομαζόμενος\FnParseFormGloss{c}{present passive participle nominative masculine singular}{ὀνομάζω}{to give a name} ᾖ\FnParse{d}{present active subjunctive 3rd singular} πόρνος\FnParseFormGloss{e}{nominative masculine singular}{πόρνος}{a fornicator} ἢ πλεονέκτης\FnParseFormGloss{f}{nominative masculine singular}{πλεονέκτης}{greedy person} ἢ εἰδωλολάτρης\FnParseFormGloss{g}{nominative masculine singular}{εἰδωλολάτρης}{idolater} ἢ λοίδορος\FnParseFormGloss{h}{nominative masculine singular}{λοίδορος}{slanderer} ἢ μέθυσος\FnParseFormGloss{i}{nominative masculine singular}{μέθυσος}{drunkard} ἢ ἅρπαξ,\FnParseFormGloss{j}{nominative masculine singular}{ἅρπαξ}{swindling} τῷ τοιούτῳ μηδὲ συνεσθίειν.\FnParseFormGloss{k}{present active infinitive}{συνεσθίω}{to eat with} 
\MainTextVerseMark{5}{12}τί γάρ ⸀μοι τοὺς ἔξω κρίνειν;\FnParse{a}{present active infinitive} οὐχὶ τοὺς ἔσω\FnFormGloss{b}{ἔσω}{in} ὑμεῖς κρίνετε,\FnParse{c}{present active indicative 2nd plural} 
\MainTextVerseMark{5}{13}τοὺς δὲ ἔξω ὁ θεὸς ⸀κρίνει;\FnParse{a}{present active indicative 3rd singular} ⸀ἐξάρατε\FnParseFormGloss{b}{aorist active imperative 2nd plural}{ἐξαίρω}{to expel} τὸν πονηρὸν ἐξ ὑμῶν αὐτῶν. 
\ChapterHeader{6}{Chapter 6}

\MainTextVerseMark{6}{1}Τολμᾷ\FnParseFormGloss{a}{present active indicative 3rd singular}{τολμάω}{to dare} τις ὑμῶν πρᾶγμα\FnParseFormGloss{b}{accusative neuter singular}{πρᾶγμα}{thing} ἔχων\FnParse{c}{present active participle nominative masculine singular} πρὸς τὸν ἕτερον κρίνεσθαι\FnParse{d}{present passive infinitive} ἐπὶ τῶν ἀδίκων,\FnParseFormGloss{e}{genitive masculine plural}{ἄδικος}{unjust} καὶ οὐχὶ ἐπὶ τῶν ἁγίων; 
\MainTextVerseMark{6}{2}⸀ἢ οὐκ οἴδατε\FnParse{a}{perfect active indicative 2nd plural} ὅτι οἱ ἅγιοι τὸν κόσμον κρινοῦσιν;\FnParse{b}{future active indicative 3rd plural} καὶ εἰ ἐν ὑμῖν κρίνεται\FnParse{c}{present passive indicative 3rd singular} ὁ κόσμος, ἀνάξιοί\FnParseFormGloss{d}{nominative masculine plural}{ἀνάξιος}{not competent} ἐστε\FnParse{e}{present active indicative 2nd plural} κριτηρίων\FnParseFormGloss{f}{genitive neuter plural}{κριτήριον}{court of law} ἐλαχίστων;\FnParseFormGloss{g}{genitive neuter plural}{ἐλάχιστος}{least} 
\MainTextVerseMark{6}{3}οὐκ οἴδατε\FnParse{a}{perfect active indicative 2nd plural} ὅτι ἀγγέλους κρινοῦμεν,\FnParse{b}{future active indicative 1st plural} μήτιγε\FnFormGloss{c}{μήτιγε}{how much more} βιωτικά;\FnParseFormGloss{d}{accusative neuter plural}{βιωτικός}{of this life} 
\MainTextVerseMark{6}{4}βιωτικὰ\FnParseFormGloss{a}{accusative neuter plural}{βιωτικός}{of this life} μὲν οὖν κριτήρια\FnParseFormGloss{b}{accusative neuter plural}{κριτήριον}{court of law} ἐὰν ἔχητε,\FnParse{c}{present active subjunctive 2nd plural} τοὺς ἐξουθενημένους\FnParseFormGloss{d}{perfect passive participle accusative masculine plural}{ἐξουθενέω}{to treat with contempt} ἐν τῇ ἐκκλησίᾳ, τούτους καθίζετε;\FnParse{e}{present active indicative 2nd plural} 
\MainTextVerseMark{6}{5}πρὸς ἐντροπὴν\FnParseFormGloss{a}{accusative feminine singular}{ἐντροπή}{shame} ὑμῖν λέγω.\FnParse{b}{present active indicative 1st singular} οὕτως οὐκ ἔνι\FnParseFormGloss{c}{present active indicative 3rd singular}{ἔνι}{there is} ἐν ὑμῖν ⸂οὐδεὶς σοφὸς⸃\FnParseFormGloss{d}{nominative masculine singular}{σοφός}{wise} ὃς δυνήσεται\FnParse{e}{future middle indicative 3rd singular} διακρῖναι\FnParseFormGloss{f}{aorist active infinitive}{διακρίνω}{to make a distinction} ἀνὰ\FnFormGloss{g}{ἀνά}{each} μέσον τοῦ ⸀ἀδελφοῦ αὐτοῦ, 
\MainTextVerseMark{6}{6}ἀλλὰ ἀδελφὸς μετὰ ἀδελφοῦ κρίνεται,\FnParse{a}{present passive indicative 3rd singular} καὶ τοῦτο ἐπὶ ἀπίστων;\FnParseFormGloss{b}{genitive masculine plural}{ἄπιστος}{unbelieving} 
\MainTextVerseMark{6}{7}ἤδη μὲν ⸀οὖν ὅλως\FnFormGloss{a}{ὅλως}{wholly} ἥττημα\FnParseFormGloss{b}{nominative neuter singular}{ἥττημα}{loss} ὑμῖν ἐστιν\FnParse{c}{present active indicative 3rd singular} ὅτι κρίματα\FnParseFormGloss{d}{accusative neuter plural}{κρίμα}{judgment} ἔχετε\FnParse{e}{present active indicative 2nd plural} μεθ’ ἑαυτῶν· διὰ τί οὐχὶ μᾶλλον ἀδικεῖσθε;\FnParseFormGloss{f}{present passive indicative 2nd plural}{ἀδικέω}{to do wrong} διὰ τί οὐχὶ μᾶλλον ἀποστερεῖσθε;\FnParseFormGloss{g}{present passive indicative 2nd plural}{ἀποστερέω}{to defraud} 
\MainTextVerseMark{6}{8}ἀλλὰ ὑμεῖς ἀδικεῖτε\FnParseFormGloss{a}{present active indicative 2nd plural}{ἀδικέω}{to do wrong} καὶ ἀποστερεῖτε,\FnParseFormGloss{b}{present active indicative 2nd plural}{ἀποστερέω}{to defraud} καὶ ⸀τοῦτο ἀδελφούς. 
\MainTextVerseMark{6}{9}Ἢ οὐκ οἴδατε\FnParse{a}{perfect active indicative 2nd plural} ὅτι ἄδικοι\FnParseFormGloss{b}{nominative masculine plural}{ἄδικος}{unjust} ⸂θεοῦ βασιλείαν⸃ οὐ κληρονομήσουσιν;\FnParseFormGloss{c}{future active indicative 3rd plural}{κληρονομέω}{to inherit} μὴ πλανᾶσθε·\FnParse{d}{present passive imperative 2nd plural} οὔτε πόρνοι\FnParseFormGloss{e}{nominative masculine plural}{πόρνος}{a fornicator} οὔτε εἰδωλολάτραι\FnParseFormGloss{f}{nominative masculine plural}{εἰδωλολάτρης}{idolater} οὔτε μοιχοὶ\FnParseFormGloss{g}{nominative masculine plural}{μοιχός}{adulterer} οὔτε μαλακοὶ\FnParseFormGloss{h}{nominative masculine plural}{μαλακός}{fine} οὔτε ἀρσενοκοῖται\FnParseFormGloss{i}{nominative masculine plural}{ἀρσενοκοίτης}{one engaging in homosexual acts} 
\MainTextVerseMark{6}{10}οὔτε ⸂κλέπται\FnParseFormGloss{a}{nominative masculine plural}{κλέπτης}{thief} οὔτε πλεονέκται⸃,\FnParseFormGloss{b}{nominative masculine plural}{πλεονέκτης}{greedy person} ⸀οὐ μέθυσοι,\FnParseFormGloss{c}{nominative masculine plural}{μέθυσος}{drunkard} οὐ λοίδοροι,\FnParseFormGloss{d}{nominative masculine plural}{λοίδορος}{slanderer} οὐχ ἅρπαγες\FnParseFormGloss{e}{nominative masculine plural}{ἅρπαξ}{swindling} βασιλείαν ⸀θεοῦ κληρονομήσουσιν.\FnParseFormGloss{f}{future active indicative 3rd plural}{κληρονομέω}{to inherit} 
\MainTextVerseMark{6}{11}καὶ ταῦτά τινες ἦτε·\FnParse{a}{imperfect active indicative 2nd plural} ἀλλὰ ἀπελούσασθε,\FnParseFormGloss{b}{aorist middle indicative 2nd plural}{ἀπολούομαι}{to wash away} ἀλλὰ ἡγιάσθητε,\FnParseFormGloss{c}{aorist passive indicative 2nd plural}{ἁγιάζω}{to sanctify} ἀλλὰ ἐδικαιώθητε\FnParse{d}{aorist passive indicative 2nd plural} ἐν τῷ ὀνόματι τοῦ κυρίου ⸀Ἰησοῦ καὶ ἐν τῷ πνεύματι τοῦ θεοῦ ἡμῶν. 
\MainTextVerseMark{6}{12}Πάντα μοι ἔξεστιν·\FnParse{a}{present active indicative 3rd singular} ἀλλ’ οὐ πάντα συμφέρει.\FnParseFormGloss{b}{present active indicative 3rd singular}{συμφέρω}{to bring together} πάντα μοι ἔξεστιν·\FnParse{c}{present active indicative 3rd singular} ἀλλ’ οὐκ ἐγὼ ἐξουσιασθήσομαι\FnParseFormGloss{d}{future passive indicative 1st singular}{ἐξουσιάζω}{to have power over} ὑπό τινος. 
\MainTextVerseMark{6}{13}τὰ βρώματα\FnParseFormGloss{a}{nominative neuter plural}{βρῶμα}{food} τῇ κοιλίᾳ,\FnParseFormGloss{b}{dative feminine singular}{κοιλία}{any and all internal organs} καὶ ἡ κοιλία\FnParseFormGloss{c}{nominative feminine singular}{κοιλία}{any and all internal organs} τοῖς βρώμασιν·\FnParseFormGloss{d}{dative neuter plural}{βρῶμα}{food} ὁ δὲ θεὸς καὶ ταύτην καὶ ταῦτα καταργήσει.\FnParseFormGloss{e}{future active indicative 3rd singular}{καταργέω}{to nullify} τὸ δὲ σῶμα οὐ τῇ πορνείᾳ\FnParseFormGloss{f}{dative feminine singular}{πορνεία}{sexual immorality} ἀλλὰ τῷ κυρίῳ, καὶ ὁ κύριος τῷ σώματι· 
\MainTextVerseMark{6}{14}ὁ δὲ θεὸς καὶ τὸν κύριον ἤγειρεν\FnParse{a}{aorist active indicative 3rd singular} καὶ ἡμᾶς ἐξεγερεῖ\FnParseFormGloss{b}{future active indicative 3rd singular}{ἐξεγείρω}{to raise} διὰ τῆς δυνάμεως αὐτοῦ. 
\MainTextVerseMark{6}{15}οὐκ οἴδατε\FnParse{a}{perfect active indicative 2nd plural} ὅτι τὰ σώματα ὑμῶν μέλη Χριστοῦ ἐστιν;\FnParse{b}{present active indicative 3rd singular} ἄρας\FnParse{c}{aorist active participle nominative masculine singular} οὖν τὰ μέλη τοῦ Χριστοῦ ποιήσω\FnParse{d}{aorist active subjunctive 1st singular} πόρνης\FnParseFormGloss{e}{genitive feminine singular}{πόρνη}{prostitute} μέλη; μὴ γένοιτο.\FnParse{f}{aorist middle optative 3rd singular} 
\MainTextVerseMark{6}{16}⸀ἢ οὐκ οἴδατε\FnParse{a}{perfect active indicative 2nd plural} ὅτι ὁ κολλώμενος\FnParseFormGloss{b}{present passive participle nominative masculine singular}{κολλάομαι}{to join} τῇ πόρνῃ\FnParseFormGloss{c}{dative feminine singular}{πόρνη}{prostitute} ἓν σῶμά ἐστιν;\FnParse{d}{present active indicative 3rd singular} Ἔσονται\FnParse{e}{future middle indicative 3rd plural} γάρ, φησίν,\FnParse{f}{present active indicative 3rd singular} οἱ δύο εἰς σάρκα μίαν. 
\MainTextVerseMark{6}{17}ὁ δὲ κολλώμενος\FnParseFormGloss{a}{present passive participle nominative masculine singular}{κολλάομαι}{to join} τῷ κυρίῳ ἓν πνεῦμά ἐστιν.\FnParse{b}{present active indicative 3rd singular} 
\MainTextVerseMark{6}{18}φεύγετε\FnParseFormGloss{a}{present active imperative 2nd plural}{φεύγω}{flee} τὴν πορνείαν·\FnParseFormGloss{b}{accusative feminine singular}{πορνεία}{sexual immorality} πᾶν ἁμάρτημα\FnParseFormGloss{c}{nominative neuter singular}{ἁμάρτημα}{sin} ὃ ἐὰν ποιήσῃ\FnParse{d}{aorist active subjunctive 3rd singular} ἄνθρωπος ἐκτὸς\FnFormGloss{e}{ἐκτός}{outside} τοῦ σώματός ἐστιν,\FnParse{f}{present active indicative 3rd singular} ὁ δὲ πορνεύων\FnParseFormGloss{g}{present active participle nominative masculine singular}{πορνεύω}{to commit fornication} εἰς τὸ ἴδιον σῶμα ἁμαρτάνει.\FnParse{h}{present active indicative 3rd singular} 
\MainTextVerseMark{6}{19}ἢ οὐκ οἴδατε\FnParse{a}{perfect active indicative 2nd plural} ὅτι τὸ σῶμα ὑμῶν ναὸς τοῦ ἐν ὑμῖν ἁγίου πνεύματός ἐστιν,\FnParse{b}{present active indicative 3rd singular} οὗ ἔχετε\FnParse{c}{present active indicative 2nd plural} ἀπὸ θεοῦ; καὶ οὐκ ἐστὲ\FnParse{d}{present active indicative 2nd plural} ἑαυτῶν, 
\MainTextVerseMark{6}{20}ἠγοράσθητε\FnParse{a}{aorist passive indicative 2nd plural} γὰρ τιμῆς· δοξάσατε\FnParse{b}{aorist active imperative 2nd plural} δὴ\FnFormGloss{c}{δή}{indeed} τὸν θεὸν ἐν τῷ σώματι ⸀ὑμῶν. 
\ChapterHeader{7}{Chapter 7}

\MainTextVerseMark{7}{1}Περὶ δὲ ὧν ⸀ἐγράψατε,\FnParse{a}{aorist active indicative 2nd plural} καλὸν ἀνθρώπῳ γυναικὸς μὴ ἅπτεσθαι·\FnParse{b}{present middle infinitive} 
\MainTextVerseMark{7}{2}διὰ δὲ τὰς πορνείας\FnParseFormGloss{a}{accusative feminine plural}{πορνεία}{sexual immorality} ἕκαστος τὴν ἑαυτοῦ γυναῖκα ἐχέτω,\FnParse{b}{present active imperative 3rd singular} καὶ ἑκάστη τὸν ἴδιον ἄνδρα ἐχέτω.\FnParse{c}{present active imperative 3rd singular} 
\MainTextVerseMark{7}{3}τῇ γυναικὶ ὁ ἀνὴρ τὴν ⸀ὀφειλὴν\FnParseFormGloss{a}{accusative feminine singular}{ὀφειλή}{debt} ἀποδιδότω,\FnParse{b}{present active imperative 3rd singular} ὁμοίως δὲ καὶ ἡ γυνὴ τῷ ἀνδρί. 
\MainTextVerseMark{7}{4}ἡ γυνὴ τοῦ ἰδίου σώματος οὐκ ἐξουσιάζει\FnParseFormGloss{a}{present active indicative 3rd singular}{ἐξουσιάζω}{to have power over} ἀλλὰ ὁ ἀνήρ· ὁμοίως δὲ καὶ ὁ ἀνὴρ τοῦ ἰδίου σώματος οὐκ ἐξουσιάζει\FnParseFormGloss{b}{present active indicative 3rd singular}{ἐξουσιάζω}{to have power over} ἀλλὰ ἡ γυνή. 
\MainTextVerseMark{7}{5}μὴ ἀποστερεῖτε\FnParseFormGloss{a}{present active imperative 2nd plural}{ἀποστερέω}{to defraud} ἀλλήλους, εἰ μήτι\FnFormGloss{b}{μήτι}{[interrogative particle]} ἂν ἐκ συμφώνου\FnParseFormGloss{c}{genitive neuter singular}{σύμφωνος}{mutually consenting} πρὸς καιρὸν ἵνα ⸀σχολάσητε\FnParseFormGloss{d}{aorist active subjunctive 2nd plural}{σχολάζω}{to devote oneself to} ⸀τῇ προσευχῇ καὶ πάλιν ἐπὶ τὸ αὐτὸ ⸀ἦτε,\FnParse{e}{present active subjunctive 2nd plural} ἵνα μὴ πειράζῃ\FnParse{f}{present active subjunctive 3rd singular} ὑμᾶς ὁ Σατανᾶς διὰ τὴν ἀκρασίαν\FnParseFormGloss{g}{accusative feminine singular}{ἀκρασία}{lack of self-control} ὑμῶν. 
\MainTextVerseMark{7}{6}τοῦτο δὲ λέγω\FnParse{a}{present active indicative 1st singular} κατὰ συγγνώμην,\FnParseFormGloss{b}{accusative feminine singular}{συγγνώμη}{concession} οὐ κατ’ ἐπιταγήν.\FnParseFormGloss{c}{accusative feminine singular}{ἐπιταγή}{command} 
\MainTextVerseMark{7}{7}θέλω\FnParse{a}{present active indicative 1st singular} ⸀δὲ πάντας ἀνθρώπους εἶναι\FnParse{b}{present active infinitive} ὡς καὶ ἐμαυτόν· ἀλλὰ ἕκαστος ἴδιον ⸂ἔχει\FnParse{c}{present active indicative 3rd singular} χάρισμα⸃\FnParseFormGloss{d}{accusative neuter singular}{χάρισμα}{gracious gift} ἐκ θεοῦ, ⸀ὁ μὲν οὕτως, ⸁ὁ δὲ οὕτως. 
\MainTextVerseMark{7}{8}Λέγω\FnParse{a}{present active indicative 1st singular} δὲ τοῖς ἀγάμοις\FnParseFormGloss{b}{dative masculine plural}{ἄγαμος}{unmarried} καὶ ταῖς χήραις,\FnParseFormGloss{c}{dative feminine plural}{χήρα}{widow} καλὸν ⸀αὐτοῖς ἐὰν μείνωσιν\FnParse{d}{aorist active subjunctive 3rd plural} ὡς κἀγώ· 
\MainTextVerseMark{7}{9}εἰ δὲ οὐκ ἐγκρατεύονται,\FnParseFormGloss{a}{present middle indicative 3rd plural}{ἐγκρατεύομαι}{to have control} γαμησάτωσαν,\FnParseFormGloss{b}{aorist active imperative 3rd plural}{γαμέω}{to marry} κρεῖττον\FnParseFormGloss{c}{nominative neuter singular}{κρείττων}{better} γάρ ἐστιν\FnParse{d}{present active indicative 3rd singular} ⸀γαμῆσαι\FnParseFormGloss{e}{aorist active infinitive}{γαμέω}{to marry} ἢ πυροῦσθαι.\FnParseFormGloss{f}{present passive infinitive}{πυρόομαι}{to burn} 
\MainTextVerseMark{7}{10}Τοῖς δὲ γεγαμηκόσιν\FnParseFormGloss{a}{perfect active participle dative masculine plural}{γαμέω}{to marry} παραγγέλλω,\FnParse{b}{present active indicative 1st singular} οὐκ ἐγὼ ἀλλὰ ὁ κύριος, γυναῖκα ἀπὸ ἀνδρὸς μὴ χωρισθῆναι—\FnParseFormGloss{c}{aorist passive infinitive}{χωρίζω}{to divide} 
\MainTextVerseMark{7}{11}ἐὰν δὲ καὶ χωρισθῇ,\FnParseFormGloss{a}{aorist passive subjunctive 3rd singular}{χωρίζω}{to divide} μενέτω\FnParse{b}{present active imperative 3rd singular} ἄγαμος\FnParseFormGloss{c}{nominative feminine singular}{ἄγαμος}{unmarried} ἢ τῷ ἀνδρὶ καταλλαγήτω—\FnParseFormGloss{d}{aorist passive imperative 3rd singular}{καταλλάσσω}{to reconcile} καὶ ἄνδρα γυναῖκα μὴ ἀφιέναι.\FnParse{e}{present active infinitive} 
\MainTextVerseMark{7}{12}Τοῖς δὲ λοιποῖς ⸂λέγω\FnParse{a}{present active indicative 1st singular} ἐγώ⸃, οὐχ ὁ κύριος· εἴ τις ἀδελφὸς γυναῖκα ἔχει\FnParse{b}{present active indicative 3rd singular} ἄπιστον,\FnParseFormGloss{c}{accusative feminine singular}{ἄπιστος}{unbelieving} καὶ αὕτη συνευδοκεῖ\FnParseFormGloss{d}{present active indicative 3rd singular}{συνευδοκέω}{to approve of} οἰκεῖν\FnParseFormGloss{e}{present active infinitive}{οἰκέω}{to live} μετ’ αὐτοῦ, μὴ ἀφιέτω\FnParse{f}{present active imperative 3rd singular} αὐτήν· 
\MainTextVerseMark{7}{13}καὶ γυνὴ ⸂εἴ τις⸃ ἔχει\FnParse{a}{present active indicative 3rd singular} ἄνδρα ἄπιστον,\FnParseFormGloss{b}{accusative masculine singular}{ἄπιστος}{unbelieving} καὶ ⸀οὗτος συνευδοκεῖ\FnParseFormGloss{c}{present active indicative 3rd singular}{συνευδοκέω}{to approve of} οἰκεῖν\FnParseFormGloss{d}{present active infinitive}{οἰκέω}{to live} μετ’ αὐτῆς, μὴ ἀφιέτω\FnParse{e}{present active imperative 3rd singular} ⸂τὸν ἄνδρα⸃. 
\MainTextVerseMark{7}{14}ἡγίασται\FnParseFormGloss{a}{perfect passive indicative 3rd singular}{ἁγιάζω}{to sanctify} γὰρ ὁ ἀνὴρ ὁ ἄπιστος\FnParseFormGloss{b}{nominative masculine singular}{ἄπιστος}{unbelieving} ἐν τῇ γυναικί, καὶ ἡγίασται\FnParseFormGloss{c}{perfect passive indicative 3rd singular}{ἁγιάζω}{to sanctify} ἡ γυνὴ ἡ ἄπιστος\FnParseFormGloss{d}{nominative feminine singular}{ἄπιστος}{unbelieving} ἐν τῷ ⸀ἀδελφῷ· ἐπεὶ\FnFormGloss{e}{ἐπεί}{since} ἄρα τὰ τέκνα ὑμῶν ἀκάθαρτά ἐστιν,\FnParse{f}{present active indicative 3rd singular} νῦν δὲ ἅγιά ἐστιν.\FnParse{g}{present active indicative 3rd singular} 
\MainTextVerseMark{7}{15}εἰ δὲ ὁ ἄπιστος\FnParseFormGloss{a}{nominative masculine singular}{ἄπιστος}{unbelieving} χωρίζεται,\FnParseFormGloss{b}{present passive indicative 3rd singular}{χωρίζω}{to divide} χωριζέσθω·\FnParseFormGloss{c}{present passive imperative 3rd singular}{χωρίζω}{to divide} οὐ δεδούλωται\FnParseFormGloss{d}{perfect passive indicative 3rd singular}{δουλόω}{to enslave} ὁ ἀδελφὸς ἢ ἡ ἀδελφὴ\FnParseFormGloss{e}{nominative feminine singular}{ἀδελφή}{sister} ἐν τοῖς τοιούτοις, ἐν δὲ εἰρήνῃ κέκληκεν\FnParse{f}{perfect active indicative 3rd singular} ⸀ἡμᾶς ὁ θεός. 
\MainTextVerseMark{7}{16}τί γὰρ οἶδας,\FnParse{a}{perfect active indicative 2nd singular} γύναι, εἰ τὸν ἄνδρα σώσεις;\FnParse{b}{future active indicative 2nd singular} ἢ τί οἶδας,\FnParse{c}{perfect active indicative 2nd singular} ἄνερ, εἰ τὴν γυναῖκα σώσεις;\FnParse{d}{future active indicative 2nd singular} 
\MainTextVerseMark{7}{17}Εἰ μὴ ἑκάστῳ ὡς ⸀ἐμέρισεν\FnParseFormGloss{a}{aorist active indicative 3rd singular}{μερίζω}{to give} ὁ ⸀κύριος, ἕκαστον ὡς κέκληκεν\FnParse{b}{perfect active indicative 3rd singular} ὁ ⸀θεός, οὕτως περιπατείτω·\FnParse{c}{present active imperative 3rd singular} καὶ οὕτως ἐν ταῖς ἐκκλησίαις πάσαις διατάσσομαι.\FnParseFormGloss{d}{present middle indicative 1st singular}{διατάσσω}{to command} 
\MainTextVerseMark{7}{18}περιτετμημένος\FnParseFormGloss{a}{perfect passive participle nominative masculine singular}{περιτέμνω}{to circumcise} τις ἐκλήθη;\FnParse{b}{aorist passive indicative 3rd singular} μὴ ἐπισπάσθω·\FnParseFormGloss{c}{present middle imperative 3rd singular}{ἐπισπάομαι}{to conceal circumcision} ἐν ἀκροβυστίᾳ\FnParseFormGloss{d}{dative feminine singular}{ἀκροβυστία}{uncircumcision} ⸂κέκληταί\FnParse{e}{perfect passive indicative 3rd singular} τις⸃; μὴ περιτεμνέσθω.\FnParseFormGloss{f}{present passive imperative 3rd singular}{περιτέμνω}{to circumcise} 
\MainTextVerseMark{7}{19}ἡ περιτομὴ οὐδέν ἐστιν,\FnParse{a}{present active indicative 3rd singular} καὶ ἡ ἀκροβυστία\FnParseFormGloss{b}{nominative feminine singular}{ἀκροβυστία}{uncircumcision} οὐδέν ἐστιν,\FnParse{c}{present active indicative 3rd singular} ἀλλὰ τήρησις\FnParseFormGloss{d}{nominative feminine singular}{τήρησις}{jail} ἐντολῶν θεοῦ. 
\MainTextVerseMark{7}{20}ἕκαστος ἐν τῇ κλήσει\FnParseFormGloss{a}{dative feminine singular}{κλῆσις}{call} ᾗ ἐκλήθη\FnParse{b}{aorist passive indicative 3rd singular} ἐν ταύτῃ μενέτω.\FnParse{c}{present active imperative 3rd singular} 
\MainTextVerseMark{7}{21}Δοῦλος ἐκλήθης;\FnParse{a}{aorist passive indicative 2nd singular} μή σοι μελέτω·\FnParseFormGloss{b}{present active imperative 3rd singular}{μέλει}{it is a care} ἀλλ’ εἰ καὶ δύνασαι\FnParse{c}{present middle indicative 2nd singular} ἐλεύθερος\FnParseFormGloss{d}{nominative masculine singular}{ἐλεύθερος}{free} γενέσθαι,\FnParse{e}{aorist middle infinitive} μᾶλλον χρῆσαι.\FnParseFormGloss{f}{aorist middle imperative 2nd singular}{χράομαι}{to make use of} 
\MainTextVerseMark{7}{22}ὁ γὰρ ἐν κυρίῳ κληθεὶς\FnParse{a}{aorist passive participle nominative masculine singular} δοῦλος ἀπελεύθερος\FnParseFormGloss{b}{nominative masculine singular}{ἀπελεύθερος}{freedman} κυρίου ἐστίν·\FnParse{c}{present active indicative 3rd singular} ⸀ὁμοίως ὁ ἐλεύθερος\FnParseFormGloss{d}{nominative masculine singular}{ἐλεύθερος}{free} κληθεὶς\FnParse{e}{aorist passive participle nominative masculine singular} δοῦλός ἐστιν\FnParse{f}{present active indicative 3rd singular} Χριστοῦ. 
\MainTextVerseMark{7}{23}τιμῆς ἠγοράσθητε·\FnParse{a}{aorist passive indicative 2nd plural} μὴ γίνεσθε\FnParse{b}{present middle imperative 2nd plural} δοῦλοι ἀνθρώπων. 
\MainTextVerseMark{7}{24}ἕκαστος ἐν ᾧ ἐκλήθη,\FnParse{a}{aorist passive indicative 3rd singular} ἀδελφοί, ἐν τούτῳ μενέτω\FnParse{b}{present active imperative 3rd singular} παρὰ θεῷ. 
\MainTextVerseMark{7}{25}Περὶ δὲ τῶν παρθένων\FnParseFormGloss{a}{genitive feminine plural}{παρθένος}{virgin} ἐπιταγὴν\FnParseFormGloss{b}{accusative feminine singular}{ἐπιταγή}{command} κυρίου οὐκ ἔχω,\FnParse{c}{present active indicative 1st singular} γνώμην\FnParseFormGloss{d}{accusative feminine singular}{γνώμη}{purpose} δὲ δίδωμι\FnParse{e}{present active indicative 1st singular} ὡς ἠλεημένος\FnParse{f}{perfect passive participle nominative masculine singular} ὑπὸ κυρίου πιστὸς εἶναι.\FnParse{g}{present active infinitive} 
\MainTextVerseMark{7}{26}νομίζω\FnParseFormGloss{a}{present active indicative 1st singular}{νομίζω}{to think} οὖν τοῦτο καλὸν ὑπάρχειν\FnParse{b}{present active infinitive} διὰ τὴν ἐνεστῶσαν\FnParseFormGloss{c}{perfect active participle accusative feminine singular}{ἐνίστημι}{to be present} ἀνάγκην,\FnParseFormGloss{d}{accusative feminine singular}{ἀνάγκη}{necessity} ὅτι καλὸν ἀνθρώπῳ τὸ οὕτως εἶναι.\FnParse{e}{present active infinitive} 
\MainTextVerseMark{7}{27}δέδεσαι\FnParse{a}{perfect passive indicative 2nd singular} γυναικί; μὴ ζήτει\FnParse{b}{present active imperative 2nd singular} λύσιν·\FnParseFormGloss{c}{accusative feminine singular}{λύσις}{divorce} λέλυσαι\FnParse{d}{perfect passive indicative 2nd singular} ἀπὸ γυναικός; μὴ ζήτει\FnParse{e}{present active imperative 2nd singular} γυναῖκα· 
\MainTextVerseMark{7}{28}ἐὰν δὲ καὶ ⸀γαμήσῃς,\FnParseFormGloss{a}{aorist active subjunctive 2nd singular}{γαμέω}{to marry} οὐχ ἥμαρτες.\FnParse{b}{aorist active indicative 2nd singular} καὶ ἐὰν γήμῃ\FnParseFormGloss{c}{aorist active subjunctive 3rd singular}{γαμέω}{to marry} ἡ παρθένος,\FnParseFormGloss{d}{nominative feminine singular}{παρθένος}{virgin} οὐχ ἥμαρτεν.\FnParse{e}{aorist active indicative 3rd singular} θλῖψιν δὲ τῇ σαρκὶ ἕξουσιν\FnParse{f}{future active indicative 3rd plural} οἱ τοιοῦτοι, ἐγὼ δὲ ὑμῶν φείδομαι.\FnParseFormGloss{g}{present middle indicative 1st singular}{φείδομαι}{to spare} 
\MainTextVerseMark{7}{29}τοῦτο δέ φημι,\FnParse{a}{present active indicative 1st singular} ἀδελφοί, ὁ καιρὸς συνεσταλμένος\FnParseFormGloss{b}{perfect passive participle nominative masculine singular}{συστέλλω}{to wrap up} ⸂ἐστίν·\FnParse{c}{present active indicative 3rd singular} τὸ λοιπὸν⸃ ἵνα καὶ οἱ ἔχοντες\FnParse{d}{present active participle nominative masculine plural} γυναῖκας ὡς μὴ ἔχοντες\FnParse{e}{present active participle nominative masculine plural} ὦσιν,\FnParse{f}{present active subjunctive 3rd plural} 
\MainTextVerseMark{7}{30}καὶ οἱ κλαίοντες\FnParse{a}{present active participle nominative masculine plural} ὡς μὴ κλαίοντες,\FnParse{b}{present active participle nominative masculine plural} καὶ οἱ χαίροντες\FnParse{c}{present active participle nominative masculine plural} ὡς μὴ χαίροντες,\FnParse{d}{present active participle nominative masculine plural} καὶ οἱ ἀγοράζοντες\FnParse{e}{present active participle nominative masculine plural} ὡς μὴ κατέχοντες,\FnParseFormGloss{f}{present active participle nominative masculine plural}{κατέχω}{to hold back} 
\MainTextVerseMark{7}{31}καὶ οἱ χρώμενοι\FnParseFormGloss{a}{present middle participle nominative masculine plural}{χράομαι}{to make use of} ⸂τὸν κόσμον⸃ ὡς μὴ καταχρώμενοι·\FnParseFormGloss{b}{present middle participle nominative masculine plural}{καταχράομαι}{to make full use of} παράγει\FnParseFormGloss{c}{present active indicative 3rd singular}{παράγω}{to pass by} γὰρ τὸ σχῆμα\FnParseFormGloss{d}{nominative neuter singular}{σχῆμα}{form} τοῦ κόσμου τούτου. 
\MainTextVerseMark{7}{32}Θέλω\FnParse{a}{present active indicative 1st singular} δὲ ὑμᾶς ἀμερίμνους\FnParseFormGloss{b}{accusative masculine plural}{ἀμέριμνος}{free from concern} εἶναι.\FnParse{c}{present active infinitive} ὁ ἄγαμος\FnParseFormGloss{d}{nominative masculine singular}{ἄγαμος}{unmarried} μεριμνᾷ\FnParseFormGloss{e}{present active indicative 3rd singular}{μεριμνάω}{to worry} τὰ τοῦ κυρίου, πῶς ⸀ἀρέσῃ\FnParseFormGloss{f}{aorist active subjunctive 3rd singular}{ἀρέσκω}{to please} τῷ κυρίῳ· 
\MainTextVerseMark{7}{33}ὁ δὲ γαμήσας\FnParseFormGloss{a}{aorist active participle nominative masculine singular}{γαμέω}{to marry} μεριμνᾷ\FnParseFormGloss{b}{present active indicative 3rd singular}{μεριμνάω}{to worry} τὰ τοῦ κόσμου, πῶς ⸀ἀρέσῃ\FnParseFormGloss{c}{aorist active subjunctive 3rd singular}{ἀρέσκω}{to please} τῇ γυναικί, 
\MainTextVerseMark{7}{34}⸀καὶ μεμέρισται.\FnParseFormGloss{a}{perfect passive indicative 3rd singular}{μερίζω}{to give} καὶ ἡ γυνὴ ⸂ἡ ἄγαμος⸃\FnParseFormGloss{b}{nominative feminine singular}{ἄγαμος}{unmarried} καὶ ἡ ⸀παρθένος\FnParseFormGloss{c}{nominative feminine singular}{παρθένος}{virgin} μεριμνᾷ\FnParseFormGloss{d}{present active indicative 3rd singular}{μεριμνάω}{to worry} τὰ τοῦ κυρίου, ἵνα ᾖ\FnParse{e}{present active subjunctive 3rd singular} ἁγία καὶ ⸀τῷ σώματι καὶ ⸁τῷ πνεύματι· ἡ δὲ γαμήσασα\FnParseFormGloss{f}{aorist active participle nominative feminine singular}{γαμέω}{to marry} μεριμνᾷ\FnParseFormGloss{g}{present active indicative 3rd singular}{μεριμνάω}{to worry} τὰ τοῦ κόσμου, πῶς ⸀ἀρέσῃ\FnParseFormGloss{h}{aorist active subjunctive 3rd singular}{ἀρέσκω}{to please} τῷ ἀνδρί. 
\MainTextVerseMark{7}{35}τοῦτο δὲ πρὸς τὸ ὑμῶν αὐτῶν ⸀σύμφορον\FnParseFormGloss{a}{accusative neuter singular}{σύμφορον}{beneficial} λέγω,\FnParse{b}{present active indicative 1st singular} οὐχ ἵνα βρόχον\FnParseFormGloss{c}{accusative masculine singular}{βρόχος}{restriction} ὑμῖν ἐπιβάλω,\FnParseFormGloss{d}{aorist active subjunctive 1st singular}{ἐπιβάλλω}{to throw over} ἀλλὰ πρὸς τὸ εὔσχημον\FnParseFormGloss{e}{accusative neuter singular}{εὐσχήμων}{presentable} καὶ ⸀εὐπάρεδρον\FnParseFormGloss{f}{accusative neuter singular}{εὐπάρεδρον}{devoted} τῷ κυρίῳ ἀπερισπάστως.\FnFormGloss{g}{ἀπερισπάστως}{undivided} 
\MainTextVerseMark{7}{36}Εἰ δέ τις ἀσχημονεῖν\FnParseFormGloss{a}{present active infinitive}{ἀσχημονέω}{to act improperly} ἐπὶ τὴν παρθένον\FnParseFormGloss{b}{accusative feminine singular}{παρθένος}{virgin} αὐτοῦ νομίζει\FnParseFormGloss{c}{present active indicative 3rd singular}{νομίζω}{to think} ἐὰν ᾖ\FnParse{d}{present active subjunctive 3rd singular} ὑπέρακμος,\FnParseFormGloss{e}{nominative feminine singular}{ὑπέρακμος}{past one's prime} καὶ οὕτως ὀφείλει\FnParse{f}{present active indicative 3rd singular} γίνεσθαι,\FnParse{g}{present middle infinitive} ὃ θέλει\FnParse{h}{present active indicative 3rd singular} ποιείτω·\FnParse{i}{present active imperative 3rd singular} οὐχ ἁμαρτάνει·\FnParse{j}{present active indicative 3rd singular} γαμείτωσαν.\FnParseFormGloss{k}{present active imperative 3rd plural}{γαμέω}{to marry} 
\MainTextVerseMark{7}{37}ὃς δὲ ἕστηκεν\FnParse{a}{perfect active indicative 3rd singular} ⸂ἐν τῇ καρδίᾳ αὐτοῦ ἑδραῖος⸃\FnParseFormGloss{b}{nominative masculine singular}{ἑδραῖος}{firm} μὴ ἔχων\FnParse{c}{present active participle nominative masculine singular} ἀνάγκην,\FnParseFormGloss{d}{accusative feminine singular}{ἀνάγκη}{necessity} ἐξουσίαν δὲ ἔχει\FnParse{e}{present active indicative 3rd singular} περὶ τοῦ ἰδίου θελήματος, καὶ τοῦτο κέκρικεν\FnParse{f}{perfect active indicative 3rd singular} ἐν τῇ ⸂ἰδίᾳ καρδίᾳ⸃, ⸀τηρεῖν\FnParse{g}{present active infinitive} τὴν ἑαυτοῦ παρθένον,\FnParseFormGloss{h}{accusative feminine singular}{παρθένος}{virgin} καλῶς ⸀ποιήσει·\FnParse{i}{future active indicative 3rd singular} 
\MainTextVerseMark{7}{38}ὥστε καὶ ὁ ⸀γαμίζων\FnParseFormGloss{a}{present active participle nominative masculine singular}{γαμίζω}{to give in marriage} ⸂τὴν παρθένον\FnParseFormGloss{b}{accusative feminine singular}{παρθένος}{virgin} ἑαυτοῦ⸃ καλῶς ποιεῖ,\FnParse{c}{present active indicative 3rd singular} ⸂καὶ ὁ⸃ μὴ ⸁γαμίζων\FnParseFormGloss{d}{present active participle nominative masculine singular}{γαμίζω}{to give in marriage} κρεῖσσον\FnParseFormGloss{e}{accusative neuter singular}{κρείττων}{better} ⸀ποιήσει.\FnParse{f}{future active indicative 3rd singular} 
\MainTextVerseMark{7}{39}Γυνὴ ⸀δέδεται\FnParse{a}{perfect passive indicative 3rd singular} ἐφ’ ὅσον χρόνον ζῇ\FnParse{b}{present active indicative 3rd singular} ὁ ἀνὴρ αὐτῆς· ἐὰν ⸀δὲ κοιμηθῇ\FnParseFormGloss{c}{aorist passive subjunctive 3rd singular}{κοιμάομαι}{to fall asleep} ὁ ἀνήρ, ἐλευθέρα\FnParseFormGloss{d}{nominative feminine singular}{ἐλεύθερος}{free} ἐστὶν\FnParse{e}{present active indicative 3rd singular} ᾧ θέλει\FnParse{f}{present active indicative 3rd singular} γαμηθῆναι,\FnParseFormGloss{g}{aorist passive infinitive}{γαμέω}{to marry} μόνον ἐν κυρίῳ· 
\MainTextVerseMark{7}{40}μακαριωτέρα δέ ἐστιν\FnParse{a}{present active indicative 3rd singular} ἐὰν οὕτως μείνῃ,\FnParse{b}{aorist active subjunctive 3rd singular} κατὰ τὴν ἐμὴν γνώμην,\FnParseFormGloss{c}{accusative feminine singular}{γνώμη}{purpose} δοκῶ\FnParse{d}{present active indicative 1st singular} ⸀δὲ κἀγὼ πνεῦμα θεοῦ ἔχειν.\FnParse{e}{present active infinitive} 
\ChapterHeader{8}{Chapter 8}

\MainTextVerseMark{8}{1}Περὶ δὲ τῶν εἰδωλοθύτων,\FnParseFormGloss{a}{genitive neuter plural}{εἰδωλόθυτον}{sacrificed to idols} οἴδαμεν\FnParse{b}{perfect active indicative 1st plural} ὅτι πάντες γνῶσιν\FnParseFormGloss{c}{accusative feminine singular}{γνῶσις}{knowledge} ἔχομεν.\FnParse{d}{present active indicative 1st plural} ἡ γνῶσις\FnParseFormGloss{e}{nominative feminine singular}{γνῶσις}{knowledge} φυσιοῖ,\FnParseFormGloss{f}{present active indicative 3rd singular}{φυσιόω}{to puff up} ἡ δὲ ἀγάπη οἰκοδομεῖ.\FnParse{g}{present active indicative 3rd singular} 
\MainTextVerseMark{8}{2}⸀εἴ τις δοκεῖ\FnParse{a}{present active indicative 3rd singular} ⸀ἐγνωκέναι\FnParse{b}{perfect active infinitive} τι, ⸀οὔπω\FnFormGloss{c}{οὔπω}{not yet} ⸀ἔγνω\FnParse{d}{aorist active indicative 3rd singular} καθὼς δεῖ\FnParse{e}{present active indicative 3rd singular} γνῶναι·\FnParse{f}{aorist active infinitive} 
\MainTextVerseMark{8}{3}εἰ δέ τις ἀγαπᾷ\FnParse{a}{present active indicative 3rd singular} τὸν θεόν, οὗτος ἔγνωσται\FnParse{b}{perfect passive indicative 3rd singular} ὑπ’ αὐτοῦ. 
\MainTextVerseMark{8}{4}Περὶ τῆς βρώσεως\FnParseFormGloss{a}{genitive feminine singular}{βρῶσις}{consumable} οὖν τῶν εἰδωλοθύτων\FnParseFormGloss{b}{genitive neuter plural}{εἰδωλόθυτον}{sacrificed to idols} οἴδαμεν\FnParse{c}{perfect active indicative 1st plural} ὅτι οὐδὲν εἴδωλον\FnParseFormGloss{d}{nominative neuter singular}{εἴδωλον}{idol} ἐν κόσμῳ, καὶ ὅτι οὐδεὶς ⸀θεὸς εἰ μὴ εἷς. 
\MainTextVerseMark{8}{5}καὶ γὰρ εἴπερ\FnFormGloss{a}{εἴπερ}{if indeed} εἰσὶν\FnParse{b}{present active indicative 3rd plural} λεγόμενοι\FnParse{c}{present passive participle nominative masculine plural} θεοὶ εἴτε ἐν οὐρανῷ εἴτε ἐπὶ γῆς, ὥσπερ εἰσὶν\FnParse{d}{present active indicative 3rd plural} θεοὶ πολλοὶ καὶ κύριοι πολλοί, 
\MainTextVerseMark{8}{6}ἀλλ’ ἡμῖν εἷς θεὸς ὁ πατήρ, ἐξ οὗ τὰ πάντα καὶ ἡμεῖς εἰς αὐτόν, καὶ εἷς κύριος Ἰησοῦς Χριστός, δι’ οὗ τὰ πάντα καὶ ἡμεῖς δι’ αὐτοῦ. 
\MainTextVerseMark{8}{7}Ἀλλ’ οὐκ ἐν πᾶσιν ἡ γνῶσις·\FnParseFormGloss{a}{nominative feminine singular}{γνῶσις}{knowledge} τινὲς δὲ τῇ ⸀συνηθείᾳ\FnParseFormGloss{b}{dative feminine singular}{συνήθεια}{custom} ⸂ἕως ἄρτι τοῦ εἰδώλου⸃\FnParseFormGloss{c}{genitive neuter singular}{εἴδωλον}{idol} ὡς εἰδωλόθυτον\FnParseFormGloss{d}{accusative neuter singular}{εἰδωλόθυτον}{sacrificed to idols} ἐσθίουσιν,\FnParse{e}{present active indicative 3rd plural} καὶ ἡ συνείδησις αὐτῶν ἀσθενὴς\FnParseFormGloss{f}{nominative feminine singular}{ἀσθενής}{weak} οὖσα\FnParse{g}{present active participle nominative feminine singular} μολύνεται.\FnParseFormGloss{h}{present passive indicative 3rd singular}{μολύνω}{to defile} 
\MainTextVerseMark{8}{8}βρῶμα\FnParseFormGloss{a}{nominative neuter singular}{βρῶμα}{food} δὲ ἡμᾶς οὐ ⸀παραστήσει\FnParse{b}{future active indicative 3rd singular} τῷ θεῷ· οὔτε ⸂γὰρ ἐὰν φάγωμεν,\FnParse{c}{aorist active subjunctive 1st plural} περισσεύομεν,\FnParse{d}{present active indicative 1st plural} οὔτε ἐὰν μὴ φάγωμεν,\FnParse{e}{aorist active subjunctive 1st plural} ὑστερούμεθα⸃.\FnParseFormGloss{f}{present passive indicative 1st plural}{ὑστερέω}{to lack} 
\MainTextVerseMark{8}{9}βλέπετε\FnParse{a}{present active imperative 2nd plural} δὲ μή πως\FnFormGloss{b}{πώς}{in any way} ἡ ἐξουσία ὑμῶν αὕτη πρόσκομμα\FnParseFormGloss{c}{nominative neuter singular}{πρόσκομμα}{stumbling block} γένηται\FnParse{d}{aorist middle subjunctive 3rd singular} τοῖς ⸀ἀσθενέσιν.\FnParseFormGloss{e}{dative masculine plural}{ἀσθενής}{weak} 
\MainTextVerseMark{8}{10}ἐὰν γάρ τις ἴδῃ\FnParse{a}{aorist active subjunctive 3rd singular} σὲ τὸν ἔχοντα\FnParse{b}{present active participle accusative masculine singular} γνῶσιν\FnParseFormGloss{c}{accusative feminine singular}{γνῶσις}{knowledge} ἐν εἰδωλείῳ\FnParseFormGloss{d}{dative neuter singular}{εἰδωλεῖον}{temple of an idol} κατακείμενον,\FnParseFormGloss{e}{present middle participle accusative masculine singular}{κατάκειμαι}{to lie down} οὐχὶ ἡ συνείδησις αὐτοῦ ἀσθενοῦς\FnParseFormGloss{f}{genitive masculine singular}{ἀσθενής}{weak} ὄντος\FnParse{g}{present active participle genitive masculine singular} οἰκοδομηθήσεται\FnParse{h}{future passive indicative 3rd singular} εἰς τὸ τὰ εἰδωλόθυτα\FnParseFormGloss{i}{accusative neuter plural}{εἰδωλόθυτον}{sacrificed to idols} ἐσθίειν;\FnParse{j}{present active infinitive} 
\MainTextVerseMark{8}{11}⸂ἀπόλλυται\FnParse{a}{present passive indicative 3rd singular} γὰρ⸃ ὁ ἀσθενῶν\FnParse{b}{present active participle nominative masculine singular} ⸀ἐν τῇ σῇ\FnParseFormGloss{c}{dative feminine singular}{σός}{your} γνώσει,\FnParseFormGloss{d}{dative feminine singular}{γνῶσις}{knowledge} ⸂ὁ ἀδελφὸς⸃ δι’ ὃν Χριστὸς ἀπέθανεν.\FnParse{e}{aorist active indicative 3rd singular} 
\MainTextVerseMark{8}{12}οὕτως δὲ ἁμαρτάνοντες\FnParse{a}{present active participle nominative masculine plural} εἰς τοὺς ἀδελφοὺς καὶ τύπτοντες\FnParseFormGloss{b}{present active participle nominative masculine plural}{τύπτω}{to strike} αὐτῶν τὴν συνείδησιν ἀσθενοῦσαν\FnParse{c}{present active participle accusative feminine singular} εἰς Χριστὸν ἁμαρτάνετε.\FnParse{d}{present active indicative 2nd plural} 
\MainTextVerseMark{8}{13}διόπερ\FnFormGloss{a}{διόπερ}{therefore} εἰ βρῶμα\FnParseFormGloss{b}{nominative neuter singular}{βρῶμα}{food} σκανδαλίζει\FnParse{c}{present active indicative 3rd singular} τὸν ἀδελφόν μου, οὐ μὴ φάγω\FnParse{d}{aorist active subjunctive 1st singular} κρέα\FnParseFormGloss{e}{accusative neuter plural}{κρέας}{meat} εἰς τὸν αἰῶνα, ἵνα μὴ τὸν ἀδελφόν μου σκανδαλίσω.\FnParse{f}{aorist active subjunctive 1st singular} 
\ChapterHeader{9}{Chapter 9}

\MainTextVerseMark{9}{1}Οὐκ εἰμὶ\FnParse{a}{present active indicative 1st singular} ⸂ἐλεύθερος;\FnParseFormGloss{b}{nominative masculine singular}{ἐλεύθερος}{free} οὐκ εἰμὶ\FnParse{c}{present active indicative 1st singular} ἀπόστολος⸃; οὐχὶ ⸀Ἰησοῦν τὸν κύριον ἡμῶν ἑόρακα;\FnParse{d}{perfect active indicative 1st singular} οὐ τὸ ἔργον μου ὑμεῖς ἐστε\FnParse{e}{present active indicative 2nd plural} ἐν κυρίῳ; 
\MainTextVerseMark{9}{2}εἰ ἄλλοις οὐκ εἰμὶ\FnParse{a}{present active indicative 1st singular} ἀπόστολος, ἀλλά γε\FnFormGloss{b}{γέ}{indeed} ὑμῖν εἰμι,\FnParse{c}{present active indicative 1st singular} ἡ γὰρ σφραγίς\FnParseFormGloss{d}{nominative feminine singular}{σφραγίς}{seal} ⸂μου τῆς⸃ ἀποστολῆς\FnParseFormGloss{e}{genitive feminine singular}{ἀποστολή}{apostleship} ὑμεῖς ἐστε\FnParse{f}{present active indicative 2nd plural} ἐν κυρίῳ. 
\MainTextVerseMark{9}{3}Ἡ ἐμὴ ἀπολογία\FnParseFormGloss{a}{nominative feminine singular}{ἀπολογία}{defense} τοῖς ἐμὲ ἀνακρίνουσίν\FnParseFormGloss{b}{present active participle dative masculine plural}{ἀνακρίνω}{to examine} ⸂ἐστιν\FnParse{c}{present active indicative 3rd singular} αὕτη⸃. 
\MainTextVerseMark{9}{4}μὴ οὐκ ἔχομεν\FnParse{a}{present active indicative 1st plural} ἐξουσίαν φαγεῖν\FnParse{b}{aorist active infinitive} καὶ ⸀πεῖν;\FnParse{c}{aorist active infinitive} 
\MainTextVerseMark{9}{5}μὴ οὐκ ἔχομεν\FnParse{a}{present active indicative 1st plural} ἐξουσίαν ἀδελφὴν\FnParseFormGloss{b}{accusative feminine singular}{ἀδελφή}{sister} γυναῖκα περιάγειν,\FnParseFormGloss{c}{present active infinitive}{περιάγω}{to take} ὡς καὶ οἱ λοιποὶ ἀπόστολοι καὶ οἱ ἀδελφοὶ τοῦ κυρίου καὶ Κηφᾶς;\FnParseFormGloss{d}{nominative masculine singular}{Κηφᾶς}{Cephas} 
\MainTextVerseMark{9}{6}ἢ μόνος ἐγὼ καὶ Βαρναβᾶς\FnParseFormGloss{a}{nominative masculine singular}{Βαρναβᾶς}{Barnabas} οὐκ ἔχομεν\FnParse{b}{present active indicative 1st plural} ἐξουσίαν ⸀μὴ ἐργάζεσθαι;\FnParse{c}{present middle infinitive} 
\MainTextVerseMark{9}{7}τίς στρατεύεται\FnParseFormGloss{a}{present middle indicative 3rd singular}{στρατεύομαι}{to serve as a soldier} ἰδίοις ὀψωνίοις\FnParseFormGloss{b}{dative neuter plural}{ὀψώνιον}{pay} ποτέ;\FnFormGloss{c}{ποτέ}{once} τίς φυτεύει\FnParseFormGloss{d}{present active indicative 3rd singular}{φυτεύω}{to plant} ἀμπελῶνα\FnParseFormGloss{e}{accusative masculine singular}{ἀμπελών}{vineyard} καὶ ⸂τὸν καρπὸν⸃ αὐτοῦ οὐκ ἐσθίει;\FnParse{f}{present active indicative 3rd singular} ⸀τίς ποιμαίνει\FnParseFormGloss{g}{present active indicative 3rd singular}{ποιμαίνω}{to shepherd} ποίμνην\FnParseFormGloss{h}{accusative feminine singular}{ποίμνη}{flock} καὶ ἐκ τοῦ γάλακτος\FnParseFormGloss{i}{genitive neuter singular}{γάλα}{milk} τῆς ποίμνης\FnParseFormGloss{j}{genitive feminine singular}{ποίμνη}{flock} οὐκ ἐσθίει;\FnParse{k}{present active indicative 3rd singular} 
\MainTextVerseMark{9}{8}Μὴ κατὰ ἄνθρωπον ταῦτα λαλῶ\FnParse{a}{present active indicative 1st singular} ἢ ⸂καὶ ὁ νόμος ταῦτα οὐ⸃ λέγει;\FnParse{b}{present active indicative 3rd singular} 
\MainTextVerseMark{9}{9}ἐν γὰρ τῷ Μωϋσέως νόμῳ γέγραπται·\FnParse{a}{perfect passive indicative 3rd singular} Οὐ ⸀κημώσεις\FnParseFormGloss{b}{future active indicative 2nd singular}{κημόω}{to muzzle} βοῦν\FnParseFormGloss{c}{accusative masculine singular}{βοῦς}{ox} ἀλοῶντα.\FnParseFormGloss{d}{present active participle accusative masculine singular}{ἀλοάω}{to tread} μὴ τῶν βοῶν\FnParseFormGloss{e}{genitive masculine plural}{βοῦς}{ox} μέλει\FnParseFormGloss{f}{present active indicative 3rd singular}{μέλει}{it is a care} τῷ θεῷ, 
\MainTextVerseMark{9}{10}ἢ δι’ ἡμᾶς πάντως\FnFormGloss{a}{πάντως}{surely} λέγει;\FnParse{b}{present active indicative 3rd singular} δι’ ἡμᾶς γὰρ ἐγράφη,\FnParse{c}{aorist passive indicative 3rd singular} ὅτι ⸂ὀφείλει\FnParse{d}{present active indicative 3rd singular} ἐπ’ ἐλπίδι⸃ ὁ ἀροτριῶν\FnParseFormGloss{e}{present active participle nominative masculine singular}{ἀροτριάω}{to plow} ἀροτριᾶν,\FnParseFormGloss{f}{present active infinitive}{ἀροτριάω}{to plow} καὶ ὁ ἀλοῶν\FnParseFormGloss{g}{present active participle nominative masculine singular}{ἀλοάω}{to tread} ⸂ἐπ’ ἐλπίδι τοῦ μετέχειν⸃.\FnParseFormGloss{h}{present active infinitive}{μετέχω}{to share in} 
\MainTextVerseMark{9}{11}εἰ ἡμεῖς ὑμῖν τὰ πνευματικὰ\FnParseFormGloss{a}{accusative neuter plural}{πνευματικός}{spiritual} ἐσπείραμεν,\FnParse{b}{aorist active indicative 1st plural} μέγα εἰ ἡμεῖς ὑμῶν τὰ σαρκικὰ\FnParseFormGloss{c}{accusative neuter plural}{σαρκικός}{material} θερίσομεν;\FnParseFormGloss{d}{future active indicative 1st plural}{θερίζω}{to reap} 
\MainTextVerseMark{9}{12}εἰ ἄλλοι τῆς ⸂ὑμῶν ἐξουσίας⸃ μετέχουσιν,\FnParseFormGloss{a}{present active indicative 3rd plural}{μετέχω}{to share in} οὐ μᾶλλον ἡμεῖς; Ἀλλ’ οὐκ ἐχρησάμεθα\FnParseFormGloss{b}{aorist middle indicative 1st plural}{χράομαι}{to make use of} τῇ ἐξουσίᾳ ταύτῃ, ἀλλὰ πάντα στέγομεν\FnParseFormGloss{c}{present active indicative 1st plural}{στέγω}{to put up with} ἵνα μή ⸂τινα ἐγκοπὴν⸃\FnParseFormGloss{d}{accusative feminine singular}{ἐγκοπή}{hinderance} δῶμεν\FnParse{e}{aorist active subjunctive 1st plural} τῷ εὐαγγελίῳ τοῦ Χριστοῦ. 
\MainTextVerseMark{9}{13}οὐκ οἴδατε\FnParse{a}{perfect active indicative 2nd plural} ὅτι οἱ τὰ ἱερὰ ἐργαζόμενοι\FnParse{b}{present middle participle nominative masculine plural} ⸀τὰ ἐκ τοῦ ἱεροῦ ἐσθίουσιν,\FnParse{c}{present active indicative 3rd plural} οἱ τῷ θυσιαστηρίῳ\FnParseFormGloss{d}{dative neuter singular}{θυσιαστήριον}{altar} ⸀παρεδρεύοντες\FnParseFormGloss{e}{present active participle nominative masculine plural}{παρεδρεύω}{to serve regularly} τῷ θυσιαστηρίῳ\FnParseFormGloss{f}{dative neuter singular}{θυσιαστήριον}{altar} συμμερίζονται;\FnParseFormGloss{g}{present middle indicative 3rd plural}{συμμερίζομαι}{to share with} 
\MainTextVerseMark{9}{14}οὕτως καὶ ὁ κύριος διέταξεν\FnParseFormGloss{a}{aorist active indicative 3rd singular}{διατάσσω}{to command} τοῖς τὸ εὐαγγέλιον καταγγέλλουσιν\FnParseFormGloss{b}{present active participle dative masculine plural}{καταγγέλλω}{to preach} ἐκ τοῦ εὐαγγελίου ζῆν.\FnParse{c}{present active infinitive} 
\MainTextVerseMark{9}{15}Ἐγὼ δὲ ⸂οὐ κέχρημαι\FnParseFormGloss{a}{perfect middle indicative 1st singular}{χράομαι}{to make use of} οὐδενὶ⸃ τούτων. οὐκ ἔγραψα\FnParse{b}{aorist active indicative 1st singular} δὲ ταῦτα ἵνα οὕτως γένηται\FnParse{c}{aorist middle subjunctive 3rd singular} ἐν ἐμοί, καλὸν γάρ μοι μᾶλλον ἀποθανεῖν\FnParse{d}{aorist active infinitive} ἤ— τὸ καύχημά\FnParseFormGloss{e}{accusative neuter singular}{καύχημα}{something to boast about} μου ⸂οὐδεὶς κενώσει⸃.\FnParseFormGloss{f}{future active indicative 3rd singular}{κενόω}{to empty} 
\MainTextVerseMark{9}{16}ἐὰν γὰρ εὐαγγελίζωμαι,\FnParse{a}{present middle subjunctive 1st singular} οὐκ ἔστιν\FnParse{b}{present active indicative 3rd singular} μοι καύχημα,\FnParseFormGloss{c}{nominative neuter singular}{καύχημα}{something to boast about} ἀνάγκη\FnParseFormGloss{d}{nominative feminine singular}{ἀνάγκη}{necessity} γάρ μοι ἐπίκειται·\FnParseFormGloss{e}{present middle indicative 3rd singular}{ἐπίκειμαι}{to lay upon} οὐαὶ ⸀γάρ μοί ἐστιν\FnParse{f}{present active indicative 3rd singular} ἐὰν μὴ ⸀εὐαγγελίσωμαι.\FnParse{g}{aorist middle subjunctive 1st singular} 
\MainTextVerseMark{9}{17}εἰ γὰρ ἑκὼν\FnParseFormGloss{a}{nominative masculine singular}{ἑκών}{voluntarily} τοῦτο πράσσω,\FnParse{b}{present active indicative 1st singular} μισθὸν\FnParseFormGloss{c}{accusative masculine singular}{μισθός}{wage} ἔχω·\FnParse{d}{present active indicative 1st singular} εἰ δὲ ἄκων,\FnParseFormGloss{e}{nominative masculine singular}{ἄκων}{not voluntary} οἰκονομίαν\FnParseFormGloss{f}{accusative feminine singular}{οἰκονομία}{management} πεπίστευμαι.\FnParse{g}{perfect passive indicative 1st singular} 
\MainTextVerseMark{9}{18}τίς οὖν ⸀μού ἐστιν\FnParse{a}{present active indicative 3rd singular} ὁ μισθός;\FnParseFormGloss{b}{nominative masculine singular}{μισθός}{wage} ἵνα εὐαγγελιζόμενος\FnParse{c}{present middle participle nominative masculine singular} ἀδάπανον\FnParseFormGloss{d}{accusative neuter singular}{ἀδάπανος}{free of charge} θήσω\FnParse{e}{aorist active subjunctive 1st singular} τὸ ⸀εὐαγγέλιον, εἰς τὸ μὴ καταχρήσασθαι\FnParseFormGloss{f}{aorist middle infinitive}{καταχράομαι}{to make full use of} τῇ ἐξουσίᾳ μου ἐν τῷ εὐαγγελίῳ. 
\MainTextVerseMark{9}{19}Ἐλεύθερος\FnParseFormGloss{a}{nominative masculine singular}{ἐλεύθερος}{free} γὰρ ὢν\FnParse{b}{present active participle nominative masculine singular} ἐκ πάντων πᾶσιν ἐμαυτὸν ἐδούλωσα,\FnParseFormGloss{c}{aorist active indicative 1st singular}{δουλόω}{to enslave} ἵνα τοὺς πλείονας κερδήσω·\FnParseFormGloss{d}{aorist active subjunctive 1st singular}{κερδαίνω}{to gain} 
\MainTextVerseMark{9}{20}καὶ ἐγενόμην\FnParse{a}{aorist middle indicative 1st singular} τοῖς Ἰουδαίοις ὡς Ἰουδαῖος, ἵνα Ἰουδαίους κερδήσω·\FnParseFormGloss{b}{aorist active subjunctive 1st singular}{κερδαίνω}{to gain} τοῖς ὑπὸ νόμον ὡς ὑπὸ νόμον, ⸂μὴ ὢν\FnParse{c}{present active participle nominative masculine singular} αὐτὸς ὑπὸ νόμον⸃, ἵνα τοὺς ὑπὸ νόμον κερδήσω·\FnParseFormGloss{d}{aorist active subjunctive 1st singular}{κερδαίνω}{to gain} 
\MainTextVerseMark{9}{21}τοῖς ἀνόμοις\FnParseFormGloss{a}{dative masculine plural}{ἄνομος}{without law} ὡς ἄνομος,\FnParseFormGloss{b}{nominative masculine singular}{ἄνομος}{without law} μὴ ὢν\FnParse{c}{present active participle nominative masculine singular} ἄνομος\FnParseFormGloss{d}{nominative masculine singular}{ἄνομος}{without law} ⸀θεοῦ ἀλλ’ ἔννομος\FnParseFormGloss{e}{nominative masculine singular}{ἔννομος}{under law} ⸀Χριστοῦ, ἵνα ⸀κερδάνω\FnParseFormGloss{f}{aorist active subjunctive 1st singular}{κερδαίνω}{to gain} ⸀τοὺς ἀνόμους·\FnParseFormGloss{g}{accusative masculine plural}{ἄνομος}{without law} 
\MainTextVerseMark{9}{22}ἐγενόμην\FnParse{a}{aorist middle indicative 1st singular} τοῖς ⸀ἀσθενέσιν\FnParseFormGloss{b}{dative masculine plural}{ἀσθενής}{weak} ἀσθενής,\FnParseFormGloss{c}{nominative masculine singular}{ἀσθενής}{weak} ἵνα τοὺς ἀσθενεῖς\FnParseFormGloss{d}{accusative masculine plural}{ἀσθενής}{weak} κερδήσω·\FnParseFormGloss{e}{aorist active subjunctive 1st singular}{κερδαίνω}{to gain} τοῖς πᾶσιν ⸀γέγονα\FnParse{f}{perfect active indicative 1st singular} πάντα, ἵνα πάντως\FnFormGloss{g}{πάντως}{surely} τινὰς σώσω.\FnParse{h}{aorist active subjunctive 1st singular} 
\MainTextVerseMark{9}{23}⸀πάντα δὲ ποιῶ\FnParse{a}{present active indicative 1st singular} διὰ τὸ εὐαγγέλιον, ἵνα συγκοινωνὸς\FnParseFormGloss{b}{nominative masculine singular}{συγκοινωνός}{sharer} αὐτοῦ γένωμαι.\FnParse{c}{aorist middle subjunctive 1st singular} 
\MainTextVerseMark{9}{24}Οὐκ οἴδατε\FnParse{a}{perfect active indicative 2nd plural} ὅτι οἱ ἐν σταδίῳ\FnParseFormGloss{b}{dative neuter singular}{στάδιον}{arena} τρέχοντες\FnParseFormGloss{c}{present active participle nominative masculine plural}{τρέχω}{to run} πάντες μὲν τρέχουσιν,\FnParseFormGloss{d}{present active indicative 3rd plural}{τρέχω}{to run} εἷς δὲ λαμβάνει\FnParse{e}{present active indicative 3rd singular} τὸ βραβεῖον;\FnParseFormGloss{f}{accusative neuter singular}{βραβεῖον}{prize} οὕτως τρέχετε\FnParseFormGloss{g}{present active imperative 2nd plural}{τρέχω}{to run} ἵνα καταλάβητε.\FnParseFormGloss{h}{aorist active subjunctive 2nd plural}{καταλαμβάνω}{to obtain} 
\MainTextVerseMark{9}{25}πᾶς δὲ ὁ ἀγωνιζόμενος\FnParseFormGloss{a}{present middle participle nominative masculine singular}{ἀγωνίζομαι}{to fight} πάντα ἐγκρατεύεται,\FnParseFormGloss{b}{present middle indicative 3rd singular}{ἐγκρατεύομαι}{to have control} ἐκεῖνοι μὲν οὖν ἵνα φθαρτὸν\FnParseFormGloss{c}{accusative masculine singular}{φθαρτός}{perishable} στέφανον\FnParseFormGloss{d}{accusative masculine singular}{στέφανος}{a crown} λάβωσιν,\FnParse{e}{aorist active subjunctive 3rd plural} ἡμεῖς δὲ ἄφθαρτον.\FnParseFormGloss{f}{accusative masculine singular}{ἄφθαρτος}{imperishable} 
\MainTextVerseMark{9}{26}ἐγὼ τοίνυν\FnFormGloss{a}{τοίνυν}{then} οὕτως τρέχω\FnParseFormGloss{b}{present active indicative 1st singular}{τρέχω}{to run} ὡς οὐκ ἀδήλως,\FnFormGloss{c}{ἀδήλως}{not manifestly} οὕτως πυκτεύω\FnParseFormGloss{d}{present active indicative 1st singular}{πυκτεύω}{to fight with the fist} ὡς οὐκ ἀέρα\FnParseFormGloss{e}{accusative masculine singular}{ἀήρ}{air} δέρων·\FnParseFormGloss{f}{present active participle nominative masculine singular}{δέρω}{to beat up} 
\MainTextVerseMark{9}{27}ἀλλὰ ὑπωπιάζω\FnParseFormGloss{a}{present active indicative 1st singular}{ὑπωπιάζω}{to weary} μου τὸ σῶμα καὶ δουλαγωγῶ,\FnParseFormGloss{b}{present active indicative 1st singular}{δουλαγωγέω}{to enslave} μή πως\FnFormGloss{c}{πώς}{in any way} ἄλλοις κηρύξας\FnParse{d}{aorist active participle nominative masculine singular} αὐτὸς ἀδόκιμος\FnParseFormGloss{e}{nominative masculine singular}{ἀδόκιμος}{failing the test} γένωμαι.\FnParse{f}{aorist middle subjunctive 1st singular} 
\ChapterHeader{10}{Chapter 10}

\MainTextVerseMark{10}{1}Οὐ θέλω\FnParse{a}{present active indicative 1st singular} ⸀γὰρ ὑμᾶς ἀγνοεῖν,\FnParseFormGloss{b}{present active infinitive}{ἀγνοέω}{to be ignorant} ἀδελφοί, ὅτι οἱ πατέρες ἡμῶν πάντες ὑπὸ τὴν νεφέλην\FnParseFormGloss{c}{accusative feminine singular}{νεφέλη}{cloud} ἦσαν\FnParse{d}{imperfect active indicative 3rd plural} καὶ πάντες διὰ τῆς θαλάσσης διῆλθον,\FnParse{e}{aorist active indicative 3rd plural} 
\MainTextVerseMark{10}{2}καὶ πάντες εἰς τὸν Μωϋσῆν ⸀ἐβαπτίσαντο\FnParse{a}{aorist middle indicative 3rd plural} ἐν τῇ νεφέλῃ\FnParseFormGloss{b}{dative feminine singular}{νεφέλη}{cloud} καὶ ἐν τῇ θαλάσσῃ, 
\MainTextVerseMark{10}{3}καὶ πάντες τὸ αὐτὸ ⸂πνευματικὸν\FnParseFormGloss{a}{accusative neuter singular}{πνευματικός}{spiritual} βρῶμα⸃\FnParseFormGloss{b}{accusative neuter singular}{βρῶμα}{food} ἔφαγον\FnParse{c}{aorist active indicative 3rd plural} 
\MainTextVerseMark{10}{4}καὶ πάντες τὸ αὐτὸ ⸂πνευματικὸν\FnParseFormGloss{a}{accusative neuter singular}{πνευματικός}{spiritual} ἔπιον\FnParse{b}{aorist active indicative 3rd plural} πόμα⸃,\FnParseFormGloss{c}{accusative neuter singular}{πόμα}{drink} ἔπινον\FnParse{d}{imperfect active indicative 3rd plural} γὰρ ἐκ πνευματικῆς\FnParseFormGloss{e}{genitive feminine singular}{πνευματικός}{spiritual} ἀκολουθούσης\FnParse{f}{present active participle genitive feminine singular} πέτρας,\FnParseFormGloss{g}{genitive feminine singular}{πέτρα}{rock} ἡ ⸂πέτρα\FnParseFormGloss{h}{nominative feminine singular}{πέτρα}{rock} δὲ⸃ ἦν\FnParse{i}{imperfect active indicative 3rd singular} ὁ Χριστός· 
\MainTextVerseMark{10}{5}ἀλλ’ οὐκ ἐν τοῖς πλείοσιν αὐτῶν ηὐδόκησεν\FnParseFormGloss{a}{aorist active indicative 3rd singular}{εὐδοκέω}{to be well pleased} ὁ θεός, κατεστρώθησαν\FnParseFormGloss{b}{aorist passive indicative 3rd plural}{καταστρώννυμι}{to be scattered} γὰρ ἐν τῇ ἐρήμῳ. 
\MainTextVerseMark{10}{6}Ταῦτα δὲ τύποι\FnParseFormGloss{a}{nominative masculine plural}{τύπος}{pattern} ἡμῶν ἐγενήθησαν,\FnParse{b}{aorist passive indicative 3rd plural} εἰς τὸ μὴ εἶναι\FnParse{c}{present active infinitive} ἡμᾶς ἐπιθυμητὰς\FnParseFormGloss{d}{accusative masculine plural}{ἐπιθυμητής}{one who desires} κακῶν, καθὼς κἀκεῖνοι\FnParseFormGloss{e}{nominative masculine plural}{κἀκεῖνος}{and that one} ἐπεθύμησαν.\FnParseFormGloss{f}{aorist active indicative 3rd plural}{ἐπιθυμέω}{to long for} 
\MainTextVerseMark{10}{7}μηδὲ εἰδωλολάτραι\FnParseFormGloss{a}{nominative masculine plural}{εἰδωλολάτρης}{idolater} γίνεσθε,\FnParse{b}{present middle imperative 2nd plural} καθώς τινες αὐτῶν· ὥσπερ γέγραπται·\FnParse{c}{perfect passive indicative 3rd singular} Ἐκάθισεν\FnParse{d}{aorist active indicative 3rd singular} ὁ λαὸς φαγεῖν\FnParse{e}{aorist active infinitive} καὶ πεῖν,\FnParse{f}{aorist active infinitive} καὶ ἀνέστησαν\FnParse{g}{aorist active indicative 3rd plural} παίζειν.\FnParseFormGloss{h}{present active infinitive}{παίζω}{to indulge in revelry} 
\MainTextVerseMark{10}{8}μηδὲ πορνεύωμεν,\FnParseFormGloss{a}{present active subjunctive 1st plural}{πορνεύω}{to commit fornication} καθώς τινες αὐτῶν ἐπόρνευσαν,\FnParseFormGloss{b}{aorist active indicative 3rd plural}{πορνεύω}{to commit fornication} καὶ ἔπεσαν\FnParse{c}{aorist active indicative 3rd plural} μιᾷ ἡμέρᾳ εἴκοσι\FnParseFormGloss{d}{nominative feminine plural}{εἴκοσι(ν)}{twenty} τρεῖς χιλιάδες.\FnParseFormGloss{e}{nominative feminine plural}{χιλιάς}{thousand} 
\MainTextVerseMark{10}{9}μηδὲ ἐκπειράζωμεν\FnParseFormGloss{a}{present active subjunctive 1st plural}{ἐκπειράζω}{to test} τὸν ⸀Χριστόν, ⸀καθώς τινες αὐτῶν ἐπείρασαν,\FnParse{b}{aorist active indicative 3rd plural} καὶ ὑπὸ τῶν ὄφεων\FnParseFormGloss{c}{genitive masculine plural}{ὄφις}{snake} ⸀ἀπώλλυντο.\FnParse{d}{imperfect passive indicative 3rd plural} 
\MainTextVerseMark{10}{10}μηδὲ γογγύζετε,\FnParseFormGloss{a}{present active imperative 2nd plural}{γογγύζω}{to grumble} ⸀καθάπερ\FnFormGloss{b}{καθάπερ}{as} τινὲς αὐτῶν ἐγόγγυσαν,\FnParseFormGloss{c}{aorist active indicative 3rd plural}{γογγύζω}{to grumble} καὶ ἀπώλοντο\FnParse{d}{aorist middle indicative 3rd plural} ὑπὸ τοῦ ὀλοθρευτοῦ.\FnParseFormGloss{e}{genitive masculine singular}{ὀλοθρευτής}{destroyer} 
\MainTextVerseMark{10}{11}ταῦτα ⸀δὲ ⸂τυπικῶς\FnFormGloss{a}{τυπικῶς}{as an example} συνέβαινεν⸃\FnParseFormGloss{b}{imperfect active indicative 3rd singular}{συμβαίνω}{to happen} ἐκείνοις, ἐγράφη\FnParse{c}{aorist passive indicative 3rd singular} δὲ πρὸς νουθεσίαν\FnParseFormGloss{d}{accusative feminine singular}{νουθεσία}{warning} ἡμῶν, εἰς οὓς τὰ τέλη τῶν αἰώνων ⸀κατήντηκεν.\FnParseFormGloss{e}{perfect active indicative 3rd singular}{καταντάω}{to come to} 
\MainTextVerseMark{10}{12}ὥστε ὁ δοκῶν\FnParse{a}{present active participle nominative masculine singular} ἑστάναι\FnParse{b}{perfect active infinitive} βλεπέτω\FnParse{c}{present active imperative 3rd singular} μὴ πέσῃ,\FnParse{d}{aorist active subjunctive 3rd singular} 
\MainTextVerseMark{10}{13}πειρασμὸς\FnParseFormGloss{a}{nominative masculine singular}{πειρασμός}{test} ὑμᾶς οὐκ εἴληφεν\FnParse{b}{perfect active indicative 3rd singular} εἰ μὴ ἀνθρώπινος·\FnParseFormGloss{c}{nominative masculine singular}{ἀνθρώπινος}{human} πιστὸς δὲ ὁ θεός, ὃς οὐκ ἐάσει\FnParseFormGloss{d}{future active indicative 3rd singular}{ἐάω}{to let} ὑμᾶς πειρασθῆναι\FnParse{e}{aorist passive infinitive} ὑπὲρ ὃ δύνασθε,\FnParse{f}{present middle indicative 2nd plural} ἀλλὰ ποιήσει\FnParse{g}{future active indicative 3rd singular} σὺν τῷ πειρασμῷ\FnParseFormGloss{h}{dative masculine singular}{πειρασμός}{test} καὶ τὴν ἔκβασιν\FnParseFormGloss{i}{accusative feminine singular}{ἔκβασις}{way out} τοῦ ⸀δύνασθαι\FnParse{j}{present middle infinitive} ὑπενεγκεῖν.\FnParseFormGloss{k}{aorist active infinitive}{ὑποφέρω}{to endure} 
\MainTextVerseMark{10}{14}Διόπερ,\FnFormGloss{a}{διόπερ}{therefore} ἀγαπητοί μου, φεύγετε\FnParseFormGloss{b}{present active imperative 2nd plural}{φεύγω}{flee} ἀπὸ τῆς εἰδωλολατρίας.\FnParseFormGloss{c}{genitive feminine singular}{εἰδωλολατρία}{idolatry} 
\MainTextVerseMark{10}{15}ὡς φρονίμοις\FnParseFormGloss{a}{dative masculine plural}{φρόνιμος}{wise} λέγω·\FnParse{b}{present active indicative 1st singular} κρίνατε\FnParse{c}{aorist active imperative 2nd plural} ὑμεῖς ὅ φημι.\FnParse{d}{present active indicative 1st singular} 
\MainTextVerseMark{10}{16}τὸ ποτήριον τῆς εὐλογίας\FnParseFormGloss{a}{genitive feminine singular}{εὐλογία}{blessing} ὃ εὐλογοῦμεν,\FnParse{b}{present active indicative 1st plural} οὐχὶ κοινωνία\FnParseFormGloss{c}{nominative feminine singular}{κοινωνία}{fellowship} ⸂ἐστὶν\FnParse{d}{present active indicative 3rd singular} τοῦ αἵματος τοῦ Χριστοῦ⸃; τὸν ἄρτον ὃν κλῶμεν,\FnParseFormGloss{e}{present active indicative 1st plural}{κλάω}{to break} οὐχὶ κοινωνία\FnParseFormGloss{f}{nominative feminine singular}{κοινωνία}{fellowship} τοῦ σώματος τοῦ Χριστοῦ ἐστιν;\FnParse{g}{present active indicative 3rd singular} 
\MainTextVerseMark{10}{17}ὅτι εἷς ἄρτος, ἓν σῶμα οἱ πολλοί ἐσμεν,\FnParse{a}{present active indicative 1st plural} οἱ γὰρ πάντες ἐκ τοῦ ἑνὸς ἄρτου μετέχομεν.\FnParseFormGloss{b}{present active indicative 1st plural}{μετέχω}{to share in} 
\MainTextVerseMark{10}{18}βλέπετε\FnParse{a}{present active imperative 2nd plural} τὸν Ἰσραὴλ κατὰ σάρκα· ⸀οὐχ οἱ ἐσθίοντες\FnParse{b}{present active participle nominative masculine plural} τὰς θυσίας\FnParseFormGloss{c}{accusative feminine plural}{θυσία}{sacrifice} κοινωνοὶ\FnParseFormGloss{d}{nominative masculine plural}{κοινωνός}{partner} τοῦ θυσιαστηρίου\FnParseFormGloss{e}{genitive neuter singular}{θυσιαστήριον}{altar} εἰσίν;\FnParse{f}{present active indicative 3rd plural} 
\MainTextVerseMark{10}{19}τί οὖν φημι;\FnParse{a}{present active indicative 1st singular} ὅτι ⸂εἰδωλόθυτόν\FnParseFormGloss{b}{nominative neuter singular}{εἰδωλόθυτον}{sacrificed to idols} τί ἐστιν,\FnParse{c}{present active indicative 3rd singular} ἢ ὅτι εἴδωλόν⸃\FnParseFormGloss{d}{nominative neuter singular}{εἴδωλον}{idol} τί ἐστιν;\FnParse{e}{present active indicative 3rd singular} 
\MainTextVerseMark{10}{20}ἀλλ’ ὅτι ἃ ⸀θύουσιν,\FnParseFormGloss{a}{present active indicative 3rd plural}{θύω}{to kill} δαιμονίοις ⸂καὶ οὐ θεῷ θύουσιν⸃,\FnParseFormGloss{b}{present active indicative 3rd plural}{θύω}{to kill} οὐ θέλω\FnParse{c}{present active indicative 1st singular} δὲ ὑμᾶς κοινωνοὺς\FnParseFormGloss{d}{accusative masculine plural}{κοινωνός}{partner} τῶν δαιμονίων γίνεσθαι.\FnParse{e}{present middle infinitive} 
\MainTextVerseMark{10}{21}οὐ δύνασθε\FnParse{a}{present middle indicative 2nd plural} ποτήριον κυρίου πίνειν\FnParse{b}{present active infinitive} καὶ ποτήριον δαιμονίων· οὐ δύνασθε\FnParse{c}{present middle indicative 2nd plural} τραπέζης\FnParseFormGloss{d}{genitive feminine singular}{τράπεζα}{table} κυρίου μετέχειν\FnParseFormGloss{e}{present active infinitive}{μετέχω}{to share in} καὶ τραπέζης\FnParseFormGloss{f}{genitive feminine singular}{τράπεζα}{table} δαιμονίων. 
\MainTextVerseMark{10}{22}ἢ παραζηλοῦμεν\FnParseFormGloss{a}{present active indicative 1st plural}{παραζηλόω}{to make envious} τὸν κύριον; μὴ ἰσχυρότεροι\FnParseFormGloss{b}{nominative masculine plural}{ἰσχυρός}{powerful} αὐτοῦ ἐσμεν;\FnParse{c}{present active indicative 1st plural} 
\MainTextVerseMark{10}{23}⸀Πάντα ἔξεστιν·\FnParse{a}{present active indicative 3rd singular} ἀλλ’ οὐ πάντα συμφέρει.\FnParseFormGloss{b}{present active indicative 3rd singular}{συμφέρω}{to bring together} ⸁πάντα ἔξεστιν·\FnParse{c}{present active indicative 3rd singular} ἀλλ’ οὐ πάντα οἰκοδομεῖ.\FnParse{d}{present active indicative 3rd singular} 
\MainTextVerseMark{10}{24}μηδεὶς τὸ ἑαυτοῦ ζητείτω\FnParse{a}{present active imperative 3rd singular} ἀλλὰ τὸ τοῦ ⸀ἑτέρου. 
\MainTextVerseMark{10}{25}πᾶν τὸ ἐν μακέλλῳ\FnParseFormGloss{a}{dative neuter singular}{μάκελλον}{meat market} πωλούμενον\FnParseFormGloss{b}{present passive participle accusative neuter singular}{πωλέω}{to sell} ἐσθίετε\FnParse{c}{present active imperative 2nd plural} μηδὲν ἀνακρίνοντες\FnParseFormGloss{d}{present active participle nominative masculine plural}{ἀνακρίνω}{to examine} διὰ τὴν συνείδησιν, 
\MainTextVerseMark{10}{26}τοῦ ⸂κυρίου γὰρ⸃ ἡ γῆ καὶ τὸ πλήρωμα\FnParseFormGloss{a}{nominative neuter singular}{πλήρωμα}{fullness} αὐτῆς. 
\MainTextVerseMark{10}{27}⸀εἴ τις καλεῖ\FnParse{a}{present active indicative 3rd singular} ὑμᾶς τῶν ἀπίστων\FnParseFormGloss{b}{genitive masculine plural}{ἄπιστος}{unbelieving} καὶ θέλετε\FnParse{c}{present active indicative 2nd plural} πορεύεσθαι,\FnParse{d}{present middle infinitive} πᾶν τὸ παρατιθέμενον\FnParseFormGloss{e}{present passive participle accusative neuter singular}{παρατίθημι}{to set before} ὑμῖν ἐσθίετε\FnParse{f}{present active imperative 2nd plural} μηδὲν ἀνακρίνοντες\FnParseFormGloss{g}{present active participle nominative masculine plural}{ἀνακρίνω}{to examine} διὰ τὴν συνείδησιν· 
\MainTextVerseMark{10}{28}ἐὰν δέ τις ὑμῖν εἴπῃ·\FnParse{a}{aorist active subjunctive 3rd singular} Τοῦτο ⸀ἱερόθυτόν\FnParseFormGloss{b}{nominative neuter singular}{ἱερόθυτος}{sacrificed to pagan gods} ἐστιν,\FnParse{c}{present active indicative 3rd singular} μὴ ἐσθίετε\FnParse{d}{present active imperative 2nd plural} δι’ ἐκεῖνον τὸν μηνύσαντα\FnParseFormGloss{e}{aorist active participle accusative masculine singular}{μηνύω}{to inform} καὶ τὴν ⸀συνείδησιν· 
\MainTextVerseMark{10}{29}συνείδησιν δὲ λέγω\FnParse{a}{present active indicative 1st singular} οὐχὶ τὴν ἑαυτοῦ ἀλλὰ τὴν τοῦ ἑτέρου· ἱνατί\FnFormGloss{b}{ἱνατί}{why?} γὰρ ἡ ἐλευθερία\FnParseFormGloss{c}{nominative feminine singular}{ἐλευθερία}{freedom} μου κρίνεται\FnParse{d}{present passive indicative 3rd singular} ὑπὸ ἄλλης συνειδήσεως; 
\MainTextVerseMark{10}{30}εἰ ἐγὼ χάριτι μετέχω,\FnParseFormGloss{a}{present active indicative 1st singular}{μετέχω}{to share in} τί βλασφημοῦμαι\FnParse{b}{present passive indicative 1st singular} ὑπὲρ οὗ ἐγὼ εὐχαριστῶ;\FnParse{c}{present active indicative 1st singular} 
\MainTextVerseMark{10}{31}Εἴτε οὖν ἐσθίετε\FnParse{a}{present active indicative 2nd plural} εἴτε πίνετε\FnParse{b}{present active indicative 2nd plural} εἴτε τι ποιεῖτε,\FnParse{c}{present active indicative 2nd plural} πάντα εἰς δόξαν θεοῦ ποιεῖτε.\FnParse{d}{present active imperative 2nd plural} 
\MainTextVerseMark{10}{32}ἀπρόσκοποι\FnParseFormGloss{a}{nominative masculine plural}{ἀπρόσκοπος}{blameless} ⸂καὶ Ἰουδαίοις γίνεσθε⸃\FnParse{b}{present middle imperative 2nd plural} καὶ Ἕλλησιν\FnParseFormGloss{c}{dative masculine plural}{Ἕλλην}{Greek} καὶ τῇ ἐκκλησίᾳ τοῦ θεοῦ, 
\MainTextVerseMark{10}{33}καθὼς κἀγὼ πάντα πᾶσιν ἀρέσκω,\FnParseFormGloss{a}{present active indicative 1st singular}{ἀρέσκω}{to please} μὴ ζητῶν\FnParse{b}{present active participle nominative masculine singular} τὸ ἐμαυτοῦ ⸀σύμφορον\FnParseFormGloss{c}{accusative neuter singular}{σύμφορον}{beneficial} ἀλλὰ τὸ τῶν πολλῶν, ἵνα σωθῶσιν.\FnParse{d}{aorist passive subjunctive 3rd plural} 
\ChapterHeader{11}{Chapter 11}

\MainTextVerseMark{11}{1}μιμηταί\FnParseFormGloss{a}{nominative masculine plural}{μιμητής}{imitator} μου γίνεσθε,\FnParse{b}{present middle imperative 2nd plural} καθὼς κἀγὼ Χριστοῦ. 
\MainTextVerseMark{11}{2}Ἐπαινῶ\FnParseFormGloss{a}{present active indicative 1st singular}{ἐπαινέω}{to praise} δὲ ⸀ὑμᾶς ὅτι πάντα μου μέμνησθε\FnParseFormGloss{b}{perfect middle indicative 2nd plural}{μιμνῄσκομαι}{to remember} καὶ καθὼς παρέδωκα\FnParse{c}{aorist active indicative 1st singular} ὑμῖν τὰς παραδόσεις\FnParseFormGloss{d}{accusative feminine plural}{παράδοσις}{tradition} κατέχετε.\FnParseFormGloss{e}{present active indicative 2nd plural}{κατέχω}{to hold back} 
\MainTextVerseMark{11}{3}θέλω\FnParse{a}{present active indicative 1st singular} δὲ ὑμᾶς εἰδέναι\FnParse{b}{perfect active infinitive} ὅτι παντὸς ἀνδρὸς ἡ κεφαλὴ ὁ Χριστός ἐστιν,\FnParse{c}{present active indicative 3rd singular} κεφαλὴ δὲ γυναικὸς ὁ ἀνήρ, κεφαλὴ δὲ ⸀τοῦ Χριστοῦ ὁ θεός. 
\MainTextVerseMark{11}{4}πᾶς ἀνὴρ προσευχόμενος\FnParse{a}{present middle participle nominative masculine singular} ἢ προφητεύων\FnParseFormGloss{b}{present active participle nominative masculine singular}{προφητεύω}{to prophesy} κατὰ κεφαλῆς ἔχων\FnParse{c}{present active participle nominative masculine singular} καταισχύνει\FnParseFormGloss{d}{present active indicative 3rd singular}{καταισχύνω}{to dishonor} τὴν κεφαλὴν αὐτοῦ· 
\MainTextVerseMark{11}{5}πᾶσα δὲ γυνὴ προσευχομένη\FnParse{a}{present middle participle nominative feminine singular} ἢ προφητεύουσα\FnParseFormGloss{b}{present active participle nominative feminine singular}{προφητεύω}{to prophesy} ἀκατακαλύπτῳ\FnParseFormGloss{c}{dative feminine singular}{ἀκατακάλυπτος}{uncovered} τῇ κεφαλῇ καταισχύνει\FnParseFormGloss{d}{present active indicative 3rd singular}{καταισχύνω}{to dishonor} τὴν κεφαλὴν ⸀αὐτῆς, ἓν γάρ ἐστιν\FnParse{e}{present active indicative 3rd singular} καὶ τὸ αὐτὸ τῇ ἐξυρημένῃ.\FnParseFormGloss{f}{perfect passive participle dative feminine singular}{ξυράομαι}{to cut off the hair} 
\MainTextVerseMark{11}{6}εἰ γὰρ οὐ κατακαλύπτεται\FnParseFormGloss{a}{present middle indicative 3rd singular}{κατακαλύπτομαι}{to cover} γυνή, καὶ κειράσθω·\FnParseFormGloss{b}{aorist middle imperative 3rd singular}{κείρω}{to shear} εἰ δὲ αἰσχρὸν\FnParseFormGloss{c}{nominative neuter singular}{αἰσχρός}{disgraceful} γυναικὶ τὸ κείρασθαι\FnParseFormGloss{d}{aorist middle infinitive}{κείρω}{to shear} ἢ ξυρᾶσθαι,\FnParseFormGloss{e}{aorist middle infinitive}{ξυράομαι}{to cut off the hair} κατακαλυπτέσθω.\FnParseFormGloss{f}{present middle imperative 3rd singular}{κατακαλύπτομαι}{to cover} 
\MainTextVerseMark{11}{7}ἀνὴρ μὲν γὰρ οὐκ ὀφείλει\FnParse{a}{present active indicative 3rd singular} κατακαλύπτεσθαι\FnParseFormGloss{b}{present middle infinitive}{κατακαλύπτομαι}{to cover} τὴν κεφαλήν, εἰκὼν\FnParseFormGloss{c}{nominative feminine singular}{εἰκών}{image} καὶ δόξα θεοῦ ὑπάρχων·\FnParse{d}{present active participle nominative masculine singular} ⸀ἡ γυνὴ δὲ δόξα ἀνδρός ἐστιν.\FnParse{e}{present active indicative 3rd singular} 
\MainTextVerseMark{11}{8}οὐ γάρ ἐστιν\FnParse{a}{present active indicative 3rd singular} ἀνὴρ ἐκ γυναικός, ἀλλὰ γυνὴ ἐξ ἀνδρός· 
\MainTextVerseMark{11}{9}καὶ γὰρ οὐκ ἐκτίσθη\FnParseFormGloss{a}{aorist passive indicative 3rd singular}{κτίζω}{to create} ἀνὴρ διὰ τὴν γυναῖκα, ἀλλὰ γυνὴ διὰ τὸν ἄνδρα. 
\MainTextVerseMark{11}{10}διὰ τοῦτο ὀφείλει\FnParse{a}{present active indicative 3rd singular} ἡ γυνὴ ἐξουσίαν ἔχειν\FnParse{b}{present active infinitive} ἐπὶ τῆς κεφαλῆς διὰ τοὺς ἀγγέλους. 
\MainTextVerseMark{11}{11}πλὴν οὔτε ⸂γυνὴ χωρὶς ἀνδρὸς οὔτε ἀνὴρ χωρὶς γυναικὸς⸃ ἐν κυρίῳ· 
\MainTextVerseMark{11}{12}ὥσπερ γὰρ ἡ γυνὴ ἐκ τοῦ ἀνδρός, οὕτως καὶ ὁ ἀνὴρ διὰ τῆς γυναικός· τὰ δὲ πάντα ἐκ τοῦ θεοῦ. 
\MainTextVerseMark{11}{13}ἐν ὑμῖν αὐτοῖς κρίνατε·\FnParse{a}{aorist active imperative 2nd plural} πρέπον\FnParseFormGloss{b}{present active participle nominative neuter singular}{πρέπω}{to be proper} ἐστὶν\FnParse{c}{present active indicative 3rd singular} γυναῖκα ἀκατακάλυπτον\FnParseFormGloss{d}{accusative feminine singular}{ἀκατακάλυπτος}{uncovered} τῷ θεῷ προσεύχεσθαι;\FnParse{e}{present middle infinitive} 
\MainTextVerseMark{11}{14}⸀οὐδὲ ⸂ἡ φύσις\FnParseFormGloss{a}{nominative feminine singular}{φύσις}{nature} αὐτὴ⸃ διδάσκει\FnParse{b}{present active indicative 3rd singular} ὑμᾶς ὅτι ἀνὴρ μὲν ἐὰν κομᾷ,\FnParseFormGloss{c}{present active subjunctive 3rd singular}{κομάω}{to have long hair} ἀτιμία\FnParseFormGloss{d}{nominative feminine singular}{ἀτιμία}{dishonor} αὐτῷ ἐστιν,\FnParse{e}{present active indicative 3rd singular} 
\MainTextVerseMark{11}{15}γυνὴ δὲ ἐὰν κομᾷ,\FnParseFormGloss{a}{present active subjunctive 3rd singular}{κομάω}{to have long hair} δόξα αὐτῇ ἐστιν;\FnParse{b}{present active indicative 3rd singular} ὅτι ἡ κόμη\FnParseFormGloss{c}{nominative feminine singular}{κόμη}{hair} ἀντὶ\FnFormGloss{d}{ἀντί}{in exchange for} περιβολαίου\FnParseFormGloss{e}{genitive neuter singular}{περιβόλαιον}{covering} ⸀δέδοται.\FnParse{f}{perfect passive indicative 3rd singular} 
\MainTextVerseMark{11}{16}εἰ δέ τις δοκεῖ\FnParse{a}{present active indicative 3rd singular} φιλόνεικος\FnParseFormGloss{b}{nominative masculine singular}{φιλόνεικος}{contentious} εἶναι,\FnParse{c}{present active infinitive} ἡμεῖς τοιαύτην συνήθειαν\FnParseFormGloss{d}{accusative feminine singular}{συνήθεια}{custom} οὐκ ἔχομεν,\FnParse{e}{present active indicative 1st plural} οὐδὲ αἱ ἐκκλησίαι τοῦ θεοῦ. 
\MainTextVerseMark{11}{17}Τοῦτο δὲ ⸂παραγγέλλων\FnParse{a}{present active participle nominative masculine singular} οὐκ ἐπαινῶ⸃\FnParseFormGloss{b}{present active indicative 1st singular}{ἐπαινέω}{to praise} ὅτι οὐκ εἰς τὸ κρεῖσσον\FnParseFormGloss{c}{accusative neuter singular}{κρείττων}{better} ἀλλὰ εἰς τὸ ἧσσον\FnParseFormGloss{d}{accusative neuter singular}{ἥσσων}{for the worse} συνέρχεσθε.\FnParse{e}{present middle indicative 2nd plural} 
\MainTextVerseMark{11}{18}πρῶτον μὲν γὰρ συνερχομένων\FnParse{a}{present middle participle genitive masculine plural} ὑμῶν ἐν ἐκκλησίᾳ ἀκούω\FnParse{b}{present active indicative 1st singular} σχίσματα\FnParseFormGloss{c}{accusative neuter plural}{σχίσμα}{tear} ἐν ὑμῖν ὑπάρχειν,\FnParse{d}{present active infinitive} καὶ μέρος τι πιστεύω.\FnParse{e}{present active indicative 1st singular} 
\MainTextVerseMark{11}{19}δεῖ\FnParse{a}{present active indicative 3rd singular} γὰρ καὶ αἱρέσεις\FnParseFormGloss{b}{accusative feminine plural}{αἵρεσις}{sect} ἐν ὑμῖν εἶναι,\FnParse{c}{present active infinitive} ἵνα ⸀καὶ οἱ δόκιμοι\FnParseFormGloss{d}{nominative masculine plural}{δόκιμος}{approved by testing} φανεροὶ\FnParseFormGloss{e}{nominative masculine plural}{φανερός}{visible} γένωνται\FnParse{f}{aorist middle subjunctive 3rd plural} ἐν ὑμῖν. 
\MainTextVerseMark{11}{20}συνερχομένων\FnParse{a}{present middle participle genitive masculine plural} οὖν ὑμῶν ἐπὶ τὸ αὐτὸ οὐκ ἔστιν\FnParse{b}{present active indicative 3rd singular} κυριακὸν\FnParseFormGloss{c}{accusative neuter singular}{κυριακός}{belonging to the Lord} δεῖπνον\FnParseFormGloss{d}{accusative neuter singular}{δεῖπνον}{banquet} φαγεῖν,\FnParse{e}{aorist active infinitive} 
\MainTextVerseMark{11}{21}ἕκαστος γὰρ τὸ ἴδιον δεῖπνον\FnParseFormGloss{a}{accusative neuter singular}{δεῖπνον}{banquet} προλαμβάνει\FnParseFormGloss{b}{present active indicative 3rd singular}{προλαμβάνω}{to take beforehand} ἐν τῷ φαγεῖν,\FnParse{c}{aorist active infinitive} καὶ ὃς μὲν πεινᾷ,\FnParseFormGloss{d}{present active indicative 3rd singular}{πεινάω}{to be hungry} ὃς δὲ μεθύει.\FnParseFormGloss{e}{present active indicative 3rd singular}{μεθύω}{to get drunk} 
\MainTextVerseMark{11}{22}μὴ γὰρ οἰκίας οὐκ ἔχετε\FnParse{a}{present active indicative 2nd plural} εἰς τὸ ἐσθίειν\FnParse{b}{present active infinitive} καὶ πίνειν;\FnParse{c}{present active infinitive} ἢ τῆς ἐκκλησίας τοῦ θεοῦ καταφρονεῖτε,\FnParseFormGloss{d}{present active indicative 2nd plural}{καταφρονέω}{to despise} καὶ καταισχύνετε\FnParseFormGloss{e}{present active indicative 2nd plural}{καταισχύνω}{to dishonor} τοὺς μὴ ἔχοντας;\FnParse{f}{present active participle accusative masculine plural} τί ⸂εἴπω\FnParse{g}{aorist active subjunctive 1st singular} ὑμῖν⸃; ἐπαινέσω\FnParseFormGloss{h}{aorist active subjunctive 1st singular}{ἐπαινέω}{to praise} ὑμᾶς; ἐν τούτῳ οὐκ ἐπαινῶ.\FnParseFormGloss{i}{present active indicative 1st singular}{ἐπαινέω}{to praise} 
\MainTextVerseMark{11}{23}Ἐγὼ γὰρ παρέλαβον\FnParse{a}{aorist active indicative 1st singular} ἀπὸ τοῦ κυρίου, ὃ καὶ παρέδωκα\FnParse{b}{aorist active indicative 1st singular} ὑμῖν, ὅτι ὁ κύριος Ἰησοῦς ἐν τῇ νυκτὶ ᾗ ⸀παρεδίδετο\FnParse{c}{imperfect passive indicative 3rd singular} ἔλαβεν\FnParse{d}{aorist active indicative 3rd singular} ἄρτον 
\MainTextVerseMark{11}{24}καὶ εὐχαριστήσας\FnParse{a}{aorist active participle nominative masculine singular} ἔκλασεν\FnParseFormGloss{b}{aorist active indicative 3rd singular}{κλάω}{to break} καὶ ⸀εἶπεν·\FnParse{c}{aorist active indicative 3rd singular} Τοῦτό μού ἐστιν\FnParse{d}{present active indicative 3rd singular} τὸ σῶμα τὸ ὑπὲρ ⸀ὑμῶν· τοῦτο ποιεῖτε\FnParse{e}{present active imperative 2nd plural} εἰς τὴν ἐμὴν ἀνάμνησιν.\FnParseFormGloss{f}{accusative feminine singular}{ἀνάμνησις}{reminder} 
\MainTextVerseMark{11}{25}ὡσαύτως\FnFormGloss{a}{ὡσαύτως}{in the same way} καὶ τὸ ποτήριον μετὰ τὸ δειπνῆσαι,\FnParseFormGloss{b}{aorist active infinitive}{δειπνέω}{to eat supper} λέγων·\FnParse{c}{present active participle nominative masculine singular} Τοῦτο τὸ ποτήριον ἡ καινὴ διαθήκη ἐστὶν\FnParse{d}{present active indicative 3rd singular} ἐν τῷ ἐμῷ αἵματι· τοῦτο ποιεῖτε,\FnParse{e}{present active imperative 2nd plural} ὁσάκις\FnFormGloss{f}{ὁσάκις}{as often as} ⸀ἐὰν πίνητε,\FnParse{g}{present active subjunctive 2nd plural} εἰς τὴν ἐμὴν ἀνάμνησιν.\FnParseFormGloss{h}{accusative feminine singular}{ἀνάμνησις}{reminder} 
\MainTextVerseMark{11}{26}ὁσάκις\FnFormGloss{a}{ὁσάκις}{as often as} γὰρ ⸀ἐὰν ἐσθίητε\FnParse{b}{present active subjunctive 2nd plural} τὸν ἄρτον τοῦτον καὶ τὸ ⸀ποτήριον πίνητε,\FnParse{c}{present active subjunctive 2nd plural} τὸν θάνατον τοῦ κυρίου καταγγέλλετε,\FnParseFormGloss{d}{present active indicative 2nd plural}{καταγγέλλω}{to preach} ἄχρι ⸀οὗ ἔλθῃ.\FnParse{e}{aorist active subjunctive 3rd singular} 
\MainTextVerseMark{11}{27}Ὥστε ὃς ἂν ἐσθίῃ\FnParse{a}{present active subjunctive 3rd singular} τὸν ⸀ἄρτον ἢ πίνῃ\FnParse{b}{present active subjunctive 3rd singular} τὸ ποτήριον τοῦ κυρίου ⸀ἀναξίως,\FnFormGloss{c}{ἀναξίως}{unworthily} ἔνοχος\FnParseFormGloss{d}{nominative masculine singular}{ἔνοχος}{subject to} ἔσται\FnParse{e}{future middle indicative 3rd singular} τοῦ σώματος καὶ τοῦ αἵματος τοῦ κυρίου. 
\MainTextVerseMark{11}{28}δοκιμαζέτω\FnParseFormGloss{a}{present active imperative 3rd singular}{δοκιμάζω}{to test} δὲ ἄνθρωπος ἑαυτόν, καὶ οὕτως ἐκ τοῦ ἄρτου ἐσθιέτω\FnParse{b}{present active imperative 3rd singular} καὶ ἐκ τοῦ ποτηρίου πινέτω·\FnParse{c}{present active imperative 3rd singular} 
\MainTextVerseMark{11}{29}ὁ γὰρ ἐσθίων\FnParse{a}{present active participle nominative masculine singular} καὶ ⸀πίνων\FnParse{b}{present active participle nominative masculine singular} κρίμα\FnParseFormGloss{c}{accusative neuter singular}{κρίμα}{judgment} ἑαυτῷ ἐσθίει\FnParse{d}{present active indicative 3rd singular} καὶ πίνει\FnParse{e}{present active indicative 3rd singular} μὴ διακρίνων\FnParseFormGloss{f}{present active participle nominative masculine singular}{διακρίνω}{to make a distinction} τὸ ⸀σῶμα. 
\MainTextVerseMark{11}{30}διὰ τοῦτο ἐν ὑμῖν πολλοὶ ἀσθενεῖς\FnParseFormGloss{a}{nominative masculine plural}{ἀσθενής}{weak} καὶ ἄρρωστοι\FnParseFormGloss{b}{nominative masculine plural}{ἄρρωστος}{sick} καὶ κοιμῶνται\FnParseFormGloss{c}{present middle indicative 3rd plural}{κοιμάομαι}{to fall asleep} ἱκανοί. 
\MainTextVerseMark{11}{31}εἰ ⸀δὲ ἑαυτοὺς διεκρίνομεν,\FnParseFormGloss{a}{imperfect active indicative 1st plural}{διακρίνω}{to make a distinction} οὐκ ἂν ἐκρινόμεθα·\FnParse{b}{imperfect passive indicative 1st plural} 
\MainTextVerseMark{11}{32}κρινόμενοι\FnParse{a}{present passive participle nominative masculine plural} δὲ ⸀ὑπὸ κυρίου παιδευόμεθα,\FnParseFormGloss{b}{present passive indicative 1st plural}{παιδεύω}{instruct} ἵνα μὴ σὺν τῷ κόσμῳ κατακριθῶμεν.\FnParseFormGloss{c}{aorist passive subjunctive 1st plural}{κατακρίνω}{to condemn} 
\MainTextVerseMark{11}{33}Ὥστε, ἀδελφοί μου, συνερχόμενοι\FnParse{a}{present middle participle nominative masculine plural} εἰς τὸ φαγεῖν\FnParse{b}{aorist active infinitive} ἀλλήλους ἐκδέχεσθε.\FnParseFormGloss{c}{present middle imperative 2nd plural}{ἐκδέχομαι}{to wait for} 
\MainTextVerseMark{11}{34}⸀εἴ τις πεινᾷ,\FnParseFormGloss{a}{present active indicative 3rd singular}{πεινάω}{to be hungry} ἐν οἴκῳ ἐσθιέτω,\FnParse{b}{present active imperative 3rd singular} ἵνα μὴ εἰς κρίμα\FnParseFormGloss{c}{accusative neuter singular}{κρίμα}{judgment} συνέρχησθε.\FnParse{d}{present middle subjunctive 2nd plural} Τὰ δὲ λοιπὰ ὡς ἂν ἔλθω\FnParse{e}{aorist active subjunctive 1st singular} διατάξομαι.\FnParseFormGloss{f}{future middle indicative 1st singular}{διατάσσω}{to command} 
\ChapterHeader{12}{Chapter 12}

\MainTextVerseMark{12}{1}Περὶ δὲ τῶν πνευματικῶν,\FnParseFormGloss{a}{genitive neuter plural}{πνευματικός}{spiritual} ἀδελφοί, οὐ θέλω\FnParse{b}{present active indicative 1st singular} ὑμᾶς ἀγνοεῖν.\FnParseFormGloss{c}{present active infinitive}{ἀγνοέω}{to be ignorant} 
\MainTextVerseMark{12}{2}οἴδατε\FnParse{a}{perfect active indicative 2nd plural} ὅτι ὅτε ἔθνη ἦτε\FnParse{b}{imperfect active indicative 2nd plural} πρὸς τὰ εἴδωλα\FnParseFormGloss{c}{accusative neuter plural}{εἴδωλον}{idol} τὰ ἄφωνα\FnParseFormGloss{d}{accusative neuter plural}{ἄφωνος}{silent} ὡς ἂν ἤγεσθε\FnParse{e}{imperfect passive indicative 2nd plural} ἀπαγόμενοι.\FnParseFormGloss{f}{present passive participle nominative masculine plural}{ἀπάγω}{to lead away} 
\MainTextVerseMark{12}{3}διὸ γνωρίζω\FnParseFormGloss{a}{present active indicative 1st singular}{γνωρίζω}{to make known} ὑμῖν ὅτι οὐδεὶς ἐν πνεύματι θεοῦ λαλῶν\FnParse{b}{present active participle nominative masculine singular} λέγει·\FnParse{c}{present active indicative 3rd singular} Ἀνάθεμα\FnParseFormGloss{d}{nominative neuter singular}{ἀνάθεμα}{curse} ⸀Ἰησοῦς, καὶ οὐδεὶς δύναται\FnParse{e}{present middle indicative 3rd singular} εἰπεῖν·\FnParse{f}{aorist active infinitive} ⸂Κύριος Ἰησοῦς⸃ εἰ μὴ ἐν πνεύματι ἁγίῳ. 
\MainTextVerseMark{12}{4}Διαιρέσεις\FnParseFormGloss{a}{nominative feminine plural}{διαίρεσις}{difference} δὲ χαρισμάτων\FnParseFormGloss{b}{genitive neuter plural}{χάρισμα}{gracious gift} εἰσίν,\FnParse{c}{present active indicative 3rd plural} τὸ δὲ αὐτὸ πνεῦμα· 
\MainTextVerseMark{12}{5}καὶ διαιρέσεις\FnParseFormGloss{a}{nominative feminine plural}{διαίρεσις}{difference} διακονιῶν εἰσιν,\FnParse{b}{present active indicative 3rd plural} καὶ ὁ αὐτὸς κύριος· 
\MainTextVerseMark{12}{6}καὶ διαιρέσεις\FnParseFormGloss{a}{nominative feminine plural}{διαίρεσις}{difference} ἐνεργημάτων\FnParseFormGloss{b}{genitive neuter plural}{ἐνέργημα}{working} εἰσίν,\FnParse{c}{present active indicative 3rd plural} ⸂ὁ δὲ⸃ ⸀αὐτὸς θεός, ὁ ἐνεργῶν\FnParseFormGloss{d}{present active participle nominative masculine singular}{ἐνεργέω}{to be at work in} τὰ πάντα ἐν πᾶσιν. 
\MainTextVerseMark{12}{7}ἑκάστῳ δὲ δίδοται\FnParse{a}{present passive indicative 3rd singular} ἡ φανέρωσις\FnParseFormGloss{b}{nominative feminine singular}{φανέρωσις}{manifestation} τοῦ πνεύματος πρὸς τὸ συμφέρον.\FnParseFormGloss{c}{present active participle accusative neuter singular}{συμφέρω}{to bring together} 
\MainTextVerseMark{12}{8}ᾧ μὲν γὰρ διὰ τοῦ πνεύματος δίδοται\FnParse{a}{present passive indicative 3rd singular} λόγος σοφίας, ἄλλῳ δὲ λόγος γνώσεως\FnParseFormGloss{b}{genitive feminine singular}{γνῶσις}{knowledge} κατὰ τὸ αὐτὸ πνεῦμα, 
\MainTextVerseMark{12}{9}⸀ἑτέρῳ πίστις ἐν τῷ αὐτῷ πνεύματι, ⸀ἄλλῳ χαρίσματα\FnParseFormGloss{a}{nominative neuter plural}{χάρισμα}{gracious gift} ἰαμάτων\FnParseFormGloss{b}{genitive neuter plural}{ἴαμα}{healing} ἐν τῷ ⸀ἑνὶ πνεύματι, 
\MainTextVerseMark{12}{10}⸀ἄλλῳ ἐνεργήματα\FnParseFormGloss{a}{nominative neuter plural}{ἐνέργημα}{working} δυνάμεων, ⸁ἄλλῳ προφητεία,\FnParseFormGloss{b}{nominative feminine singular}{προφητεία}{prophecy} ⸀1ἄλλῳ διακρίσεις\FnParseFormGloss{c}{nominative feminine plural}{διάκρισις}{distinguishing} πνευμάτων, ⸀ἑτέρῳ γένη\FnParseFormGloss{d}{nominative neuter plural}{γένος}{family} γλωσσῶν, ⸀2ἄλλῳ ἑρμηνεία\FnParseFormGloss{e}{nominative feminine singular}{ἑρμηνεία}{interpretation} γλωσσῶν· 
\MainTextVerseMark{12}{11}πάντα δὲ ταῦτα ἐνεργεῖ\FnParseFormGloss{a}{present active indicative 3rd singular}{ἐνεργέω}{to be at work in} τὸ ἓν καὶ τὸ αὐτὸ πνεῦμα, διαιροῦν\FnParseFormGloss{b}{present active participle nominative neuter singular}{διαιρέω}{to divide} ἰδίᾳ ἑκάστῳ καθὼς βούλεται.\FnParse{c}{present middle indicative 3rd singular} 
\MainTextVerseMark{12}{12}Καθάπερ\FnFormGloss{a}{καθάπερ}{as} γὰρ τὸ σῶμα ἕν ἐστιν\FnParse{b}{present active indicative 3rd singular} καὶ μέλη ⸂πολλὰ ἔχει⸃,\FnParse{c}{present active indicative 3rd singular} πάντα δὲ τὰ μέλη τοῦ ⸀σώματος πολλὰ ὄντα\FnParse{d}{present active participle nominative neuter plural} ἕν ἐστιν\FnParse{e}{present active indicative 3rd singular} σῶμα, οὕτως καὶ ὁ Χριστός· 
\MainTextVerseMark{12}{13}καὶ γὰρ ἐν ἑνὶ πνεύματι ἡμεῖς πάντες εἰς ἓν σῶμα ἐβαπτίσθημεν,\FnParse{a}{aorist passive indicative 1st plural} εἴτε Ἰουδαῖοι εἴτε Ἕλληνες,\FnParseFormGloss{b}{nominative masculine plural}{Ἕλλην}{Greek} εἴτε δοῦλοι εἴτε ἐλεύθεροι,\FnParseFormGloss{c}{nominative masculine plural}{ἐλεύθερος}{free} καὶ ⸀πάντες ἓν πνεῦμα ἐποτίσθημεν.\FnParseFormGloss{d}{aorist passive indicative 1st plural}{ποτίζω}{to give a drink} 
\MainTextVerseMark{12}{14}Καὶ γὰρ τὸ σῶμα οὐκ ἔστιν\FnParse{a}{present active indicative 3rd singular} ἓν μέλος ἀλλὰ πολλά. 
\MainTextVerseMark{12}{15}ἐὰν εἴπῃ\FnParse{a}{aorist active subjunctive 3rd singular} ὁ πούς· Ὅτι οὐκ εἰμὶ\FnParse{b}{present active indicative 1st singular} χείρ, οὐκ εἰμὶ\FnParse{c}{present active indicative 1st singular} ἐκ τοῦ σώματος, οὐ παρὰ τοῦτο οὐκ ἔστιν\FnParse{d}{present active indicative 3rd singular} ἐκ τοῦ σώματος; 
\MainTextVerseMark{12}{16}καὶ ἐὰν εἴπῃ\FnParse{a}{aorist active subjunctive 3rd singular} τὸ οὖς· Ὅτι οὐκ εἰμὶ\FnParse{b}{present active indicative 1st singular} ὀφθαλμός, οὐκ εἰμὶ\FnParse{c}{present active indicative 1st singular} ἐκ τοῦ σώματος, οὐ παρὰ τοῦτο οὐκ ἔστιν\FnParse{d}{present active indicative 3rd singular} ἐκ τοῦ σώματος· 
\MainTextVerseMark{12}{17}εἰ ὅλον τὸ σῶμα ὀφθαλμός, ποῦ ἡ ἀκοή;\FnParseFormGloss{a}{nominative feminine singular}{ἀκοή}{hearing} εἰ ὅλον ἀκοή,\FnParseFormGloss{b}{nominative feminine singular}{ἀκοή}{hearing} ποῦ ἡ ὄσφρησις;\FnParseFormGloss{c}{nominative feminine singular}{ὄσφρησις}{sense of smell} 
\MainTextVerseMark{12}{18}⸀νυνὶ\FnFormGloss{a}{νυνί}{now} δὲ ὁ θεὸς ἔθετο\FnParse{b}{aorist middle indicative 3rd singular} τὰ μέλη, ἓν ἕκαστον αὐτῶν, ἐν τῷ σώματι καθὼς ἠθέλησεν.\FnParse{c}{aorist active indicative 3rd singular} 
\MainTextVerseMark{12}{19}εἰ δὲ ἦν\FnParse{a}{imperfect active indicative 3rd singular} τὰ πάντα ἓν μέλος, ποῦ τὸ σῶμα; 
\MainTextVerseMark{12}{20}νῦν δὲ πολλὰ ⸀μὲν μέλη, ἓν δὲ σῶμα. 
\MainTextVerseMark{12}{21}οὐ δύναται\FnParse{a}{present middle indicative 3rd singular} δὲ ὁ ὀφθαλμὸς εἰπεῖν\FnParse{b}{aorist active infinitive} τῇ χειρί· Χρείαν σου οὐκ ἔχω,\FnParse{c}{present active indicative 1st singular} ἢ πάλιν ἡ κεφαλὴ τοῖς ποσίν· Χρείαν ὑμῶν οὐκ ἔχω·\FnParse{d}{present active indicative 1st singular} 
\MainTextVerseMark{12}{22}ἀλλὰ πολλῷ μᾶλλον τὰ δοκοῦντα\FnParse{a}{present active participle nominative neuter plural} μέλη τοῦ σώματος ἀσθενέστερα\FnParseFormGloss{b}{nominative neuter plural}{ἀσθενής}{weak} ὑπάρχειν\FnParse{c}{present active infinitive} ἀναγκαῖά\FnParseFormGloss{d}{nominative neuter plural}{ἀναγκαῖος}{necessary} ἐστιν,\FnParse{e}{present active indicative 3rd singular} 
\MainTextVerseMark{12}{23}καὶ ἃ δοκοῦμεν\FnParse{a}{present active indicative 1st plural} ἀτιμότερα\FnParseFormGloss{b}{accusative neuter plural}{ἄτιμος}{without honor} εἶναι\FnParse{c}{present active infinitive} τοῦ σώματος, τούτοις τιμὴν περισσοτέραν\FnParseFormGloss{d}{accusative feminine singular}{περισσότερος}{more than} περιτίθεμεν,\FnParseFormGloss{e}{present active indicative 1st plural}{περιτίθημι}{to put on} καὶ τὰ ἀσχήμονα\FnParseFormGloss{f}{nominative neuter plural}{ἀσχήμων}{unpresentable} ἡμῶν εὐσχημοσύνην\FnParseFormGloss{g}{accusative feminine singular}{εὐσχημοσύνη}{modesty} περισσοτέραν\FnParseFormGloss{h}{accusative feminine singular}{περισσότερος}{more than} ἔχει,\FnParse{i}{present active indicative 3rd singular} 
\MainTextVerseMark{12}{24}τὰ δὲ εὐσχήμονα\FnParseFormGloss{a}{nominative neuter plural}{εὐσχήμων}{presentable} ἡμῶν οὐ χρείαν ⸀ἔχει.\FnParse{b}{present active indicative 3rd singular} ἀλλὰ ὁ θεὸς συνεκέρασεν\FnParseFormGloss{c}{aorist active indicative 3rd singular}{συγκεράννυμι}{to combine} τὸ σῶμα, τῷ ⸀ὑστεροῦντι\FnParseFormGloss{d}{present active participle dative neuter singular}{ὑστερέω}{to lack} περισσοτέραν\FnParseFormGloss{e}{accusative feminine singular}{περισσότερος}{more than} δοὺς\FnParse{f}{aorist active participle nominative masculine singular} τιμήν, 
\MainTextVerseMark{12}{25}ἵνα μὴ ᾖ\FnParse{a}{present active subjunctive 3rd singular} ⸀σχίσμα\FnParseFormGloss{b}{nominative neuter singular}{σχίσμα}{tear} ἐν τῷ σώματι, ἀλλὰ τὸ αὐτὸ ὑπὲρ ἀλλήλων μεριμνῶσι\FnParseFormGloss{c}{present active subjunctive 3rd plural}{μεριμνάω}{to worry} τὰ μέλη. 
\MainTextVerseMark{12}{26}καὶ ⸀εἴτε πάσχει\FnParse{a}{present active indicative 3rd singular} ἓν μέλος, συμπάσχει\FnParseFormGloss{b}{present active indicative 3rd singular}{συμπάσχω}{to suffer with} πάντα τὰ μέλη· εἴτε δοξάζεται\FnParse{c}{present passive indicative 3rd singular} ⸀μέλος, συγχαίρει\FnParseFormGloss{d}{present active indicative 3rd singular}{συγχαίρω}{to rejoice with} πάντα τὰ μέλη. 
\MainTextVerseMark{12}{27}Ὑμεῖς δέ ἐστε\FnParse{a}{present active indicative 2nd plural} σῶμα Χριστοῦ καὶ μέλη ἐκ μέρους. 
\MainTextVerseMark{12}{28}καὶ οὓς μὲν ἔθετο\FnParse{a}{aorist middle indicative 3rd singular} ὁ θεὸς ἐν τῇ ἐκκλησίᾳ πρῶτον ἀποστόλους, δεύτερον προφήτας, τρίτον διδασκάλους, ἔπειτα\FnFormGloss{b}{ἔπειτα}{then} δυνάμεις, ⸀ἔπειτα\FnFormGloss{c}{ἔπειτα}{then} χαρίσματα\FnParseFormGloss{d}{accusative neuter plural}{χάρισμα}{gracious gift} ἰαμάτων,\FnParseFormGloss{e}{genitive neuter plural}{ἴαμα}{healing} ἀντιλήμψεις,\FnParseFormGloss{f}{accusative feminine plural}{ἀντίλημψις}{help} κυβερνήσεις,\FnParseFormGloss{g}{accusative feminine plural}{κυβέρνησις}{administration} γένη\FnParseFormGloss{h}{accusative neuter plural}{γένος}{family} γλωσσῶν. 
\MainTextVerseMark{12}{29}μὴ πάντες ἀπόστολοι; μὴ πάντες προφῆται; μὴ πάντες διδάσκαλοι; μὴ πάντες δυνάμεις; 
\MainTextVerseMark{12}{30}μὴ πάντες χαρίσματα\FnParseFormGloss{a}{accusative neuter plural}{χάρισμα}{gracious gift} ἔχουσιν\FnParse{b}{present active indicative 3rd plural} ἰαμάτων;\FnParseFormGloss{c}{genitive neuter plural}{ἴαμα}{healing} μὴ πάντες γλώσσαις λαλοῦσιν;\FnParse{d}{present active indicative 3rd plural} μὴ πάντες διερμηνεύουσιν;\FnParseFormGloss{e}{present active indicative 3rd plural}{διερμηνεύω}{to interpret} 
\MainTextVerseMark{12}{31}ζηλοῦτε\FnParseFormGloss{a}{present active imperative 2nd plural}{ζηλόω}{to desire} δὲ τὰ χαρίσματα\FnParseFormGloss{b}{accusative neuter plural}{χάρισμα}{gracious gift} τὰ ⸀μείζονα. καὶ ἔτι καθ’ ὑπερβολὴν\FnParseFormGloss{c}{accusative feminine singular}{ὑπερβολή}{all-surpassing} ὁδὸν ὑμῖν δείκνυμι.\FnParse{d}{present active indicative 1st singular} 
\ChapterHeader{13}{Chapter 13}

\MainTextVerseMark{13}{1}Ἐὰν ταῖς γλώσσαις τῶν ἀνθρώπων λαλῶ\FnParse{a}{present active subjunctive 1st singular} καὶ τῶν ἀγγέλων, ἀγάπην δὲ μὴ ἔχω,\FnParse{b}{present active subjunctive 1st singular} γέγονα\FnParse{c}{perfect active indicative 1st singular} χαλκὸς\FnParseFormGloss{d}{nominative masculine singular}{χαλκός}{copper} ἠχῶν\FnParseFormGloss{e}{present active participle nominative masculine singular}{ἠχέω}{to resound} ἢ κύμβαλον\FnParseFormGloss{f}{nominative neuter singular}{κύμβαλον}{cymbal} ἀλαλάζον.\FnParseFormGloss{g}{present active participle nominative neuter singular}{ἀλαλάζω}{to clang} 
\MainTextVerseMark{13}{2}⸂καὶ ἐὰν⸃ ἔχω\FnParse{a}{present active subjunctive 1st singular} προφητείαν\FnParseFormGloss{b}{accusative feminine singular}{προφητεία}{prophecy} καὶ εἰδῶ\FnParse{c}{perfect active subjunctive 1st singular} τὰ μυστήρια\FnParseFormGloss{d}{accusative neuter plural}{μυστήριον}{mystery} πάντα καὶ πᾶσαν τὴν γνῶσιν,\FnParseFormGloss{e}{accusative feminine singular}{γνῶσις}{knowledge} ⸄καὶ ἐὰν⸅ ἔχω\FnParse{f}{present active subjunctive 1st singular} πᾶσαν τὴν πίστιν ὥστε ὄρη ⸀μεθιστάναι,\FnParseFormGloss{g}{present active infinitive}{μεθίστημι}{to move} ἀγάπην δὲ μὴ ἔχω,\FnParse{h}{present active subjunctive 1st singular} οὐθέν εἰμι.\FnParse{i}{present active indicative 1st singular} 
\MainTextVerseMark{13}{3}⸂καὶ ἐὰν⸃ ψωμίσω\FnParseFormGloss{a}{aorist active subjunctive 1st singular}{ψωμίζω}{to feed} πάντα τὰ ὑπάρχοντά\FnParse{b}{present active participle accusative neuter plural} μου, ⸄καὶ ἐὰν⸃ παραδῶ\FnParse{c}{aorist active subjunctive 1st singular} τὸ σῶμά μου, ἵνα ⸀καυθήσομαι,\FnParseFormGloss{d}{future passive indicative 1st singular}{καίω}{to light} ἀγάπην δὲ μὴ ἔχω,\FnParse{e}{present active subjunctive 1st singular} οὐδὲν ὠφελοῦμαι.\FnParseFormGloss{f}{present passive indicative 1st singular}{ὠφελέω}{to be of good use} 
\MainTextVerseMark{13}{4}Ἡ ἀγάπη μακροθυμεῖ,\FnParseFormGloss{a}{present active indicative 3rd singular}{μακροθυμέω}{to have patience} χρηστεύεται\FnParseFormGloss{b}{present middle indicative 3rd singular}{χρηστεύομαι}{to be kind} ἡ ἀγάπη, οὐ ζηλοῖ\FnParseFormGloss{c}{present active indicative 3rd singular}{ζηλόω}{to desire} ⸂ἡ ἀγάπη⸃, οὐ περπερεύεται,\FnParseFormGloss{d}{present middle indicative 3rd singular}{περπερεύομαι}{to boast} οὐ φυσιοῦται,\FnParseFormGloss{e}{present passive indicative 3rd singular}{φυσιόω}{to puff up} 
\MainTextVerseMark{13}{5}οὐκ ἀσχημονεῖ,\FnParseFormGloss{a}{present active indicative 3rd singular}{ἀσχημονέω}{to act improperly} οὐ ζητεῖ\FnParse{b}{present active indicative 3rd singular} τὰ ἑαυτῆς, οὐ παροξύνεται,\FnParseFormGloss{c}{present passive indicative 3rd singular}{παροξύνομαι}{to be greatly distressed} οὐ λογίζεται\FnParse{d}{present middle indicative 3rd singular} τὸ κακόν, 
\MainTextVerseMark{13}{6}οὐ χαίρει\FnParse{a}{present active indicative 3rd singular} ἐπὶ τῇ ἀδικίᾳ,\FnParseFormGloss{b}{dative feminine singular}{ἀδικία}{wickedness} συγχαίρει\FnParseFormGloss{c}{present active indicative 3rd singular}{συγχαίρω}{to rejoice with} δὲ τῇ ἀληθείᾳ· 
\MainTextVerseMark{13}{7}πάντα στέγει,\FnParseFormGloss{a}{present active indicative 3rd singular}{στέγω}{to put up with} πάντα πιστεύει,\FnParse{b}{present active indicative 3rd singular} πάντα ἐλπίζει,\FnParse{c}{present active indicative 3rd singular} πάντα ὑπομένει.\FnParseFormGloss{d}{present active indicative 3rd singular}{ὑπομένω}{to stay behind} 
\MainTextVerseMark{13}{8}Ἡ ἀγάπη οὐδέποτε\FnFormGloss{a}{οὐδέποτε}{never} ⸀πίπτει.\FnParse{b}{present active indicative 3rd singular} εἴτε δὲ προφητεῖαι,\FnParseFormGloss{c}{nominative feminine plural}{προφητεία}{prophecy} καταργηθήσονται·\FnParseFormGloss{d}{future passive indicative 3rd plural}{καταργέω}{to nullify} εἴτε γλῶσσαι, παύσονται·\FnParseFormGloss{e}{future middle indicative 3rd plural}{παύω}{to cause to stop} εἴτε γνῶσις,\FnParseFormGloss{f}{nominative feminine singular}{γνῶσις}{knowledge} καταργηθήσεται.\FnParseFormGloss{g}{future passive indicative 3rd singular}{καταργέω}{to nullify} 
\MainTextVerseMark{13}{9}ἐκ μέρους ⸀γὰρ γινώσκομεν\FnParse{a}{present active indicative 1st plural} καὶ ἐκ μέρους προφητεύομεν·\FnParseFormGloss{b}{present active indicative 1st plural}{προφητεύω}{to prophesy} 
\MainTextVerseMark{13}{10}ὅταν δὲ ἔλθῃ\FnParse{a}{aorist active subjunctive 3rd singular} τὸ τέλειον,\FnParseFormGloss{b}{nominative neuter singular}{τέλειος}{perfect} ⸀τὸ ἐκ μέρους καταργηθήσεται.\FnParseFormGloss{c}{future passive indicative 3rd singular}{καταργέω}{to nullify} 
\MainTextVerseMark{13}{11}ὅτε ἤμην\FnParse{a}{imperfect middle indicative 1st singular} νήπιος,\FnParseFormGloss{b}{nominative masculine singular}{νήπιος}{child} ⸂ἐλάλουν\FnParse{c}{imperfect active indicative 1st singular} ὡς νήπιος,\FnParseFormGloss{d}{nominative masculine singular}{νήπιος}{child} ἐφρόνουν\FnParseFormGloss{e}{imperfect active indicative 1st singular}{φρονέω}{to think} ὡς νήπιος,\FnParseFormGloss{f}{nominative masculine singular}{νήπιος}{child} ἐλογιζόμην\FnParse{g}{imperfect middle indicative 1st singular} ὡς νήπιος⸃·\FnParseFormGloss{h}{nominative masculine singular}{νήπιος}{child} ⸀ὅτε γέγονα\FnParse{i}{perfect active indicative 1st singular} ἀνήρ, κατήργηκα\FnParseFormGloss{j}{perfect active indicative 1st singular}{καταργέω}{to nullify} τὰ τοῦ νηπίου.\FnParseFormGloss{k}{genitive masculine singular}{νήπιος}{child} 
\MainTextVerseMark{13}{12}βλέπομεν\FnParse{a}{present active indicative 1st plural} γὰρ ἄρτι δι’ ἐσόπτρου\FnParseFormGloss{b}{genitive neuter singular}{ἔσοπτρον}{mirror} ἐν αἰνίγματι,\FnParseFormGloss{c}{dative neuter singular}{αἴνιγμα}{poor reflection} τότε δὲ πρόσωπον πρὸς πρόσωπον· ἄρτι γινώσκω\FnParse{d}{present active indicative 1st singular} ἐκ μέρους, τότε δὲ ἐπιγνώσομαι\FnParse{e}{future middle indicative 1st singular} καθὼς καὶ ἐπεγνώσθην.\FnParse{f}{aorist passive indicative 1st singular} 
\MainTextVerseMark{13}{13}νυνὶ\FnFormGloss{a}{νυνί}{now} δὲ μένει\FnParse{b}{present active indicative 3rd singular} πίστις, ἐλπίς, ἀγάπη· τὰ τρία ταῦτα, μείζων δὲ τούτων ἡ ἀγάπη. 
\ChapterHeader{14}{Chapter 14}

\MainTextVerseMark{14}{1}Διώκετε\FnParse{a}{present active imperative 2nd plural} τὴν ἀγάπην, ζηλοῦτε\FnParseFormGloss{b}{present active imperative 2nd plural}{ζηλόω}{to desire} δὲ τὰ πνευματικά,\FnParseFormGloss{c}{accusative neuter plural}{πνευματικός}{spiritual} μᾶλλον δὲ ἵνα προφητεύητε.\FnParseFormGloss{d}{present active subjunctive 2nd plural}{προφητεύω}{to prophesy} 
\MainTextVerseMark{14}{2}ὁ γὰρ λαλῶν\FnParse{a}{present active participle nominative masculine singular} γλώσσῃ οὐκ ἀνθρώποις λαλεῖ\FnParse{b}{present active indicative 3rd singular} ⸀ἀλλὰ θεῷ, οὐδεὶς γὰρ ἀκούει,\FnParse{c}{present active indicative 3rd singular} πνεύματι δὲ λαλεῖ\FnParse{d}{present active indicative 3rd singular} μυστήρια·\FnParseFormGloss{e}{accusative neuter plural}{μυστήριον}{mystery} 
\MainTextVerseMark{14}{3}ὁ δὲ προφητεύων\FnParseFormGloss{a}{present active participle nominative masculine singular}{προφητεύω}{to prophesy} ἀνθρώποις λαλεῖ\FnParse{b}{present active indicative 3rd singular} οἰκοδομὴν\FnParseFormGloss{c}{accusative feminine singular}{οἰκοδομή}{building} καὶ παράκλησιν\FnParseFormGloss{d}{accusative feminine singular}{παράκλησις}{encouragement} καὶ παραμυθίαν.\FnParseFormGloss{e}{accusative feminine singular}{παραμυθία}{comfort} 
\MainTextVerseMark{14}{4}ὁ λαλῶν\FnParse{a}{present active participle nominative masculine singular} γλώσσῃ ἑαυτὸν οἰκοδομεῖ·\FnParse{b}{present active indicative 3rd singular} ὁ δὲ προφητεύων\FnParseFormGloss{c}{present active participle nominative masculine singular}{προφητεύω}{to prophesy} ἐκκλησίαν οἰκοδομεῖ.\FnParse{d}{present active indicative 3rd singular} 
\MainTextVerseMark{14}{5}θέλω\FnParse{a}{present active indicative 1st singular} δὲ πάντας ὑμᾶς λαλεῖν\FnParse{b}{present active infinitive} γλώσσαις, μᾶλλον δὲ ἵνα προφητεύητε·\FnParseFormGloss{c}{present active subjunctive 2nd plural}{προφητεύω}{to prophesy} μείζων ⸀δὲ ὁ προφητεύων\FnParseFormGloss{d}{present active participle nominative masculine singular}{προφητεύω}{to prophesy} ἢ ὁ λαλῶν\FnParse{e}{present active participle nominative masculine singular} γλώσσαις, ἐκτὸς\FnFormGloss{f}{ἐκτός}{outside} εἰ μὴ ⸀διερμηνεύῃ,\FnParseFormGloss{g}{present active subjunctive 3rd singular}{διερμηνεύω}{to interpret} ἵνα ἡ ἐκκλησία οἰκοδομὴν\FnParseFormGloss{h}{accusative feminine singular}{οἰκοδομή}{building} λάβῃ.\FnParse{i}{aorist active subjunctive 3rd singular} 
\MainTextVerseMark{14}{6}⸀Νῦν δέ, ἀδελφοί, ἐὰν ἔλθω\FnParse{a}{aorist active subjunctive 1st singular} πρὸς ὑμᾶς γλώσσαις λαλῶν,\FnParse{b}{present active participle nominative masculine singular} τί ὑμᾶς ὠφελήσω,\FnParseFormGloss{c}{future active indicative 1st singular}{ὠφελέω}{to be of good use} ἐὰν μὴ ὑμῖν λαλήσω\FnParse{d}{aorist active subjunctive 1st singular} ἢ ἐν ἀποκαλύψει\FnParseFormGloss{e}{dative feminine singular}{ἀποκάλυψις}{revelation} ἢ ἐν γνώσει\FnParseFormGloss{f}{dative feminine singular}{γνῶσις}{knowledge} ἢ ἐν προφητείᾳ\FnParseFormGloss{g}{dative feminine singular}{προφητεία}{prophecy} ἢ ἐν διδαχῇ; 
\MainTextVerseMark{14}{7}ὅμως\FnFormGloss{a}{ὅμως}{just as} τὰ ἄψυχα\FnParseFormGloss{b}{nominative neuter plural}{ἄψυχος}{lifeless} φωνὴν διδόντα,\FnParse{c}{present active participle nominative neuter plural} εἴτε αὐλὸς\FnParseFormGloss{d}{nominative masculine singular}{αὐλός}{flute} εἴτε κιθάρα,\FnParseFormGloss{e}{nominative feminine singular}{κιθάρα}{harp} ἐὰν διαστολὴν\FnParseFormGloss{f}{accusative feminine singular}{διαστολή}{difference} τοῖς φθόγγοις\FnParseFormGloss{g}{dative masculine plural}{φθόγγος}{voice} μὴ ⸀δῷ,\FnParse{h}{aorist active subjunctive 3rd singular} πῶς γνωσθήσεται\FnParse{i}{future passive indicative 3rd singular} τὸ αὐλούμενον\FnParseFormGloss{j}{present passive participle nominative neuter singular}{αὐλέω}{to play the flute} ἢ τὸ κιθαριζόμενον;\FnParseFormGloss{k}{present passive participle nominative neuter singular}{κιθαρίζω}{to play the harp} 
\MainTextVerseMark{14}{8}καὶ γὰρ ἐὰν ἄδηλον\FnParseFormGloss{a}{accusative feminine singular}{ἄδηλος}{not clear} ⸂φωνὴν σάλπιγξ⸃\FnParseFormGloss{b}{nominative feminine singular}{σάλπιγξ}{trumpet} δῷ,\FnParse{c}{aorist active subjunctive 3rd singular} τίς παρασκευάσεται\FnParseFormGloss{d}{future middle indicative 3rd singular}{παρασκευάζω}{to prepare} εἰς πόλεμον;\FnParseFormGloss{e}{accusative masculine singular}{πόλεμος}{war} 
\MainTextVerseMark{14}{9}οὕτως καὶ ὑμεῖς διὰ τῆς γλώσσης ἐὰν μὴ εὔσημον\FnParseFormGloss{a}{accusative masculine singular}{εὔσημος}{intelligible} λόγον δῶτε,\FnParse{b}{aorist active subjunctive 2nd plural} πῶς γνωσθήσεται\FnParse{c}{future passive indicative 3rd singular} τὸ λαλούμενον;\FnParse{d}{present passive participle nominative neuter singular} ἔσεσθε\FnParse{e}{future middle indicative 2nd plural} γὰρ εἰς ἀέρα\FnParseFormGloss{f}{accusative masculine singular}{ἀήρ}{air} λαλοῦντες.\FnParse{g}{present active participle nominative masculine plural} 
\MainTextVerseMark{14}{10}τοσαῦτα\FnParseFormGloss{a}{nominative neuter plural}{τοσοῦτος}{so great} εἰ τύχοι\FnParseFormGloss{b}{aorist active optative 3rd singular}{τυγχάνω}{to take part in} γένη\FnParseFormGloss{c}{nominative neuter plural}{γένος}{family} φωνῶν ⸀εἰσιν\FnParse{d}{present active indicative 3rd plural} ἐν κόσμῳ, καὶ ⸀οὐδὲν ἄφωνον·\FnParseFormGloss{e}{nominative neuter singular}{ἄφωνος}{silent} 
\MainTextVerseMark{14}{11}ἐὰν οὖν μὴ εἰδῶ\FnParse{a}{perfect active subjunctive 1st singular} τὴν δύναμιν τῆς φωνῆς, ἔσομαι\FnParse{b}{future middle indicative 1st singular} τῷ λαλοῦντι\FnParse{c}{present active participle dative masculine singular} βάρβαρος\FnParseFormGloss{d}{nominative masculine singular}{βάρβαρος}{non-Greek} καὶ ὁ λαλῶν\FnParse{e}{present active participle nominative masculine singular} ἐν ἐμοὶ βάρβαρος.\FnParseFormGloss{f}{nominative masculine singular}{βάρβαρος}{non-Greek} 
\MainTextVerseMark{14}{12}οὕτως καὶ ὑμεῖς, ἐπεὶ\FnFormGloss{a}{ἐπεί}{since} ζηλωταί\FnParseFormGloss{b}{nominative masculine plural}{ζηλωτής}{zealot} ἐστε\FnParse{c}{present active indicative 2nd plural} πνευμάτων, πρὸς τὴν οἰκοδομὴν\FnParseFormGloss{d}{accusative feminine singular}{οἰκοδομή}{building} τῆς ἐκκλησίας ζητεῖτε\FnParse{e}{present active imperative 2nd plural} ἵνα περισσεύητε.\FnParse{f}{present active subjunctive 2nd plural} 
\MainTextVerseMark{14}{13}⸀Διὸ ὁ λαλῶν\FnParse{a}{present active participle nominative masculine singular} γλώσσῃ προσευχέσθω\FnParse{b}{present middle imperative 3rd singular} ἵνα διερμηνεύῃ.\FnParseFormGloss{c}{present active subjunctive 3rd singular}{διερμηνεύω}{to interpret} 
\MainTextVerseMark{14}{14}ἐὰν γὰρ προσεύχωμαι\FnParse{a}{present middle subjunctive 1st singular} γλώσσῃ, τὸ πνεῦμά μου προσεύχεται,\FnParse{b}{present middle indicative 3rd singular} ὁ δὲ νοῦς\FnParseFormGloss{c}{nominative masculine singular}{νοῦς}{mind} μου ἄκαρπός\FnParseFormGloss{d}{nominative masculine singular}{ἄκαρπος}{unfruitful} ἐστιν.\FnParse{e}{present active indicative 3rd singular} 
\MainTextVerseMark{14}{15}τί οὖν ἐστιν;\FnParse{a}{present active indicative 3rd singular} προσεύξομαι\FnParse{b}{future middle indicative 1st singular} τῷ πνεύματι, προσεύξομαι\FnParse{c}{future middle indicative 1st singular} δὲ καὶ τῷ νοΐ·\FnParseFormGloss{d}{dative masculine singular}{νοῦς}{mind} ψαλῶ\FnParseFormGloss{e}{future active indicative 1st singular}{ψάλλω}{to sing hymns} τῷ πνεύματι, ψαλῶ\FnParseFormGloss{f}{future active indicative 1st singular}{ψάλλω}{to sing hymns} δὲ καὶ τῷ νοΐ·\FnParseFormGloss{g}{dative masculine singular}{νοῦς}{mind} 
\MainTextVerseMark{14}{16}ἐπεὶ\FnFormGloss{a}{ἐπεί}{since} ἐὰν ⸀εὐλογῇς\FnParse{b}{present active subjunctive 2nd singular} ⸀πνεύματι, ὁ ἀναπληρῶν\FnParseFormGloss{c}{present active participle nominative masculine singular}{ἀναπληρόω}{to fulfill} τὸν τόπον τοῦ ἰδιώτου\FnParseFormGloss{d}{genitive masculine singular}{ἰδιώτης}{ordinary} πῶς ἐρεῖ\FnParse{e}{future active indicative 3rd singular} τὸ Ἀμήν ἐπὶ τῇ σῇ\FnParseFormGloss{f}{dative feminine singular}{σός}{your} εὐχαριστίᾳ;\FnParseFormGloss{g}{dative feminine singular}{εὐχαριστία}{expression of thanks} ἐπειδὴ\FnFormGloss{h}{ἐπειδή}{when} τί λέγεις\FnParse{i}{present active indicative 2nd singular} οὐκ οἶδεν·\FnParse{j}{perfect active indicative 3rd singular} 
\MainTextVerseMark{14}{17}σὺ μὲν γὰρ καλῶς εὐχαριστεῖς,\FnParse{a}{present active indicative 2nd singular} ἀλλ’ ὁ ἕτερος οὐκ οἰκοδομεῖται.\FnParse{b}{present passive indicative 3rd singular} 
\MainTextVerseMark{14}{18}εὐχαριστῶ\FnParse{a}{present active indicative 1st singular} τῷ ⸀θεῷ, πάντων ὑμῶν μᾶλλον ⸀γλώσσαις ⸀λαλῶ·\FnParse{b}{present active indicative 1st singular} 
\MainTextVerseMark{14}{19}ἀλλὰ ἐν ἐκκλησίᾳ θέλω\FnParse{a}{present active indicative 1st singular} πέντε λόγους ⸂τῷ νοΐ⸃\FnParseFormGloss{b}{dative masculine singular}{νοῦς}{mind} μου λαλῆσαι,\FnParse{c}{aorist active infinitive} ἵνα καὶ ἄλλους κατηχήσω,\FnParseFormGloss{d}{aorist active subjunctive 1st singular}{κατηχέω}{to instruct} ἢ μυρίους\FnParseFormGloss{e}{accusative masculine plural}{μύριοι}{ten thousand} λόγους ἐν γλώσσῃ. 
\MainTextVerseMark{14}{20}Ἀδελφοί, μὴ παιδία γίνεσθε\FnParse{a}{present middle imperative 2nd plural} ταῖς φρεσίν,\FnParseFormGloss{b}{dative feminine plural}{φρήν}{thinking} ἀλλὰ τῇ κακίᾳ\FnParseFormGloss{c}{dative feminine singular}{κακία}{evil} νηπιάζετε,\FnParseFormGloss{d}{present active imperative 2nd plural}{νηπιάζω}{to be a child} ταῖς δὲ φρεσὶν\FnParseFormGloss{e}{dative feminine plural}{φρήν}{thinking} τέλειοι\FnParseFormGloss{f}{nominative masculine plural}{τέλειος}{perfect} γίνεσθε.\FnParse{g}{present middle imperative 2nd plural} 
\MainTextVerseMark{14}{21}ἐν τῷ νόμῳ γέγραπται\FnParse{a}{perfect passive indicative 3rd singular} ὅτι Ἐν ἑτερογλώσσοις\FnParseFormGloss{b}{dative masculine plural}{ἑτερόγλωσσος}{speaking in a foreign language} καὶ ἐν χείλεσιν\FnParseFormGloss{c}{dative neuter plural}{χεῖλος}{lip} ⸀ἑτέρων λαλήσω\FnParse{d}{future active indicative 1st singular} τῷ λαῷ τούτῳ, καὶ οὐδ’ οὕτως εἰσακούσονταί\FnParseFormGloss{e}{future middle indicative 3rd plural}{εἰσακούω}{to be heard} μου, λέγει\FnParse{f}{present active indicative 3rd singular} κύριος. 
\MainTextVerseMark{14}{22}ὥστε αἱ γλῶσσαι εἰς σημεῖόν εἰσιν\FnParse{a}{present active indicative 3rd plural} οὐ τοῖς πιστεύουσιν\FnParse{b}{present active participle dative masculine plural} ἀλλὰ τοῖς ἀπίστοις,\FnParseFormGloss{c}{dative masculine plural}{ἄπιστος}{unbelieving} ἡ δὲ προφητεία\FnParseFormGloss{d}{nominative feminine singular}{προφητεία}{prophecy} οὐ τοῖς ἀπίστοις\FnParseFormGloss{e}{dative masculine plural}{ἄπιστος}{unbelieving} ἀλλὰ τοῖς πιστεύουσιν.\FnParse{f}{present active participle dative masculine plural} 
\MainTextVerseMark{14}{23}ἐὰν οὖν συνέλθῃ\FnParse{a}{aorist active subjunctive 3rd singular} ἡ ἐκκλησία ὅλη ἐπὶ τὸ αὐτὸ καὶ πάντες ⸂λαλῶσιν\FnParse{b}{present active subjunctive 3rd plural} γλώσσαις⸃, εἰσέλθωσιν\FnParse{c}{aorist active subjunctive 3rd plural} δὲ ἰδιῶται\FnParseFormGloss{d}{nominative masculine plural}{ἰδιώτης}{ordinary} ἢ ἄπιστοι,\FnParseFormGloss{e}{nominative masculine plural}{ἄπιστος}{unbelieving} οὐκ ἐροῦσιν\FnParse{f}{future active indicative 3rd plural} ὅτι μαίνεσθε;\FnParseFormGloss{g}{present middle indicative 2nd plural}{μαίνομαι}{to rave} 
\MainTextVerseMark{14}{24}ἐὰν δὲ πάντες προφητεύωσιν,\FnParseFormGloss{a}{present active subjunctive 3rd plural}{προφητεύω}{to prophesy} εἰσέλθῃ\FnParse{b}{aorist active subjunctive 3rd singular} δέ τις ἄπιστος\FnParseFormGloss{c}{nominative masculine singular}{ἄπιστος}{unbelieving} ἢ ἰδιώτης,\FnParseFormGloss{d}{nominative masculine singular}{ἰδιώτης}{ordinary} ἐλέγχεται\FnParseFormGloss{e}{present passive indicative 3rd singular}{ἐλέγχω}{to expose} ὑπὸ πάντων, ἀνακρίνεται\FnParseFormGloss{f}{present passive indicative 3rd singular}{ἀνακρίνω}{to examine} ὑπὸ πάντων, 
\MainTextVerseMark{14}{25}⸀τὰ κρυπτὰ\FnParseFormGloss{a}{nominative neuter plural}{κρυπτός}{hidden} τῆς καρδίας αὐτοῦ φανερὰ\FnParseFormGloss{b}{nominative neuter plural}{φανερός}{visible} γίνεται,\FnParse{c}{present middle indicative 3rd singular} καὶ οὕτως πεσὼν\FnParse{d}{aorist active participle nominative masculine singular} ἐπὶ πρόσωπον προσκυνήσει\FnParse{e}{future active indicative 3rd singular} τῷ θεῷ, ἀπαγγέλλων\FnParse{f}{present active participle nominative masculine singular} ὅτι ⸂Ὄντως\FnFormGloss{g}{ὄντως}{really} ὁ θεὸς⸃ ἐν ὑμῖν ἐστιν.\FnParse{h}{present active indicative 3rd singular} 
\MainTextVerseMark{14}{26}Τί οὖν ἐστιν,\FnParse{a}{present active indicative 3rd singular} ἀδελφοί; ὅταν συνέρχησθε,\FnParse{b}{present middle subjunctive 2nd plural} ⸀ἕκαστος ψαλμὸν\FnParseFormGloss{c}{accusative masculine singular}{ψαλμός}{Psalms} ἔχει,\FnParse{d}{present active indicative 3rd singular} διδαχὴν ἔχει,\FnParse{e}{present active indicative 3rd singular} ⸂ἀποκάλυψιν\FnParseFormGloss{f}{accusative feminine singular}{ἀποκάλυψις}{revelation} ἔχει,\FnParse{g}{present active indicative 3rd singular} γλῶσσαν⸃ ἔχει,\FnParse{h}{present active indicative 3rd singular} ἑρμηνείαν\FnParseFormGloss{i}{accusative feminine singular}{ἑρμηνεία}{interpretation} ἔχει·\FnParse{j}{present active indicative 3rd singular} πάντα πρὸς οἰκοδομὴν\FnParseFormGloss{k}{accusative feminine singular}{οἰκοδομή}{building} γινέσθω.\FnParse{l}{present middle imperative 3rd singular} 
\MainTextVerseMark{14}{27}εἴτε γλώσσῃ τις λαλεῖ,\FnParse{a}{present active indicative 3rd singular} κατὰ δύο ἢ τὸ πλεῖστον τρεῖς, καὶ ἀνὰ\FnFormGloss{b}{ἀνά}{each} μέρος, καὶ εἷς διερμηνευέτω·\FnParseFormGloss{c}{present active imperative 3rd singular}{διερμηνεύω}{to interpret} 
\MainTextVerseMark{14}{28}ἐὰν δὲ μὴ ᾖ\FnParse{a}{present active subjunctive 3rd singular} ⸀διερμηνευτής,\FnParseFormGloss{b}{nominative masculine singular}{διερμηνευτής}{interpreter} σιγάτω\FnParseFormGloss{c}{present active imperative 3rd singular}{σιγάω}{to be silent} ἐν ἐκκλησίᾳ, ἑαυτῷ δὲ λαλείτω\FnParse{d}{present active imperative 3rd singular} καὶ τῷ θεῷ. 
\MainTextVerseMark{14}{29}προφῆται δὲ δύο ἢ τρεῖς λαλείτωσαν,\FnParse{a}{present active imperative 3rd plural} καὶ οἱ ἄλλοι διακρινέτωσαν·\FnParseFormGloss{b}{present active imperative 3rd plural}{διακρίνω}{to make a distinction} 
\MainTextVerseMark{14}{30}ἐὰν δὲ ἄλλῳ ἀποκαλυφθῇ\FnParseFormGloss{a}{aorist passive subjunctive 3rd singular}{ἀποκαλύπτω}{to reveal} καθημένῳ,\FnParse{b}{present middle participle dative masculine singular} ὁ πρῶτος σιγάτω.\FnParseFormGloss{c}{present active imperative 3rd singular}{σιγάω}{to be silent} 
\MainTextVerseMark{14}{31}δύνασθε\FnParse{a}{present middle indicative 2nd plural} γὰρ καθ’ ἕνα πάντες προφητεύειν,\FnParseFormGloss{b}{present active infinitive}{προφητεύω}{to prophesy} ἵνα πάντες μανθάνωσιν\FnParseFormGloss{c}{present active subjunctive 3rd plural}{μανθάνω}{to learn} καὶ πάντες παρακαλῶνται\FnParse{d}{present passive subjunctive 3rd plural} 
\MainTextVerseMark{14}{32}(καὶ πνεύματα προφητῶν προφήταις ὑποτάσσεται,\FnParse{a}{present passive indicative 3rd singular} 
\MainTextVerseMark{14}{33}οὐ γάρ ἐστιν\FnParse{a}{present active indicative 3rd singular} ἀκαταστασίας\FnParseFormGloss{b}{genitive feminine singular}{ἀκαταστασία}{disorder} ὁ θεὸς ἀλλὰ εἰρήνης), ὡς ἐν πάσαις ταῖς ἐκκλησίαις τῶν ἁγίων. 
\MainTextVerseMark{14}{34}Αἱ ⸀γυναῖκες ἐν ταῖς ἐκκλησίαις σιγάτωσαν,\FnParseFormGloss{a}{present active imperative 3rd plural}{σιγάω}{to be silent} οὐ γὰρ ⸀ἐπιτρέπεται\FnParseFormGloss{b}{present passive indicative 3rd singular}{ἐπιτρέπω}{to let} αὐταῖς λαλεῖν·\FnParse{c}{present active infinitive} ἀλλὰ ⸀ὑποτασσέσθωσαν,\FnParse{d}{present passive imperative 3rd plural} καθὼς καὶ ὁ νόμος λέγει.\FnParse{e}{present active indicative 3rd singular} 
\MainTextVerseMark{14}{35}εἰ δέ τι ⸀μαθεῖν\FnParseFormGloss{a}{aorist active infinitive}{μανθάνω}{to learn} θέλουσιν,\FnParse{b}{present active indicative 3rd plural} ἐν οἴκῳ τοὺς ἰδίους ἄνδρας ἐπερωτάτωσαν,\FnParse{c}{present active imperative 3rd plural} αἰσχρὸν\FnParseFormGloss{d}{nominative neuter singular}{αἰσχρός}{disgraceful} γάρ ἐστιν\FnParse{e}{present active indicative 3rd singular} γυναικὶ ⸂λαλεῖν\FnParse{f}{present active infinitive} ἐν ἐκκλησίᾳ⸃. 
\MainTextVerseMark{14}{36}ἢ ἀφ’ ὑμῶν ὁ λόγος τοῦ θεοῦ ἐξῆλθεν,\FnParse{a}{aorist active indicative 3rd singular} ἢ εἰς ὑμᾶς μόνους κατήντησεν;\FnParseFormGloss{b}{aorist active indicative 3rd singular}{καταντάω}{to come to} 
\MainTextVerseMark{14}{37}Εἴ τις δοκεῖ\FnParse{a}{present active indicative 3rd singular} προφήτης εἶναι\FnParse{b}{present active infinitive} ἢ πνευματικός,\FnParseFormGloss{c}{nominative masculine singular}{πνευματικός}{spiritual} ἐπιγινωσκέτω\FnParse{d}{present active imperative 3rd singular} ἃ γράφω\FnParse{e}{present active indicative 1st singular} ὑμῖν ὅτι κυρίου ⸀ἐστὶν·\FnParse{f}{present active indicative 3rd singular} 
\MainTextVerseMark{14}{38}εἰ δέ τις ἀγνοεῖ,\FnParseFormGloss{a}{present active indicative 3rd singular}{ἀγνοέω}{to be ignorant} ⸀ἀγνοεῖται.\FnParseFormGloss{b}{present passive indicative 3rd singular}{ἀγνοέω}{to be ignorant} 
\MainTextVerseMark{14}{39}ὥστε, ἀδελφοί ⸀μου, ζηλοῦτε\FnParseFormGloss{a}{present active imperative 2nd plural}{ζηλόω}{to desire} τὸ προφητεύειν,\FnParseFormGloss{b}{present active infinitive}{προφητεύω}{to prophesy} καὶ τὸ λαλεῖν\FnParse{c}{present active infinitive} ⸂μὴ κωλύετε\FnParseFormGloss{d}{present active imperative 2nd plural}{κωλύω}{to hinder} γλώσσαις⸃· 
\MainTextVerseMark{14}{40}πάντα ⸀δὲ εὐσχημόνως\FnFormGloss{a}{εὐσχημόνως}{decently} καὶ κατὰ τάξιν\FnParseFormGloss{b}{accusative feminine singular}{τάξις}{order} γινέσθω.\FnParse{c}{present middle imperative 3rd singular} 
\ChapterHeader{15}{Chapter 15}

\MainTextVerseMark{15}{1}Γνωρίζω\FnParseFormGloss{a}{present active indicative 1st singular}{γνωρίζω}{to make known} δὲ ὑμῖν, ἀδελφοί, τὸ εὐαγγέλιον ὃ εὐηγγελισάμην\FnParse{b}{aorist middle indicative 1st singular} ὑμῖν, ὃ καὶ παρελάβετε,\FnParse{c}{aorist active indicative 2nd plural} ἐν ᾧ καὶ ἑστήκατε,\FnParse{d}{perfect active indicative 2nd plural} 
\MainTextVerseMark{15}{2}δι’ οὗ καὶ σῴζεσθε,\FnParse{a}{present passive indicative 2nd plural} τίνι λόγῳ εὐηγγελισάμην\FnParse{b}{aorist middle indicative 1st singular} ὑμῖν, εἰ κατέχετε,\FnParseFormGloss{c}{present active indicative 2nd plural}{κατέχω}{to hold back} ἐκτὸς\FnFormGloss{d}{ἐκτός}{outside} εἰ μὴ εἰκῇ\FnFormGloss{e}{εἰκῇ}{in vain} ἐπιστεύσατε.\FnParse{f}{aorist active indicative 2nd plural} 
\MainTextVerseMark{15}{3}Παρέδωκα\FnParse{a}{aorist active indicative 1st singular} γὰρ ὑμῖν ἐν πρώτοις, ὃ καὶ παρέλαβον,\FnParse{b}{aorist active indicative 1st singular} ὅτι Χριστὸς ἀπέθανεν\FnParse{c}{aorist active indicative 3rd singular} ὑπὲρ τῶν ἁμαρτιῶν ἡμῶν κατὰ τὰς γραφάς, 
\MainTextVerseMark{15}{4}καὶ ὅτι ἐτάφη,\FnParseFormGloss{a}{aorist passive indicative 3rd singular}{θάπτω}{to bury} καὶ ὅτι ἐγήγερται\FnParse{b}{perfect passive indicative 3rd singular} τῇ ⸂ἡμέρᾳ τῇ τρίτῃ⸃ κατὰ τὰς γραφάς, 
\MainTextVerseMark{15}{5}καὶ ὅτι ὤφθη\FnParse{a}{aorist passive indicative 3rd singular} Κηφᾷ,\FnParseFormGloss{b}{dative masculine singular}{Κηφᾶς}{Cephas} εἶτα\FnFormGloss{c}{εἶτα}{then} τοῖς δώδεκα· 
\MainTextVerseMark{15}{6}ἔπειτα\FnFormGloss{a}{ἔπειτα}{then} ὤφθη\FnParse{b}{aorist passive indicative 3rd singular} ἐπάνω\FnFormGloss{c}{ἐπάνω}{above} πεντακοσίοις\FnParseFormGloss{d}{dative masculine plural}{πεντακόσιοι}{five hundred} ἀδελφοῖς ἐφάπαξ,\FnFormGloss{e}{ἐφάπαξ}{once for all} ἐξ ὧν οἱ ⸀πλείονες μένουσιν\FnParse{f}{present active indicative 3rd plural} ἕως ἄρτι, τινὲς ⸀δὲ ἐκοιμήθησαν·\FnParseFormGloss{g}{aorist passive indicative 3rd plural}{κοιμάομαι}{to fall asleep} 
\MainTextVerseMark{15}{7}ἔπειτα\FnFormGloss{a}{ἔπειτα}{then} ὤφθη\FnParse{b}{aorist passive indicative 3rd singular} Ἰακώβῳ, εἶτα\FnFormGloss{c}{εἶτα}{then} τοῖς ἀποστόλοις πᾶσιν· 
\MainTextVerseMark{15}{8}ἔσχατον δὲ πάντων ὡσπερεὶ\FnFormGloss{a}{ὡσπερεί}{like} τῷ ἐκτρώματι\FnParseFormGloss{b}{dative neuter singular}{ἔκτρωμα}{an abortion} ὤφθη\FnParse{c}{aorist passive indicative 3rd singular} κἀμοί. 
\MainTextVerseMark{15}{9}ἐγὼ γάρ εἰμι\FnParse{a}{present active indicative 1st singular} ὁ ἐλάχιστος\FnParseFormGloss{b}{nominative masculine singular}{ἐλάχιστος}{least} τῶν ἀποστόλων, ὃς οὐκ εἰμὶ\FnParse{c}{present active indicative 1st singular} ἱκανὸς καλεῖσθαι\FnParse{d}{present passive infinitive} ἀπόστολος, διότι\FnFormGloss{e}{διότι}{therefore} ἐδίωξα\FnParse{f}{aorist active indicative 1st singular} τὴν ἐκκλησίαν τοῦ θεοῦ· 
\MainTextVerseMark{15}{10}χάριτι δὲ θεοῦ εἰμι\FnParse{a}{present active indicative 1st singular} ὅ εἰμι,\FnParse{b}{present active indicative 1st singular} καὶ ἡ χάρις αὐτοῦ ἡ εἰς ἐμὲ οὐ κενὴ\FnParseFormGloss{c}{nominative feminine singular}{κενός}{empty} ἐγενήθη,\FnParse{d}{aorist passive indicative 3rd singular} ἀλλὰ περισσότερον\FnParseFormGloss{e}{accusative neuter singular}{περισσότερος}{more than} αὐτῶν πάντων ἐκοπίασα,\FnParseFormGloss{f}{aorist active indicative 1st singular}{κοπιάω}{to work} οὐκ ἐγὼ δὲ ἀλλὰ ἡ χάρις τοῦ θεοῦ ⸀ἡ σὺν ἐμοί. 
\MainTextVerseMark{15}{11}εἴτε οὖν ἐγὼ εἴτε ἐκεῖνοι, οὕτως κηρύσσομεν\FnParse{a}{present active indicative 1st plural} καὶ οὕτως ἐπιστεύσατε.\FnParse{b}{aorist active indicative 2nd plural} 
\MainTextVerseMark{15}{12}Εἰ δὲ Χριστὸς κηρύσσεται\FnParse{a}{present passive indicative 3rd singular} ὅτι ἐκ νεκρῶν ἐγήγερται,\FnParse{b}{perfect passive indicative 3rd singular} πῶς λέγουσιν\FnParse{c}{present active indicative 3rd plural} ⸂ἐν ὑμῖν τινες⸃ ὅτι ἀνάστασις νεκρῶν οὐκ ἔστιν;\FnParse{d}{present active indicative 3rd singular} 
\MainTextVerseMark{15}{13}εἰ δὲ ἀνάστασις νεκρῶν οὐκ ἔστιν,\FnParse{a}{present active indicative 3rd singular} οὐδὲ Χριστὸς ἐγήγερται·\FnParse{b}{perfect passive indicative 3rd singular} 
\MainTextVerseMark{15}{14}εἰ δὲ Χριστὸς οὐκ ἐγήγερται,\FnParse{a}{perfect passive indicative 3rd singular} κενὸν\FnParseFormGloss{b}{nominative neuter singular}{κενός}{empty} ⸀ἄρα τὸ κήρυγμα\FnParseFormGloss{c}{nominative neuter singular}{κήρυγμα}{preaching} ἡμῶν, ⸀κενὴ\FnParseFormGloss{d}{nominative feminine singular}{κενός}{empty} καὶ ἡ πίστις ⸀ὑμῶν, 
\MainTextVerseMark{15}{15}εὑρισκόμεθα\FnParse{a}{present passive indicative 1st plural} δὲ καὶ ψευδομάρτυρες\FnParseFormGloss{b}{nominative masculine plural}{ψευδόμαρτυς}{false witness} τοῦ θεοῦ, ὅτι ἐμαρτυρήσαμεν\FnParse{c}{aorist active indicative 1st plural} κατὰ τοῦ θεοῦ ὅτι ἤγειρεν\FnParse{d}{aorist active indicative 3rd singular} τὸν Χριστόν, ὃν οὐκ ἤγειρεν\FnParse{e}{aorist active indicative 3rd singular} εἴπερ\FnFormGloss{f}{εἴπερ}{if indeed} ἄρα νεκροὶ οὐκ ἐγείρονται.\FnParse{g}{present passive indicative 3rd plural} 
\MainTextVerseMark{15}{16}εἰ γὰρ νεκροὶ οὐκ ἐγείρονται,\FnParse{a}{present passive indicative 3rd plural} οὐδὲ Χριστὸς ἐγήγερται·\FnParse{b}{perfect passive indicative 3rd singular} 
\MainTextVerseMark{15}{17}εἰ δὲ Χριστὸς οὐκ ἐγήγερται,\FnParse{a}{perfect passive indicative 3rd singular} ματαία\FnParseFormGloss{b}{nominative feminine singular}{μάταιος}{worthless} ἡ πίστις ⸀ὑμῶν, ἔτι ἐστὲ\FnParse{c}{present active indicative 2nd plural} ἐν ταῖς ἁμαρτίαις ὑμῶν. 
\MainTextVerseMark{15}{18}ἄρα καὶ οἱ κοιμηθέντες\FnParseFormGloss{a}{aorist passive participle nominative masculine plural}{κοιμάομαι}{to fall asleep} ἐν Χριστῷ ἀπώλοντο.\FnParse{b}{aorist middle indicative 3rd plural} 
\MainTextVerseMark{15}{19}εἰ ἐν τῇ ζωῇ ταύτῃ ⸂ἐν Χριστῷ ἠλπικότες\FnParse{a}{perfect active participle nominative masculine plural} ἐσμὲν⸃\FnParse{b}{present active indicative 1st plural} μόνον, ἐλεεινότεροι\FnParseFormGloss{c}{nominative masculine plural}{ἐλεεινός}{pitiful} πάντων ἀνθρώπων ἐσμέν.\FnParse{d}{present active indicative 1st plural} 
\MainTextVerseMark{15}{20}Νυνὶ\FnFormGloss{a}{νυνί}{now} δὲ Χριστὸς ἐγήγερται\FnParse{b}{perfect passive indicative 3rd singular} ἐκ νεκρῶν, ἀπαρχὴ\FnParseFormGloss{c}{nominative feminine singular}{ἀπαρχή}{firstfruits} τῶν ⸀κεκοιμημένων.\FnParseFormGloss{d}{perfect middle participle genitive masculine plural}{κοιμάομαι}{to fall asleep} 
\MainTextVerseMark{15}{21}ἐπειδὴ\FnFormGloss{a}{ἐπειδή}{when} γὰρ δι’ ἀνθρώπου ⸀θάνατος, καὶ δι’ ἀνθρώπου ἀνάστασις νεκρῶν· 
\MainTextVerseMark{15}{22}ὥσπερ γὰρ ἐν τῷ Ἀδὰμ\FnParseFormGloss{a}{dative masculine singular}{Ἀδάμ}{Adam} πάντες ἀποθνῄσκουσιν,\FnParse{b}{present active indicative 3rd plural} οὕτως καὶ ἐν τῷ Χριστῷ πάντες ζῳοποιηθήσονται.\FnParseFormGloss{c}{future passive indicative 3rd plural}{ζῳοποιέω}{to make alive} 
\MainTextVerseMark{15}{23}ἕκαστος δὲ ἐν τῷ ἰδίῳ τάγματι·\FnParseFormGloss{a}{dative neuter singular}{τάγμα}{turn} ἀπαρχὴ\FnParseFormGloss{b}{nominative feminine singular}{ἀπαρχή}{firstfruits} Χριστός, ἔπειτα\FnFormGloss{c}{ἔπειτα}{then} οἱ τοῦ Χριστοῦ ἐν τῇ παρουσίᾳ\FnParseFormGloss{d}{dative feminine singular}{παρουσία}{presence} αὐτοῦ· 
\MainTextVerseMark{15}{24}εἶτα\FnFormGloss{a}{εἶτα}{then} τὸ τέλος, ὅταν ⸀παραδιδῷ\FnParse{b}{present active subjunctive 3rd singular} τὴν βασιλείαν τῷ θεῷ καὶ πατρί, ὅταν καταργήσῃ\FnParseFormGloss{c}{aorist active subjunctive 3rd singular}{καταργέω}{to nullify} πᾶσαν ἀρχὴν καὶ πᾶσαν ἐξουσίαν καὶ δύναμιν, 
\MainTextVerseMark{15}{25}δεῖ\FnParse{a}{present active indicative 3rd singular} γὰρ αὐτὸν βασιλεύειν\FnParseFormGloss{b}{present active infinitive}{βασιλεύω}{to reign as a king} ἄχρι ⸀οὗ θῇ\FnParse{c}{aorist active subjunctive 3rd singular} πάντας τοὺς ἐχθροὺς ὑπὸ τοὺς πόδας αὐτοῦ. 
\MainTextVerseMark{15}{26}ἔσχατος ἐχθρὸς καταργεῖται\FnParseFormGloss{a}{present passive indicative 3rd singular}{καταργέω}{to nullify} ὁ θάνατος, 
\MainTextVerseMark{15}{27}πάντα γὰρ ὑπέταξεν\FnParse{a}{aorist active indicative 3rd singular} ὑπὸ τοὺς πόδας αὐτοῦ. ὅταν δὲ εἴπῃ\FnParse{b}{aorist active subjunctive 3rd singular} ὅτι πάντα ὑποτέτακται,\FnParse{c}{perfect passive indicative 3rd singular} δῆλον\FnParseFormGloss{d}{nominative neuter singular}{δῆλος}{clear} ὅτι ἐκτὸς\FnFormGloss{e}{ἐκτός}{outside} τοῦ ὑποτάξαντος\FnParse{f}{aorist active participle genitive masculine singular} αὐτῷ τὰ πάντα. 
\MainTextVerseMark{15}{28}ὅταν δὲ ὑποταγῇ\FnParse{a}{aorist passive subjunctive 3rd singular} αὐτῷ τὰ πάντα, ⸀τότε αὐτὸς ὁ υἱὸς ὑποταγήσεται\FnParse{b}{future passive indicative 3rd singular} τῷ ὑποτάξαντι\FnParse{c}{aorist active participle dative masculine singular} αὐτῷ τὰ πάντα, ἵνα ᾖ\FnParse{d}{present active subjunctive 3rd singular} ὁ θεὸς ⸀πάντα ἐν πᾶσιν. 
\MainTextVerseMark{15}{29}Ἐπεὶ\FnFormGloss{a}{ἐπεί}{since} τί ποιήσουσιν\FnParse{b}{future active indicative 3rd plural} οἱ βαπτιζόμενοι\FnParse{c}{present passive participle nominative masculine plural} ὑπὲρ τῶν νεκρῶν; εἰ ὅλως\FnFormGloss{d}{ὅλως}{wholly} νεκροὶ οὐκ ἐγείρονται,\FnParse{e}{present passive indicative 3rd plural} τί καὶ βαπτίζονται\FnParse{f}{present passive indicative 3rd plural} ὑπὲρ ⸀αὐτῶν; 
\MainTextVerseMark{15}{30}τί καὶ ἡμεῖς κινδυνεύομεν\FnParseFormGloss{a}{present active indicative 1st plural}{κινδυνεύω}{to be in danger} πᾶσαν ὥραν; 
\MainTextVerseMark{15}{31}καθ’ ἡμέραν ἀποθνῄσκω,\FnParse{a}{present active indicative 1st singular} νὴ\FnFormGloss{b}{νή}{as surely as} τὴν ὑμετέραν\FnParseFormGloss{c}{accusative feminine singular}{ὑμέτερος}{your} ⸀καύχησιν,\FnParseFormGloss{d}{accusative feminine singular}{καύχησις}{boasting} ἣν ἔχω\FnParse{e}{present active indicative 1st singular} ἐν Χριστῷ Ἰησοῦ τῷ κυρίῳ ἡμῶν. 
\MainTextVerseMark{15}{32}εἰ κατὰ ἄνθρωπον ἐθηριομάχησα\FnParseFormGloss{a}{aorist active indicative 1st singular}{θηριομαχέω}{to fight wild animals} ἐν Ἐφέσῳ,\FnParseFormGloss{b}{dative feminine singular}{Ἔφεσος}{Ephesus} τί μοι τὸ ὄφελος;\FnParseFormGloss{c}{nominative neuter singular}{ὄφελος}{good} εἰ νεκροὶ οὐκ ἐγείρονται,\FnParse{d}{present passive indicative 3rd plural} Φάγωμεν\FnParse{e}{aorist active subjunctive 1st plural} καὶ πίωμεν,\FnParse{f}{aorist active subjunctive 1st plural} αὔριον\FnFormGloss{g}{αὔριον}{tomorrow} γὰρ ἀποθνῄσκομεν.\FnParse{h}{present active indicative 1st plural} 
\MainTextVerseMark{15}{33}μὴ πλανᾶσθε·\FnParse{a}{present passive imperative 2nd plural} φθείρουσιν\FnParseFormGloss{b}{present active indicative 3rd plural}{φθείρω}{to destroy} ἤθη\FnParseFormGloss{c}{accusative neuter plural}{ἦθος}{a place of customary resort;} χρηστὰ\FnParseFormGloss{d}{accusative neuter plural}{χρηστός}{easy} ὁμιλίαι\FnParseFormGloss{e}{nominative feminine plural}{ὁμιλία}{company} κακαί. 
\MainTextVerseMark{15}{34}ἐκνήψατε\FnParseFormGloss{a}{aorist active imperative 2nd plural}{ἐκνήφω}{to come to one's sense} δικαίως\FnFormGloss{b}{δικαίως}{justly} καὶ μὴ ἁμαρτάνετε,\FnParse{c}{present active imperative 2nd plural} ἀγνωσίαν\FnParseFormGloss{d}{accusative feminine singular}{ἀγνωσία}{ignorance} γὰρ θεοῦ τινες ἔχουσιν·\FnParse{e}{present active indicative 3rd plural} πρὸς ἐντροπὴν\FnParseFormGloss{f}{accusative feminine singular}{ἐντροπή}{shame} ὑμῖν ⸀λαλῶ.\FnParse{g}{present active indicative 1st singular} 
\MainTextVerseMark{15}{35}Ἀλλὰ ἐρεῖ\FnParse{a}{future active indicative 3rd singular} τις· Πῶς ἐγείρονται\FnParse{b}{present passive indicative 3rd plural} οἱ νεκροί, ποίῳ δὲ σώματι ἔρχονται;\FnParse{c}{present middle indicative 3rd plural} 
\MainTextVerseMark{15}{36}⸀ἄφρων,\FnParseFormGloss{a}{masculine singular}{ἄφρων}{foolish} σὺ ὃ σπείρεις,\FnParse{b}{present active indicative 2nd singular} οὐ ζῳοποιεῖται\FnParseFormGloss{c}{present passive indicative 3rd singular}{ζῳοποιέω}{to make alive} ἐὰν μὴ ἀποθάνῃ·\FnParse{d}{aorist active subjunctive 3rd singular} 
\MainTextVerseMark{15}{37}καὶ ὃ σπείρεις,\FnParse{a}{present active indicative 2nd singular} οὐ τὸ σῶμα τὸ γενησόμενον\FnParse{b}{future middle participle accusative neuter singular} σπείρεις\FnParse{c}{present active indicative 2nd singular} ἀλλὰ γυμνὸν\FnParseFormGloss{d}{accusative masculine singular}{γυμνός}{naked} κόκκον\FnParseFormGloss{e}{accusative masculine singular}{κόκκος}{seed} εἰ τύχοι\FnParseFormGloss{f}{aorist active optative 3rd singular}{τυγχάνω}{to take part in} σίτου\FnParseFormGloss{g}{genitive masculine singular}{σῖτος}{wheat} ἤ τινος τῶν λοιπῶν· 
\MainTextVerseMark{15}{38}ὁ δὲ θεὸς ⸂δίδωσιν\FnParse{a}{present active indicative 3rd singular} αὐτῷ⸃ σῶμα καθὼς ἠθέλησεν,\FnParse{b}{aorist active indicative 3rd singular} καὶ ἑκάστῳ τῶν σπερμάτων ⸀ἴδιον σῶμα. 
\MainTextVerseMark{15}{39}οὐ πᾶσα σὰρξ ἡ αὐτὴ σάρξ, ἀλλὰ ἄλλη μὲν ἀνθρώπων, ἄλλη δὲ σὰρξ κτηνῶν,\FnParseFormGloss{a}{genitive neuter plural}{κτῆνος}{donkey} ἄλλη δὲ ⸂σὰρξ πτηνῶν⸃,\FnParseFormGloss{b}{genitive neuter plural}{πτηνόν}{winged, with feathers} ἄλλη δὲ ⸀ἰχθύων.\FnParseFormGloss{c}{genitive masculine plural}{ἰχθύς}{fish} 
\MainTextVerseMark{15}{40}καὶ σώματα ἐπουράνια,\FnParseFormGloss{a}{nominative neuter plural}{ἐπουράνιος}{heavenly} καὶ σώματα ἐπίγεια·\FnParseFormGloss{b}{nominative neuter plural}{ἐπίγειος}{being on the earth} ἀλλὰ ἑτέρα μὲν ἡ τῶν ἐπουρανίων\FnParseFormGloss{c}{genitive neuter plural}{ἐπουράνιος}{heavenly} δόξα, ἑτέρα δὲ ἡ τῶν ἐπιγείων.\FnParseFormGloss{d}{genitive neuter plural}{ἐπίγειος}{being on the earth} 
\MainTextVerseMark{15}{41}ἄλλη δόξα ἡλίου, καὶ ἄλλη δόξα σελήνης,\FnParseFormGloss{a}{genitive feminine singular}{σελήνη}{moon} καὶ ἄλλη δόξα ἀστέρων,\FnParseFormGloss{b}{genitive masculine plural}{ἀστήρ}{star} ἀστὴρ\FnParseFormGloss{c}{nominative masculine singular}{ἀστήρ}{star} γὰρ ἀστέρος\FnParseFormGloss{d}{genitive masculine singular}{ἀστήρ}{star} διαφέρει\FnParseFormGloss{e}{present active indicative 3rd singular}{διαφέρω}{to carry} ἐν δόξῃ. 
\MainTextVerseMark{15}{42}Οὕτως καὶ ἡ ἀνάστασις τῶν νεκρῶν. σπείρεται\FnParse{a}{present passive indicative 3rd singular} ἐν φθορᾷ,\FnParseFormGloss{b}{dative feminine singular}{φθορά}{perishableness} ἐγείρεται\FnParse{c}{present passive indicative 3rd singular} ἐν ἀφθαρσίᾳ·\FnParseFormGloss{d}{dative feminine singular}{ἀφθαρσία}{imperishableness} 
\MainTextVerseMark{15}{43}σπείρεται\FnParse{a}{present passive indicative 3rd singular} ἐν ἀτιμίᾳ,\FnParseFormGloss{b}{dative feminine singular}{ἀτιμία}{dishonor} ἐγείρεται\FnParse{c}{present passive indicative 3rd singular} ἐν δόξῃ· σπείρεται\FnParse{d}{present passive indicative 3rd singular} ἐν ἀσθενείᾳ,\FnParseFormGloss{e}{dative feminine singular}{ἀσθένεια}{weakness} ἐγείρεται\FnParse{f}{present passive indicative 3rd singular} ἐν δυνάμει· 
\MainTextVerseMark{15}{44}σπείρεται\FnParse{a}{present passive indicative 3rd singular} σῶμα ψυχικόν,\FnParseFormGloss{b}{nominative neuter singular}{ψυχικός}{physical} ἐγείρεται\FnParse{c}{present passive indicative 3rd singular} σῶμα πνευματικόν.\FnParseFormGloss{d}{nominative neuter singular}{πνευματικός}{spiritual} ⸀Εἰ ἔστιν\FnParse{e}{present active indicative 3rd singular} σῶμα ψυχικόν,\FnParseFormGloss{f}{nominative neuter singular}{ψυχικός}{physical} ⸂ἔστιν\FnParse{g}{present active indicative 3rd singular} καὶ⸃ πνευματικόν.\FnParseFormGloss{h}{nominative neuter singular}{πνευματικός}{spiritual} 
\MainTextVerseMark{15}{45}οὕτως καὶ γέγραπται·\FnParse{a}{perfect passive indicative 3rd singular} Ἐγένετο\FnParse{b}{aorist middle indicative 3rd singular} ὁ πρῶτος ἄνθρωπος Ἀδὰμ\FnParseFormGloss{c}{nominative masculine singular}{Ἀδάμ}{Adam} εἰς ψυχὴν ζῶσαν·\FnParse{d}{present active participle accusative feminine singular} ὁ ἔσχατος Ἀδὰμ\FnParseFormGloss{e}{nominative masculine singular}{Ἀδάμ}{Adam} εἰς πνεῦμα ζῳοποιοῦν.\FnParseFormGloss{f}{present active participle accusative neuter singular}{ζῳοποιέω}{to make alive} 
\MainTextVerseMark{15}{46}ἀλλ’ οὐ πρῶτον τὸ πνευματικὸν\FnParseFormGloss{a}{nominative neuter singular}{πνευματικός}{spiritual} ἀλλὰ τὸ ψυχικόν,\FnParseFormGloss{b}{nominative neuter singular}{ψυχικός}{physical} ἔπειτα\FnFormGloss{c}{ἔπειτα}{then} τὸ πνευματικόν.\FnParseFormGloss{d}{nominative neuter singular}{πνευματικός}{spiritual} 
\MainTextVerseMark{15}{47}ὁ πρῶτος ἄνθρωπος ἐκ γῆς χοϊκός,\FnParseFormGloss{a}{nominative masculine singular}{χοϊκός}{made of dust} ὁ δεύτερος ⸀ἄνθρωπος ἐξ οὐρανοῦ. 
\MainTextVerseMark{15}{48}οἷος\FnParseFormGloss{a}{nominative masculine singular}{οἷος}{what sort of} ὁ χοϊκός,\FnParseFormGloss{b}{nominative masculine singular}{χοϊκός}{made of dust} τοιοῦτοι καὶ οἱ χοϊκοί,\FnParseFormGloss{c}{nominative masculine plural}{χοϊκός}{made of dust} καὶ οἷος\FnParseFormGloss{d}{nominative masculine singular}{οἷος}{what sort of} ὁ ἐπουράνιος,\FnParseFormGloss{e}{nominative masculine singular}{ἐπουράνιος}{heavenly} τοιοῦτοι καὶ οἱ ἐπουράνιοι·\FnParseFormGloss{f}{nominative masculine plural}{ἐπουράνιος}{heavenly} 
\MainTextVerseMark{15}{49}καὶ καθὼς ἐφορέσαμεν\FnParseFormGloss{a}{aorist active indicative 1st plural}{φορέω}{to wear} τὴν εἰκόνα\FnParseFormGloss{b}{accusative feminine singular}{εἰκών}{image} τοῦ χοϊκοῦ,\FnParseFormGloss{c}{genitive masculine singular}{χοϊκός}{made of dust} ⸀φορέσομεν\FnParseFormGloss{d}{future active indicative 1st plural}{φορέω}{to wear} καὶ τὴν εἰκόνα\FnParseFormGloss{e}{accusative feminine singular}{εἰκών}{image} τοῦ ἐπουρανίου.\FnParseFormGloss{f}{genitive masculine singular}{ἐπουράνιος}{heavenly} 
\MainTextVerseMark{15}{50}Τοῦτο δέ φημι,\FnParse{a}{present active indicative 1st singular} ἀδελφοί, ὅτι σὰρξ καὶ αἷμα βασιλείαν θεοῦ κληρονομῆσαι\FnParseFormGloss{b}{aorist active infinitive}{κληρονομέω}{to inherit} οὐ ⸀δύναται,\FnParse{c}{present middle indicative 3rd singular} οὐδὲ ἡ φθορὰ\FnParseFormGloss{d}{nominative feminine singular}{φθορά}{perishableness} τὴν ἀφθαρσίαν\FnParseFormGloss{e}{accusative feminine singular}{ἀφθαρσία}{imperishableness} κληρονομεῖ.\FnParseFormGloss{f}{present active indicative 3rd singular}{κληρονομέω}{to inherit} 
\MainTextVerseMark{15}{51}ἰδοὺ μυστήριον\FnParseFormGloss{a}{accusative neuter singular}{μυστήριον}{mystery} ὑμῖν λέγω·\FnParse{b}{present active indicative 1st singular} ⸀πάντες οὐ κοιμηθησόμεθα\FnParseFormGloss{c}{future passive indicative 1st plural}{κοιμάομαι}{to fall asleep} πάντες δὲ ἀλλαγησόμεθα,\FnParseFormGloss{d}{future passive indicative 1st plural}{ἀλλάσσω}{to change} 
\MainTextVerseMark{15}{52}ἐν ἀτόμῳ,\FnParseFormGloss{a}{dative neuter singular}{ἄτομος}{in a flash} ἐν ῥιπῇ\FnParseFormGloss{b}{dative feminine singular}{ῥιπή}{twinkling} ὀφθαλμοῦ, ἐν τῇ ἐσχάτῃ σάλπιγγι·\FnParseFormGloss{c}{dative feminine singular}{σάλπιγξ}{trumpet} σαλπίσει\FnParseFormGloss{d}{future active indicative 3rd singular}{σαλπίζω}{to sound a trumpet} γάρ, καὶ οἱ νεκροὶ ἐγερθήσονται\FnParse{e}{future passive indicative 3rd plural} ἄφθαρτοι,\FnParseFormGloss{f}{nominative masculine plural}{ἄφθαρτος}{imperishable} καὶ ἡμεῖς ἀλλαγησόμεθα.\FnParseFormGloss{g}{future passive indicative 1st plural}{ἀλλάσσω}{to change} 
\MainTextVerseMark{15}{53}δεῖ\FnParse{a}{present active indicative 3rd singular} γὰρ τὸ φθαρτὸν\FnParseFormGloss{b}{accusative neuter singular}{φθαρτός}{perishable} τοῦτο ἐνδύσασθαι\FnParseFormGloss{c}{aorist middle infinitive}{ἐνδύω}{to clothe} ἀφθαρσίαν\FnParseFormGloss{d}{accusative feminine singular}{ἀφθαρσία}{imperishableness} καὶ τὸ θνητὸν\FnParseFormGloss{e}{accusative neuter singular}{θνητός}{mortal} τοῦτο ἐνδύσασθαι\FnParseFormGloss{f}{aorist middle infinitive}{ἐνδύω}{to clothe} ἀθανασίαν.\FnParseFormGloss{g}{accusative feminine singular}{ἀθανασία}{immortality} 
\MainTextVerseMark{15}{54}ὅταν δὲ τὸ ⸂φθαρτὸν\FnParseFormGloss{a}{nominative neuter singular}{φθαρτός}{perishable} τοῦτο ἐνδύσηται\FnParseFormGloss{b}{aorist middle subjunctive 3rd singular}{ἐνδύω}{to clothe} ἀφθαρσίαν\FnParseFormGloss{c}{accusative feminine singular}{ἀφθαρσία}{imperishableness} καὶ τὸ⸃ θνητὸν\FnParseFormGloss{d}{nominative neuter singular}{θνητός}{mortal} τοῦτο ἐνδύσηται\FnParseFormGloss{e}{aorist middle subjunctive 3rd singular}{ἐνδύω}{to clothe} ⸀ἀθανασίαν,\FnParseFormGloss{f}{accusative feminine singular}{ἀθανασία}{immortality} τότε γενήσεται\FnParse{g}{future middle indicative 3rd singular} ὁ λόγος ὁ γεγραμμένος·\FnParse{h}{perfect passive participle nominative masculine singular} Κατεπόθη\FnParseFormGloss{i}{aorist passive indicative 3rd singular}{καταπίνω}{to swallow} ὁ θάνατος εἰς νῖκος.\FnParseFormGloss{j}{accusative neuter singular}{νῖκος}{victory} 
\MainTextVerseMark{15}{55}ποῦ σου, θάνατε, τὸ ⸂νῖκος;\FnParseFormGloss{a}{nominative neuter singular}{νῖκος}{victory} ποῦ σου, θάνατε, τὸ κέντρον⸃;\FnParseFormGloss{b}{nominative neuter singular}{κέντρον}{sting} 
\MainTextVerseMark{15}{56}τὸ δὲ κέντρον\FnParseFormGloss{a}{nominative neuter singular}{κέντρον}{sting} τοῦ θανάτου ἡ ἁμαρτία, ἡ δὲ δύναμις τῆς ἁμαρτίας ὁ νόμος· 
\MainTextVerseMark{15}{57}τῷ δὲ θεῷ χάρις τῷ διδόντι\FnParse{a}{present active participle dative masculine singular} ἡμῖν τὸ νῖκος\FnParseFormGloss{b}{accusative neuter singular}{νῖκος}{victory} διὰ τοῦ κυρίου ἡμῶν Ἰησοῦ Χριστοῦ. 
\MainTextVerseMark{15}{58}Ὥστε, ἀδελφοί μου ἀγαπητοί, ἑδραῖοι\FnParseFormGloss{a}{nominative masculine plural}{ἑδραῖος}{firm} γίνεσθε,\FnParse{b}{present middle imperative 2nd plural} ἀμετακίνητοι,\FnParseFormGloss{c}{nominative masculine plural}{ἀμετακίνητος}{not moveable} περισσεύοντες\FnParse{d}{present active participle nominative masculine plural} ἐν τῷ ἔργῳ τοῦ κυρίου πάντοτε, εἰδότες\FnParse{e}{perfect active participle nominative masculine plural} ὅτι ὁ κόπος\FnParseFormGloss{f}{nominative masculine singular}{κόπος}{labor} ὑμῶν οὐκ ἔστιν\FnParse{g}{present active indicative 3rd singular} κενὸς\FnParseFormGloss{h}{nominative masculine singular}{κενός}{empty} ἐν κυρίῳ. 
\ChapterHeader{16}{Chapter 16}

\MainTextVerseMark{16}{1}Περὶ δὲ τῆς λογείας\FnParseFormGloss{a}{genitive feminine singular}{λογεία}{collection} τῆς εἰς τοὺς ἁγίους, ὥσπερ διέταξα\FnParseFormGloss{b}{aorist active indicative 1st singular}{διατάσσω}{to command} ταῖς ἐκκλησίαις τῆς Γαλατίας,\FnParseFormGloss{c}{genitive feminine singular}{Γαλατία}{Galatia} οὕτως καὶ ὑμεῖς ποιήσατε.\FnParse{d}{aorist active imperative 2nd plural} 
\MainTextVerseMark{16}{2}κατὰ μίαν ⸀σαββάτου ἕκαστος ὑμῶν παρ’ ἑαυτῷ τιθέτω\FnParse{a}{present active imperative 3rd singular} θησαυρίζων\FnParseFormGloss{b}{present active participle nominative masculine singular}{θησαυρίζω}{to store up} ὅ τι ⸀ἐὰν εὐοδῶται,\FnParseFormGloss{c}{present passive subjunctive 3rd singular}{εὐοδόομαι}{to get along with} ἵνα μὴ ὅταν ἔλθω\FnParse{d}{aorist active subjunctive 1st singular} τότε λογεῖαι\FnParseFormGloss{e}{nominative feminine plural}{λογεία}{collection} γίνωνται.\FnParse{f}{present middle subjunctive 3rd plural} 
\MainTextVerseMark{16}{3}ὅταν δὲ παραγένωμαι,\FnParse{a}{aorist middle subjunctive 1st singular} οὓς ⸀ἐὰν δοκιμάσητε\FnParseFormGloss{b}{aorist active subjunctive 2nd plural}{δοκιμάζω}{to test} δι’ ἐπιστολῶν,\FnParseFormGloss{c}{genitive feminine plural}{ἐπιστολή}{letter} τούτους πέμψω\FnParse{d}{future active indicative 1st singular} ἀπενεγκεῖν\FnParseFormGloss{e}{aorist active infinitive}{ἀποφέρω}{to carry away} τὴν χάριν ὑμῶν εἰς Ἰερουσαλήμ· 
\MainTextVerseMark{16}{4}ἐὰν δὲ ⸂ἄξιον ᾖ⸃\FnParse{a}{present active subjunctive 3rd singular} τοῦ κἀμὲ πορεύεσθαι,\FnParse{b}{present middle infinitive} σὺν ἐμοὶ πορεύσονται.\FnParse{c}{future middle indicative 3rd plural} 
\MainTextVerseMark{16}{5}Ἐλεύσομαι\FnParse{a}{future middle indicative 1st singular} δὲ πρὸς ὑμᾶς ὅταν Μακεδονίαν\FnParseFormGloss{b}{accusative feminine singular}{Μακεδονία}{Macedonia} διέλθω,\FnParse{c}{aorist active subjunctive 1st singular} Μακεδονίαν\FnParseFormGloss{d}{accusative feminine singular}{Μακεδονία}{Macedonia} γὰρ διέρχομαι,\FnParse{e}{present middle indicative 1st singular} 
\MainTextVerseMark{16}{6}πρὸς ὑμᾶς δὲ τυχὸν\FnParseFormGloss{a}{aorist active participle accusative neuter singular}{τυγχάνω}{to take part in} ⸀παραμενῶ\FnParseFormGloss{b}{future active indicative 1st singular}{παραμένω}{to continue} ἢ ⸀καὶ παραχειμάσω,\FnParseFormGloss{c}{future active indicative 1st singular}{παραχειμάζω}{to spend the winter} ἵνα ὑμεῖς με προπέμψητε\FnParseFormGloss{d}{aorist active subjunctive 2nd plural}{προπέμπω}{to accompany} οὗ\FnFormGloss{e}{οὗ}{where} ἐὰν πορεύωμαι.\FnParse{f}{present middle subjunctive 1st singular} 
\MainTextVerseMark{16}{7}οὐ θέλω\FnParse{a}{present active indicative 1st singular} γὰρ ὑμᾶς ἄρτι ἐν παρόδῳ\FnParseFormGloss{b}{dative feminine singular}{πάροδος}{passing by} ἰδεῖν,\FnParse{c}{aorist active infinitive} ἐλπίζω\FnParse{d}{present active indicative 1st singular} ⸀γὰρ χρόνον τινὰ ἐπιμεῖναι\FnParseFormGloss{e}{aorist active infinitive}{ἐπιμένω}{to stay} πρὸς ὑμᾶς, ἐὰν ὁ κύριος ⸀ἐπιτρέψῃ.\FnParseFormGloss{f}{aorist active subjunctive 3rd singular}{ἐπιτρέπω}{to let} 
\MainTextVerseMark{16}{8}⸀ἐπιμενῶ\FnParseFormGloss{a}{future active indicative 1st singular}{ἐπιμένω}{to stay} δὲ ἐν Ἐφέσῳ\FnParseFormGloss{b}{dative feminine singular}{Ἔφεσος}{Ephesus} ἕως τῆς πεντηκοστῆς·\FnParseFormGloss{c}{genitive feminine singular}{πεντηκοστή}{Pentecost} 
\MainTextVerseMark{16}{9}θύρα γάρ μοι ἀνέῳγεν\FnParse{a}{perfect active indicative 3rd singular} μεγάλη καὶ ἐνεργής,\FnParseFormGloss{b}{nominative feminine singular}{ἐνεργής}{active} καὶ ἀντικείμενοι\FnParseFormGloss{c}{present middle participle nominative masculine plural}{ἀντίκειμαι}{to be an opponent} πολλοί. 
\MainTextVerseMark{16}{10}Ἐὰν δὲ ἔλθῃ\FnParse{a}{aorist active subjunctive 3rd singular} Τιμόθεος,\FnParseFormGloss{b}{nominative masculine singular}{Τιμόθεος}{Timothy} βλέπετε\FnParse{c}{present active imperative 2nd plural} ἵνα ἀφόβως\FnFormGloss{d}{ἀφόβως}{fearlessly} γένηται\FnParse{e}{aorist middle subjunctive 3rd singular} πρὸς ὑμᾶς, τὸ γὰρ ἔργον κυρίου ἐργάζεται\FnParse{f}{present middle indicative 3rd singular} ὡς ⸀κἀγώ· 
\MainTextVerseMark{16}{11}μή τις οὖν αὐτὸν ἐξουθενήσῃ.\FnParseFormGloss{a}{aorist active subjunctive 3rd singular}{ἐξουθενέω}{to treat with contempt} προπέμψατε\FnParseFormGloss{b}{aorist active imperative 2nd plural}{προπέμπω}{to accompany} δὲ αὐτὸν ἐν εἰρήνῃ, ἵνα ἔλθῃ\FnParse{c}{aorist active subjunctive 3rd singular} πρός ⸀με, ἐκδέχομαι\FnParseFormGloss{d}{present middle indicative 1st singular}{ἐκδέχομαι}{to wait for} γὰρ αὐτὸν μετὰ τῶν ἀδελφῶν. 
\MainTextVerseMark{16}{12}Περὶ δὲ Ἀπολλῶ\FnParseFormGloss{a}{genitive masculine singular}{Ἀπολλῶς}{Apollos} τοῦ ἀδελφοῦ, πολλὰ παρεκάλεσα\FnParse{b}{aorist active indicative 1st singular} αὐτὸν ἵνα ἔλθῃ\FnParse{c}{aorist active subjunctive 3rd singular} πρὸς ὑμᾶς μετὰ τῶν ἀδελφῶν· καὶ πάντως\FnFormGloss{d}{πάντως}{surely} οὐκ ἦν\FnParse{e}{imperfect active indicative 3rd singular} θέλημα ἵνα νῦν ἔλθῃ,\FnParse{f}{aorist active subjunctive 3rd singular} ἐλεύσεται\FnParse{g}{future middle indicative 3rd singular} δὲ ὅταν εὐκαιρήσῃ.\FnParseFormGloss{h}{aorist active subjunctive 3rd singular}{εὐκαιρέω}{to have a chance to} 
\MainTextVerseMark{16}{13}Γρηγορεῖτε,\FnParseFormGloss{a}{present active imperative 2nd plural}{γρηγορέω}{to keep watch} στήκετε\FnParseFormGloss{b}{present active imperative 2nd plural}{στήκω}{to stand} ἐν τῇ πίστει, ἀνδρίζεσθε,\FnParseFormGloss{c}{present middle imperative 2nd plural}{ἀνδρίζομαι}{to act courageously} κραταιοῦσθε.\FnParseFormGloss{d}{present passive imperative 2nd plural}{κραταιόομαι}{to be strong} 
\MainTextVerseMark{16}{14}πάντα ὑμῶν ἐν ἀγάπῃ γινέσθω.\FnParse{a}{present middle imperative 3rd singular} 
\MainTextVerseMark{16}{15}Παρακαλῶ\FnParse{a}{present active indicative 1st singular} δὲ ὑμᾶς, ἀδελφοί· οἴδατε\FnParse{b}{perfect active indicative 2nd plural} τὴν οἰκίαν Στεφανᾶ,\FnParseFormGloss{c}{genitive masculine singular}{Στεφανᾶς}{Stephanas} ὅτι ἐστὶν\FnParse{d}{present active indicative 3rd singular} ἀπαρχὴ\FnParseFormGloss{e}{nominative feminine singular}{ἀπαρχή}{firstfruits} τῆς Ἀχαΐας\FnParseFormGloss{f}{genitive feminine singular}{Ἀχαΐα}{Achaia} καὶ εἰς διακονίαν τοῖς ἁγίοις ἔταξαν\FnParseFormGloss{g}{aorist active indicative 3rd plural}{τάσσω}{to appoint} ἑαυτούς· 
\MainTextVerseMark{16}{16}ἵνα καὶ ὑμεῖς ὑποτάσσησθε\FnParse{a}{present passive subjunctive 2nd plural} τοῖς τοιούτοις καὶ παντὶ τῷ συνεργοῦντι\FnParseFormGloss{b}{present active participle dative masculine singular}{συνεργέω}{to work together} καὶ κοπιῶντι.\FnParseFormGloss{c}{present active participle dative masculine singular}{κοπιάω}{to work} 
\MainTextVerseMark{16}{17}χαίρω\FnParse{a}{present active indicative 1st singular} δὲ ἐπὶ τῇ παρουσίᾳ\FnParseFormGloss{b}{dative feminine singular}{παρουσία}{presence} Στεφανᾶ\FnParseFormGloss{c}{genitive masculine singular}{Στεφανᾶς}{Stephanas} καὶ Φορτουνάτου\FnParseFormGloss{d}{genitive masculine singular}{Φορτουνᾶτος}{Fortunatus} καὶ Ἀχαϊκοῦ,\FnParseFormGloss{e}{genitive masculine singular}{Ἀχαϊκός}{Achaicus} ὅτι τὸ ⸀ὑμέτερον\FnParseFormGloss{f}{accusative neuter singular}{ὑμέτερος}{your} ὑστέρημα\FnParseFormGloss{g}{accusative neuter singular}{ὑστέρημα}{what is lacking} οὗτοι ἀνεπλήρωσαν,\FnParseFormGloss{h}{aorist active indicative 3rd plural}{ἀναπληρόω}{to fulfill} 
\MainTextVerseMark{16}{18}ἀνέπαυσαν\FnParseFormGloss{a}{aorist active indicative 3rd plural}{ἀναπαύω}{to rest} γὰρ τὸ ἐμὸν πνεῦμα καὶ τὸ ὑμῶν. ἐπιγινώσκετε\FnParse{b}{present active imperative 2nd plural} οὖν τοὺς τοιούτους. 
\MainTextVerseMark{16}{19}Ἀσπάζονται\FnParse{a}{present middle indicative 3rd plural} ὑμᾶς αἱ ἐκκλησίαι τῆς Ἀσίας.\FnParseFormGloss{b}{genitive feminine singular}{Ἀσία}{Asia} ⸀ἀσπάζεται\FnParse{c}{present middle indicative 3rd singular} ὑμᾶς ἐν κυρίῳ πολλὰ Ἀκύλας\FnParseFormGloss{d}{nominative masculine singular}{Ἀκύλας}{Aquila} καὶ ⸀Πρίσκα\FnParseFormGloss{e}{nominative feminine singular}{Πρίσκα}{Prisca} σὺν τῇ κατ’ οἶκον αὐτῶν ἐκκλησίᾳ. 
\MainTextVerseMark{16}{20}ἀσπάζονται\FnParse{a}{present middle indicative 3rd plural} ὑμᾶς οἱ ἀδελφοὶ πάντες. ἀσπάσασθε\FnParse{b}{aorist middle imperative 2nd plural} ἀλλήλους ἐν φιλήματι\FnParseFormGloss{c}{dative neuter singular}{φίλημα}{kiss} ἁγίῳ. 
\MainTextVerseMark{16}{21}Ὁ ἀσπασμὸς\FnParseFormGloss{a}{nominative masculine singular}{ἀσπασμός}{greeting} τῇ ἐμῇ χειρὶ Παύλου. 
\MainTextVerseMark{16}{22}εἴ τις οὐ φιλεῖ\FnParseFormGloss{a}{present active indicative 3rd singular}{φιλέω}{to love} τὸν ⸀κύριον, ἤτω\FnParse{b}{present active imperative 3rd singular} ἀνάθεμα.\FnParseFormGloss{c}{nominative neuter singular}{ἀνάθεμα}{curse} ⸂Μαράνα\FnParseFormGloss{d}{masculine singular}{μαράνα}{maranatha} θά⸃.\FnParseFormGloss{e}{aorist active imperative 2nd singular}{θά}{come} 
\MainTextVerseMark{16}{23}ἡ χάρις τοῦ κυρίου ⸀Ἰησοῦ μεθ’ ὑμῶν. 
\MainTextVerseMark{16}{24}ἡ ἀγάπη μου μετὰ πάντων ὑμῶν ἐν Χριστῷ ⸀Ἰησοῦ. 
\end{document}